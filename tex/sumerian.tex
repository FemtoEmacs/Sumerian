\documentclass[a4paper,12pt]{book}
\usepackage[english]{babel}
\usepackage[utf8]{inputenc}
\usepackage{graphicx}
\usepackage{menukeys}
\usepackage{amsmath}
\usepackage{amssymb}
\usepackage[T1]{fontenc}
\usepackage{upquote}
\usepackage{wrapfig}
\usepackage{makeidx}
\usepackage{tikz, pgf}
\usepackage{listing}
\usepackage{fancyvrb}

\makeindex

\title{An introduction to Sumerian Cuneiforms}
\author{Eduardo Costa \and Marcus Santos \and Sergio Teixeira}
\date{}


\usepackage{fontspec}

%%%\setromanfont{Noto Sans}
\newcommand{\fcn}{\setromanfont{Noto Sans Cuneiform}}
\newcommand{\fsm}{\Large\setromanfont{Noto Sans Cuneiform}}
\newcommand{\ttf}{\ttfamily}

\begin{document}
\maketitle
\thispagestyle{empty}

\chapter{Ur-Nammu-9}

The Sumerian cuneiform script was the first
writing system invented by humankind.
Therefore, all educated individuals should
learn this 5,000-year-old script.
In this tutorial, we will learn how to read
and reproduce the writing on the Ur-Nammu 9 Brick.

%% \setromanfont{Noto Sans Cuneiform}

\begin{quotation}\LARGE\fcn
𒀭𒋀𒆠

𒈗𒀀𒉌

𒌨𒀭𒇉

𒈗𒋀𒀊𒆠𒈠𒆤

𒂍𒀀𒉌

𒈬𒈾𒆕

𒂦𒋀𒀊𒆠𒈠

𒈬𒈾𒆕
\end{quotation}
There are few grammar books for Sumerian.
Unfortunately, Marie-Louise Thomsen's
{\bf ``The Sumerian Language"} does not use cuneiform,
so I cannot recommend it. This leaves us with
John Hayes' Manual of Sumerian and Joshua
Bowen's Learn to Read Ancient Sumerian.
Therefore, I advise you
to buy {\bf ``A Manual of Sumerian: Grammar and Texts"}
by Hayes to learn this ancient language in depth.
It is also a good idea to acquire
{\bf ``Learn to Read Ancient Sumerian"}
by Joshua Bowen and Megan Lewis.

\section{Disclaimer}
The authors of this book are not a scholars
in Sumerian studies in any sense.
Therefore, they may not help serious students
of cuneiforms to solve their pendencies
and questions.

For scholars and graduate students who are
writing their thesis, the authors recommend
John Hayes' {\bf Manual of Sumerian} and
Joshua Bowen's {\bf Learn to Read Ancient Sumerian}.
Hayes' manual strong points are inscriptions
and dedicatories, while Bowen and Lewis prefer
literary texts.

\section{Sentence structure}
To discuss grammar, scholars use a transliteration
of Sumerian cuneiforms to the Latin alphabet.
Below, you will find the transliteration of
the Ur-Nammu-9 document that we will study
in this lesson.
\begin{verbatim}
1- [NANNA
2-     LUGAL.ANI].{(R)} #dat           -- For his king
3- [UR-NAMMU                           -- Ur-Nammu,
4-     LUGAL.URIM5.{AK}].{E} #gen/erg  -- the king of UR,
5- [E2.ANI].{Ø}  #object               -- his temple
6- MU.NA.DU3 #verb                     -- he built
7- [BAD3.URIM5.{A(K)}].{Ø} #gen/obj    -- the city wall of Ur
8- MU.NA.DU3  #verb                    -- he built
\end{verbatim}

\section{Grammar functions in transliteration}

In the transliteration, grammar functions are
represented by indicators between braces.
In the example, the grammar functions are:
\begin{description}
\item[1,2] The benefactive has an unwritten R,
  which is represented by \verb|{(R)}|
\item[3,4] The genitive ends in \verb|{AK}| after
  consonant; the ergative ends in \verb|{E}|
\item[5] The object of the action has no ending,
  which is represented by \verb|{Ø}|
\item[7] The genitive has an unwritten K,
  which is represented by \verb|{A(K)}|
\item[8] The verbal chain {\fcn 𒈬𒈾𒆕} (tr mu na du3)
  starts with the ventive prefix {\fcn 𒈬},
  followed by a cross-reference {\fcn 𒈾} (tr na)
  to the dative.
\end{description}

Square brackets delimit a noun chain, i.e.,
a noun followed by a sequence of limiting
qualifiers that may contain adjectives,
apositives and a genitive.
Example: \verb|[UR-NAMMU LUGAL.URIM5.{AK}].{E}|
means
\begin{quote}
\begin{verbatim}
[Ur-Nammu, Ur's king].{task-doer}
\end{verbatim}
\end{quote}
After the close square bracket, a braced symbol
suffix indicates the function of the noun chain.
For instance, \verb|.{E}| shows that
\verb|[UR.NAMMU...].{E}| is the doer of
the sentence's task. The \verb|{(R)}| symbol
shows that \verb|[NANNA...].{(R)}| receives
the benefits of the task:
\verb|[God Nanna].{benefactive}|.

The noun chain may contain a genitive, as was
stated in the previous paragraph. If you don't
know the role of a genitive, it is a grammar
function that shows possession. In English,
the Saxon genitive marks the possessor with
apostrophe-s and comes before the noun:
{\em Ur's king}. In Sumerian, the possessor
follows the noun and is marked with \verb|{AK}|
after consonant and \verb|{K}| after vowel:
\verb|{URIM5 MA].{K}| is equivalent
to {\em Ur's king}.

Braces represent the grammatical function endings.
For instance, the ergative function-ending
represents the doer of the task and is written
as \verb|{E} #erg|, where \verb|#erg| is a comment
that will be omitted in more advanced lessons.
The person who receives the benefit of the
action is called dative and is represented
as \verb|{RA} #dat|, where the \verb|#dat|
comment is usually omitted.

The empty ending of the object is commented
as \verb|{∅} #obj| or simply as \verb|{} #obj|.
In the example, the objects are the constructions
of king Ur-Nammu:
\begin{quote}
\begin{verbatim}
[E2 A NI].{∅}               -- his temple
[[BAD3.URIM5].{A(K)}].{∅}   -- the city wall of Ur
\end{verbatim}
\end{quote}
Unwritten endings are placed between parentheses,
such as \verb|{(R)}|.

\section{Line 1 \& 2}
The Ur-Nammu 9 document is divided into eight lines.\\

\begin{tabular}[!h]{l l l}
\fcn\Large 𒀭 𒋀𒆠
&\fcn\Large 𒈗 &\fcn\Large 𒀀𒉌\\
  $^d$nanna & lugal & a ni\\
\multicolumn{3}{l}{\ttf (tr an nanna lugal a ni)}\\
\multicolumn{3}{l}{\em For the god Nanna, his master,}\\
\hline\\
\multicolumn{3}{l}{{\fcn 𒀭 𒋀𒆠}
                    ($^d$nanna) the god Nanna }\\
\multicolumn{3}{l}{{\fcn 𒈗}
                    (lugal) king, master }\\
\multicolumn{3}{l}{{\fcn 𒀀𒉌}
                    (a ni) his }\\
\end{tabular}

\verb||\\
 In the first line,
the text {\fcn 𒀭𒋀𒆠} is written, which is
the Sumerogram for the name of Nanna, the god of the Moon.
The {\fcn 𒀭} symbol is read as \verb|an|
(or \verb|digir|) and is determinative for deity.
We will learn in the next paragraph that this word
is in the dative case; therefore, the translation
of the rectangle is {\em ``For Nanna."}

Sumerian uses symbols, called determinatives,
to make the meaning clearer. The star {\fcn 𒀭}
in front of a god's name is the determinative
of divinity. In transliteration, the determinatives
are represented as a superscript letter, such
as $^d$\verb|nanna|.

The Emacs command \verb|(tr an nanna lugal a ni)| is used
to typeset Sumerian. There are instructions about
this command on the page where you found this tutorial.

\section{Line 3 \& 4}
The third line of the Ur-Nammu-9 document contains the name of
Ur-Nammu ({\fcn 𒌨𒀭𒇉}), the king who rebuilt the temple
of $^d$Nanna and is the document's author.
The king's name is formed by {\fcn 𒌨} (\verb|ur|),
which means {\em man} or {\em dog},
and {\fcn 𒀭𒇉} ($^d$\verb|nanna|),
the Mother Earth of the Sumerians.
Therefore, the king's name, {\fcn 𒌨𒀭𒇉},
means {\em ``The Man of Nammu."}
Note that the determinative of
deity ({\fcn 𒀭}) precedes the goddess' name.\\

\begin{tabular}[!h]{l l l l l l l l}
\fcn\Large 𒌨𒀭𒇉
&\fcn\Large 𒈗 &\fcn\Large 𒋀𒀊𒆠 &
\fcn\Large 𒈠 & \fcn\Large 𒆤\\
  ur-$^d$nammu & lugal & urim & ma & ke4\\
\multicolumn{5}{l}{\ttf (tr ur nammu lugal urim ma ke4)}\\
\multicolumn{5}{l}{\em Ur-Nammu, the king of Ur,}\\
\hline\\
\multicolumn{5}{l}{{\fcn 𒌨𒀭𒇉}
                    (ur-$^d$nammu) King Ur-Nammu}\\
\multicolumn{5}{l}{{\fcn 𒈗}
                    (lugal) king, master }\\
\multicolumn{5}{l}{{\fcn 𒋀𒀊𒆠}
                    (urim$^{ki}$) the city of Ur }\\
\multicolumn{5}{l}{{\fcn 𒆠}
     (ki) {\em determinative of places} }\\
\multicolumn{5}{l}{{\fcn 𒈠}
     (ma(k)) {\em dative after the consonant ``M''} }\\
\multicolumn{5}{l}{{\fcn 𒆤}
     (ke4) {\em contraction of dative with ergative} }\\
\multicolumn{5}{l}{{\fcn 𒈠𒆤}
     (ma ke4) {\em genitive contracted with ergative} }\\
\end{tabular} 

\verb||\\
The fourth line contains {\fcn 𒈗𒋀𒀕𒆠𒈠𒆤}
(tr lugal urim2 ma ke4), where {\fcn 𒋀𒀕𒆠} (tr urim)
represents the city that was the cult center of Nanna.
It is formed by the Sumerograms (tr shesh) ({\fcn 𒋀})
and (tr unug) ({\fcn 𒀕}).
The Sumerogram {\fcn 𒆠} is the determinative
for geographic names. Determinatives,
such as {\fcn 𒀭} ("digir" - deity)
and {\fcn 𒆠} ("ki" - place), are not pronounced.
Their role is to make the meaning of the word clearer.

The genitive case denotes possession.
Unlike the dative, English has a genitive case,
formed by an apostrophe followed by {\bf ``s."}
In English, one would say, {\bf ``Urim's King."}
In Sumerian, the genitive follows the possessor
and is marked with \verb|{ak}| after consonants
and \verb|{k}| after vowels. In this nominal
chain, the ``\verb|a|" of \verb|{ak}| was
assimilated with the previous consonant,
becoming {\fcn 𒈠} (\verb|ma|).
The Sumerogram {\fcn 𒆤} (\verb|ke4|)
represents the \verb|{k}| of the genitive
and the \verb|{e}| of the ergative.

Sumerian is an ergative language, meaning the agent
of transitive actions is marked. In Sumerian, the
ergative marker is \verb|{e}|. However, the subject
of an intransitive verb, like ``to go" or ``to sleep,"
does not receive the \verb|{e}| that marks the agent,
whom linguists call ergative. Unmarked functions,
such as the Sumerian subject of an intransitive
verb and direct object of a transitive verb,
are called absolutive and said to be marked
with the null symbol \verb|{Ø}|.
In short, for the Sumerians and
modern Basques, if the subject of a sentence
does not perform a task, it cannot be called ergative.

\section{Line 5}
The fifth rectangle introduces the
temple (e2 - {\fcn 𒂍}) that Ur-Nammu built.
The expression {\fcn 𒂍𒀀𒉌} (e2 ani)
means ``{\em his temple}." It is in the absolutive
case and, therefore, receives the null
symbol mark \verb|{Ø}|, a technical way of
saying it does not bear a mark.
The noun chain {\fcn 𒂍𒀀𒉌} (e2 ani) undergoes
the consequences of the task performed.
Thus, it is often called patient, accusative or target.

\verb||\\
\begin{tabular}[!h]{l l l}
\fcn\Large 𒂍
&\fcn\Large 𒀀 &\fcn\Large 𒉌\\
  e2 & a & ni\\
\multicolumn{3}{l}{\ttf (tr e2 a ni)}\\
\multicolumn{3}{l}{\em his temple}\\
\hline\\
\multicolumn{3}{l}{{\fcn 𒂍}
  (e2) house, temple }\\
\multicolumn{3}{l}{{\fcn 𒂍𒈨𒌍} 
                    (e2 me esh-pl) pl. houses, temples }\\
\multicolumn{3}{l}{{\fcn 𒀀𒉌}
                    (a ni) his }\\
\end{tabular}

\newpage
\section{Line 6}
A verbal stem prefixed by a sequence of
particles and possibly followed by a suffix
is called a {\em verbal chain}. The verbal
chain {\fcn 𒈬𒈾𒆕} (mu-na-du3) can be translated
as ``{\em built}."\\

\begin{tabular}[!h]{l l l}
\fcn\Large 𒈬
&\fcn\Large 𒈾 &\fcn\Large 𒆕\\
  mu & na & du3\\
\multicolumn{3}{l}{\ttf (tr mu na du3)}\\
\multicolumn{3}{l}{\em he has built for the god}\\
\hline\\
\multicolumn{3}{l}{{\fcn 𒆕}
                    (du3) to build, to make, to plant }\\
\multicolumn{3}{l}{{\fcn 𒈬}
  (mu) {\em conjugation prefix (CP), ventive prefix,}
            here}\\
\multicolumn{3}{l}{{\fcn 𒈾}
                    (na) {\em cross-references the dative} }\\
\end{tabular}

\verb||\\
The verbal chain of the example has two
prefixes and a stem:\\

\begin{description}
\item[\fcn 𒈬] — Ventive Conjugation Prefix (CP).
  The Ventive CP indicates that the action occurs
  here, close to the speaker.
\item[\fcn 𒈾] — Dimensional Prefix (DP) cross-referencing
  the dative. Sumerian has a DP for each sentence component,
  except the ergative and the absolutive cases.
\item[\fcn 𒆕] — verbal stem, {\em he has built}
\end{description}

\section{Line 7 \& 8}
The noun phrase {\fcn 𒂦𒋀𒀕𒆠𒈠}
(tr bad3 urim ma) means ``{\em wall of Ur}."
The sumerogram {\fcn 𒂦} (tr bad3) means
``{\em city wall}." The \verb|{(k)}| of the
genitive is omitted, meaning it is not expressed
because it was not pronounced at
the end of a nominal phrase.\\

\verb||\\
\begin{tabular}[!h]{l l l l l l l l}
\fcn\Large 𒂦
&\fcn\Large 𒋀𒀕𒆠 &\fcn\Large 𒈠 &
\fcn\Large 𒈬 & \fcn\Large 𒈾
& \fcn\Large  𒆕\\
  bad3 & urim & ma & mu & na & du3\\
\multicolumn{6}{l}{\ttf (tr bad3 urim ma mu na du3)}\\
\multicolumn{6}{l}{\em the city wall of Ur, he has built}\\
\hline\\
\multicolumn{5}{l}{{\fcn 𒊏}
                    (ra) ra, {\em dative ending}}\\
\multicolumn{5}{l}{{\fcn 𒈾}
                    (na) {\em reference to dative} }\\
\end{tabular} 

\section{Reading the brick}
Let's read the whole brick inscription.

\begin{enumerate}
\item (tr an nanna) ({\fcn 𒀭𒋀𒆠 }) {\bf\em-- For the god Nanna...}
\item (tr lugal ani) ({\fcn 𒈗𒀀𒉌}) {\bf\em -- his master,}
  // The word `lugal' means king or master. It is formed
  from `lu2,' ({\fcn 𒇽}) which means `man,'
  and `gal,' ({\fcn 𒃲}) which can be translated
  as `great.' The expression `a-ni' ({\fcn 𒀀𒉌 })
  is equivalent to the possessive pronoun `his.'
\item (tr ur-nammu) ({\fcn 𒌨𒀭𒇉}) {\bf\em -- Ur-Nammu,}
\item (tr lugal urim ki ma ke4) ({\fcn 𒈗𒋀𒀊𒆠𒈠𒆤})
  {\bf\em -- the king of Ur,}
\item (tr e2 a ni) ({\fcn 𒂍𒀀𒉌}) {\bf\em -- his temple,}
  // Remember that you already learned the
  meaning of `a ni.'
\item (tr mu na du3) ({\fcn 𒈬𒈾𒆕}) {\bf\em -- he has built.}
\item (tr bad3 urim ma) ({\fcn 𒂦𒋀𒀕𒆠𒈠})
  {\bf\em -- The wall of Ur,}
\item (tr mu na du3) ({\fcn 𒈬𒈾𒆕})
  {\bf\em -- he built for Nanna.}
\end{enumerate}

\section{Translation}
The meaning of the whole document is something
like this: {\bf\em``For the god Nanna, his Master, Ur-Nammu, the King of Ur, built his temple. The king also built the city walls of Ur for Nanna."}

\section{The method}
I will use the method I employed in this first
chapter to introduce a few other documents.
In other words, each chapter will contain
grammar, vocabulary, syllables,
and essential Sumerograms for reading
a Sumerian document. This methodology ensures
you can handle a manageable amount of information
initially.

After discussing how to read a Sumerian inscription,
each chapter contains an in-depth presentation
of the Sumerian grammar. Initially,
you can do without reading this final
grammar section. You can return
to it after practicing Sumerian with a few inscriptions.


\section{APPENDIX 1: Grammar notes}

Congratulations. You have finished the first lesson.
This appendix gives further details about the
case elements, the noun chain and the verbal chain.
If you don't feel like it, you don't need to read
it now. You can return to this lesson after completing
a few Sumerian documents.

\section*{Case elements}
The subject of a sentence is the topic of the conversation.
Besides the subject, the sentence may have other marked
components called case elements. Case elements may have
references in the verbal chain. The leading case elements
with their marks and references are:

\subsection*{Ergative: {\fcn 𒂊} \{e\} task doer}

{\fcn\Large 𒈗𒂊  𒂦 𒋀𒀕𒆠𒈠  𒈬𒈾𒆕}\\
(tr lugal e bad3 urim ma mu na du3)\\
The king built the city wall of Ur.

\subsection*{Dative: {\fcn 𒊏} 
\{ra\}/ {\fcn 𒈾} (-na-) for}

{\fcn\Large 𒎏𒀀𒉌𒊏𒈗𒂊𒂦𒋀𒀊𒆠𒈠𒈬𒈾𒆕}\\
(tr nin a ni ra lugal e bad3 urim ma mu na du3)\\
The king built the wall of Ur for his lady.

\subsection*{Locative: {\fcn 𒀀}  \{a\}
/{\fcn 𒉌}  (-ni-) in, on}

{\fcn\Large 𒈗𒂊𒌷𒀀𒂍𒈬𒉌𒆕}\\
(tr lugal e uru a e2 mu ni du3)\\
The king built a house in the city.

\subsection*{Terminative: {\fcn 𒂠}
  \{še\}/{\fcn 𒅆}  (-ši-) in order to}

{\fcn\Large 𒂷𒂊𒌷𒈬𒂠𒂵𒅆𒁺}\\
(tr ĝe26 e uru ĝu10-my she-goal ga shi ĝen)\\
I will go there to my city.


\subsection*{Ablative: {\fcn 𒋫}  \{ta\}/ {\fcn 𒋫}
   (-ta-) or {\fcn 𒊏} (-ra-) out of}

{\fcn\Large 𒌷𒋫𒁀𒋫𒁺}\\
(cn uru ta ba ta ĝen)\\
He went out from the city.

\subsection*{Comitative: {\fcn 𒁕}  \{da\}
  / {\fcn 𒁕}  (-da-)  with}

{\fcn\Large 𒈗𒂊𒌉𒀀𒉌𒁕𒂍𒈬𒌦𒁕𒆕}\\
(tr lugal e dumu a ni da e2 mu un da du3)\\
The king built the house with his son.

\subsection*{Equitative: {\fcn 𒁶} \{gin\}
  / {\fcn 𒁶}  (-gin-) like, as}

{\fcn 𒀀𒁀𒋀𒈬𒁶}\\
(tr a ba shesh ĝu10-my gin-equitative)\\
Who is like my brother?

\subsection*{Absolutive: $\emptyset$}

{\fcn\Large 𒎏𒀀𒉌𒊏𒈗𒂊𒂦𒈬𒈾𒆕}\\
(tr nin a ni ra lugal e bad3 mu na du3)\\
For his lady, the king has built the city wall.

\section*{Dative conjugation}
When used as a prefix to a verb, the dative takes
different forms depending on the person and number
it is referring to.

\subsection*{{\fcn 𒈠} (-ma-) to me}

{\fcn\Large 𒂷𒊏𒈗𒂊𒂍𒈬𒈠𒆕}\\
(tr ĝe26 ra lugal e e2 mu ma du3)\\
The king built a house for me.

\subsection*{{\fcn 𒊏} (-ra-) to you}

{\fcn\Large 𒍢𒊏𒈗𒂊𒂍𒈬𒊏𒆕}\\
(tr ze2 ra lugal e e2 mu ra du3)\\
The king has built a house for you.

\subsection*{{\fcn 𒈾} (-na-) to him/to her}

{\fcn\Large 𒎏𒊏𒈗𒂊𒂍𒈬𒈾𒆕}\\
(tr nin ra lugal e e2 mu na du3)\\
The king has built a house for the lady.

\subsection*{{\fcn 𒈨} (-me-) to us}

{\fcn\Large 𒈗𒂊𒂍𒈬𒈨𒆕}\\
(tr lugal e e2 mu me du3)\\
The king has built a house for us.

\subsection*{{\fcn 𒉈} (-ne-) to them}
{\fcn\Large 𒈗𒂊𒂍𒈬𒉈𒆕}\\
(tr lugal e e2 mu ne du3)\\
The king has built a house for tem.

\section*{Transitive verbs}
A transitive verb describes an action that
transitions from a subject to a direct object.
In a transitive verb, the subject is the doer
of the action and is called ergative, which
is the Greek term for the person who performs a task.
In Sumerian, the ergative is marked
with {\fcn 𒂊} \verb|{e}|.

The absolutive case is the entity that undergoes
the consequences of a task. The absolutive case
can be the person accused of a deed. In this case,
it is called accusative.

The absolutive case can also be a target of a shooting.
Or it can be the object of health care, in which case
it is called patient by the doctors.

Some linguists call {\em patient} all kinds of
absolutive cases of a transitive verb, while others
prefer the term accusative. In Sumerian,
the absolutive case receives no mark, but the
linguists say it is marked by the
null symbol \verb|{Ø}|.

The transitive verb itself comes last in a Sumerian
sentence, and is described by a chain of affixes
surrounding the stem. This verbal chain may contain
a Modal Prefix (MP, such as {\fcn 𒉡} • nu • not),
a Conjugation Prefix (CP, such as {\fcn 𒈬}
• mu • {\em ventive}, here), initial pronominal
prefix (IPP, such as N in {\fcn 𒈬𒌦𒆪𒂊}
• mu-n.dab.e • he seizes her) and suffix
pronouns ({\fcn 𒂗𒉈𒂗} • en-de3-en
• us, {\fcn 𒌦𒍢𒂗} • un-ze2-en • you people).
Below, there are examples of all initial
pronominal prefixes.

\section*{Initial Pronominal Prefixes (IPP)}
In the verbal chain, the Initial Pronominal
Prefixes (IPP) come after the Conjugation
Prefix (CP) that is {\fcn 𒈬} (-mu-) in the examples below. 
The {\fcn 𒈬} (-mu-) prefix is the ventive,
i.e., it shows that the
action is performed towards the speaker.
English uses different verbs for the
{\em andative} (motion away from the speaker)
and the {\em ventive} (motion towards the speaker):
{\em to take away / to bring}, {\em to go / to come}, etc.
Sumerian gets the same effect by adding
the {\em ventive} Conjugation Prefix (CP)
to the verbal chain.


\subsection*{{\fcn\Large 𒈬𒆪𒂊} (mu'dab.e)}
He seizes me.

\subsection*{{\fcn\Large 𒈬𒂊𒆪𒂊} (mu e dab e)}
He seizes you.

\subsection*{{\fcn\Large 𒈬𒌦𒆪𒂊} (mu un dab e)}
He seizes her.

\subsection*{{\fcn\Large 𒈬𒈨𒆪𒂊} (mu me dab e)}
He seizes us.

\subsection*{{\fcn\Large 𒈬𒌦𒉈𒆪𒂊} (mu un ne dab e)}
He seizes them.

\verb||\\
I have for you a complete example of a transitive
sentence below. I provide  a pronunciation
key and vocabulary, so I hope you can
scan the sentence.\\

\verb||\\
{\fcn\Large 𒊩𒊏𒇽𒂊𒊺𒌷𒀀𒈬𒈾𒀊𒋧𒂊}\\
(tr munus ra lu2 e she uru a mu na ab shum2 e)

\verb||\\
\begin{tabular}[!h]{l l l l l l l l}
  \fcn\Large 𒊩𒊏 &\fcn\Large 𒇽𒂊 &\fcn\Large 𒊺
  &\fcn\Large 𒌷𒀀 &\fcn\Large 𒈬𒈾𒀊𒋧𒂊\\
  munus ra & lu2 e & she & uru a & mu na ab shum2 e\\
  for the woman & the man & barley & in the city
  & he gave it to her\\
\end{tabular}

The translation of the sentence is: {\em
The man gave barley to the woman in the city.}
The person who receives the barley is
marked with the dative {\fcn 𒊏} \verb|{ra}|;
the doer of the action has the ergative
marker {\fcn 𒂊} \verb|{e}|, and the
place of the occurrence has the locative
marker {\fcn 𒀀} \verb|{a}|.\\
\verb||\\
{\fcn\Large 𒊩} • (munus) woman, female\\
\verb||\\
{\fcn\Large 𒊏} • (ra) {\em dative marker}\\
\verb||\\
{\fcn\Large 𒇽} • (lu2) man, male\\
\verb||\\
{\fcn\Large 𒂊} • (e) {\em ergative marker}\\
\verb||\\
{\fcn\Large 𒊺} • (še) barley, grain\\
\verb||\\
{\fcn\Large 𒌷} • (uru) city\\
\verb||\\
{\fcn\Large 𒀀} • (a) {\em locative marker}\\
\verb||\\
{\fcn\Large 𒈬} • (mu) {\em venitive conjugation prefix}, here\\
\verb||\\
{\fcn\Large 𒈾} • (na) {\em cross-reference to
the dative}, to her\\
\verb||\\
{\fcn\Large 𒀊} • (ab) {\em Initial Prefix Pronoun}, it\\
\verb||\\
{\fcn\Large 𒋧} • (shum2) to give\\

\section*{Intransitive verb}
An intransitive verb does not have a direct object.
In Sumerian, the subject of an intransitive verb
goes to the absolutive case and, therefore, is not marked.\\

\verb||\\
{\fcn\Large 𒈗 𒌷𒈬𒂠 𒉌𒅎𒁺}\\
(tr lu2 uru ĝu10-my she-goal i3 im ĝen)\\
\begin{tabular}[!h]{l | l | l | l | l l l l}
  \fcn\Large 𒈗 &\fcn\Large 𒌷𒈬𒂠
  &\fcn\Large 𒉌  𒅎
  &\fcn\Large 𒁺\\
  lu2 & uru ĝu10 she-goal & i3  im & ĝen\\
  the king & to my city
  &\em finite verb prefix & came\\
\end{tabular}

The translation of the above example is:
{\em The king came to my city.} You find
the vocabulary necessary to scan this
example below.\\
\verb||\\
{\fcn\Large 𒈗} • (lugal) king \\
\verb||\\
{\fcn\Large 𒉌𒅎} • (im) {\em finite verb marker} \\
\verb||\\
{\fcn\Large 𒁺} • (g̃en) to come\\
\verb||\\
{\fcn\Large 𒂠} • (še3) to, towards\\
\verb||\\
{\fcn\Large 𒌷} • (uru) city\\
\verb||\\
{\fcn\Large 𒌷𒈬} • (uru.ĝu10) my city\\
\verb||\\
{\fcn\Large 𒌷𒈬𒂠} • (uru ĝu10 she-goal) to my city\\
\verb||\\

\section*{Modal Prefix (MP)}
The modal prefixes express modality, i.e.,
relationships to reality or truth.
You can only learn the indicative and negation
modal prefixes for now. You may learn the other
prefixes when you encounter them in Sumerian documents.\\

\subsection*{Indicative: ($\emptyset$-)}
In Sumerian, the indicative is unmarked.
The empty prefix \verb|/Ø-/| may represent
this fact in transliteration. However,
people rarely show unmarked prefixes.

\subsection*{Negation: {\fcn\Large 𒉡} /nu-/}

{\fcn\Large 𒉡𒌦𒅥}\\
(tr nu un gu7)\\
He didn't eat it.

\subsection*{Let him: {\fcn\Large 𒃶} hhe2-}
{\fcn\Large 𒃶𒅁𒅥𒂊}\\
(tr hhe2 ib gu7 e)\\
Let him eat it.

\subsection*{Indeed: {\fcn\Large 𒄩} hha-}
{\fcn\Large 𒄩𒀭𒅥}\\
(tr hha an gu7)\\
He ate it, indeed.

\subsection*{Cohortative: {\fcn\Large 𒂵} ga-}
{\fcn\Large 𒂵𒉌𒌈𒃻𒊑𒂗𒉈𒂗}\\
(tr ga i3 ib2 ĝar re en ne en)\\
Let us put it there.

\subsection*{Prohibitive: {\fcn\Large 𒈾} na-}
{\fcn\Large 𒈾𒀊𒅥𒂊}\\
(tr na ab gu7 e)\\
He must not eat it.

\section*{Conjugation Prefix (CP)}
The main Conjugation Prefixes (CP) are \verb|/mu-/|
to indicate that the action occurs here,
\verb|/bi2-/| in front of open vowels
such as \verb|/i/|, \verb|/ba/| to form
middle/passive voice, \verb|/i3/| to create
a finite verbal tense, and \verb|/ma/| in
combination with \verb|/ra/| of benefit.
Below, you will find a fairly complete list
of Conjugation Prefixes, but you need
to learn only \verb|/mu-/| and \verb|/i3/|
for this first lesson.

\subsection*{1. Here: {\fcn\Large 𒈬} -
  {\fcn\Large 𒌦𒁺}}
(tr mu un re6)\\
He brought it here.

\subsection*{2. Followed by open vowel:
  {\fcn\Large 𒉈} - {\fcn\Large 𒅔𒁺}}
(tr bi2 in re6)\\
He made the team bring it.

\subsection*{3. Followed by ra:
  {\fcn\Large 𒈠} - {\fcn\Large 𒊏𒀭𒁺}}
(tr ma ra an re6)\\
He brought it here to you.

\subsection*{4. Finite verb: {\fcn\Large 𒉌}
  - {\fcn\Large 𒅎𒁺}}
(tr i3 im ĝen)\\
He came here.\\

\subsection*{5. Middle voice: {\fcn\Large 𒁀}
  - {\fcn\Large 𒀭𒁺}}
(tr ba an re6)\\
He took it for himself.\\
{\em Obs. The middle voice with {\ttf /ba/}
  indicates an action that affects the doer.}

\subsection*{6. Passive voice: {\fcn\Large 𒁀}
  - {\fcn\Large 𒁺}}
(tr ba re6)\\
It was brought.

\section*{Nominal chain}
In Sumerian, most adjectives are formed from
verbs by adding the suffix {\fcn 𒀀} \verb|{a}|.
For example, the verb below means to be strong.

\subsection*{{\fcn\Large 𒆗} • (kalag) to be strong}
To form an adjective from kalag, one adds an \verb|{a}|.
In Sumerian, different from English, the adjectives
follow the noun.\\

\verb||\\
\verb||\\

The expression below means {\bf\it mighty king}.
Note that the adjective follows the verb,
and the {\fcn 𒀀} marker contracts with
the previous consonant to form the
{\fcn 𒂵} (ga) syllable.

\subsection*{\fcn\Large 𒈗 𒆗 𒂵}
(tr lugal kalag ga)\\
a mighty king\\

In English, the Saxon genitive is marked
with S and precedes the verb. Therefore,
one writes ``{\bf\it Elil's Warrior}."
In Sumerian, the genitive is marked
with \verb|{k}| after a vowel
and \verb|{ak}| after a consonant.
Like the adjective, the genitive follows the noun.
The \verb|{k}| of the genitive was rarely
written except when combined with the ergative.
In this case, it was written
as {\fcn 𒆤} \verb|{ke4}|.\\

Below, there is another example of
the adjective {\fcn 𒀀} \verb|{a}|
marker contracting with the previous
consonant to form an open syllable.

\subsection*{\fcn\Large 𒂍𒈗𒆷}
(tr e2 lugal la)\\
the king's house\\

\verb||\\
Now, let us examine a somewhat longer
example of a noun chain.\\

\verb||\\
{\fcn\Large
𒂼𒀀𒉌𒊏\\
𒌉𒈗𒆷𒆤\\
𒂍\\
𒈬𒈾𒆕}\\
\verb||\\
(tr ama a ni ra) for his mother,\\
(tr dumu lugal la ke4) by the king's son\\
(tr e2) a house\\
(tr mu na du3) was built for her\\

\chapter{Inscription in Inanna's temple}

\begin{quotation}\LARGE\fcn
  𒀭𒈹 𒎏𒀀𒉌
  
  𒌨𒀭𒇉
  
  𒍑𒆗𒂵
  
  𒈗𒋀𒀊𒆠𒈠
  
  𒈗𒆠𒂗𒄀 𒆠𒌵𒆤
  
  𒂍𒀀𒉌
  
𒈬𒈾𒆕
\end{quotation}

Translation:
{\bf\em For Inanna, his lady, Ur-Nammu,
  the mighty man, the king of Ur,
  the king of Sumer and Akad, built her temple.}

\section{Sentence structure}
\begin{verbatim}
1- [inanna nin a ni].{(r)}             -- For Inanna, his Lady,
2- [ur-nammu                           -- Ur-Nammu,
3-   [nita kalag].{a}                  -- the mighty man,
4-   [lugal urim ma].{(k)}             -- the king of Ur,
5-   [lugal ki-en-gi ki uri].{k}].{e}  -- the king of Sumer and Akkad,
6- [e2 a ni].{}                        -- her (Inanna's) temple
7- mu na du3                           -- built.
\end{verbatim}

From now on, the sentence structure
will not contain the comments \verb|.{k} #gen|
for the genitive, \verb|.{r} #dat| for the dative
or \verb|.{e} #erg| for the ergative (doer of the task).
The suffixes \verb|.{r}| for the dative,
\verb|.{k}| for the genitive
and \verb|.{e}| for the ergative
should suffice for showing the grammatical
function of the noun chain and its components.
However, functional suffixes you didn't learn
in the previous lessons will be commented on.

\section{Annotations}

{\LARGE\fcn 𒀭𒈹 𒎏𒀀𒉌}\\
\begin{tabular}[!h]{l l l l l l l}
  \Large\fcn 𒀭 &\Large\fcn 𒈹
  &\Large\fcn 𒎏 &\Large\fcn 𒀀𒉌\\
  an & inanna & nin & a ni\\
  \multicolumn{4}{l}{\ttf (tr an inanna nin a ni)}\\
  \multicolumn{4}{l}{\em For Inanna, his lady,}\\
  \hline\\
  \multicolumn{4}{l}{{\fcn 𒀭𒈹} • (dinana) Inanna}\\
  \multicolumn{4}{l}{{\fcn 𒎏} • (nin) lady, queen, mistress}\\
  \multicolumn{4}{l}{{\fcn 𒀀𒉌} • (a ni) his, her}\\
\end{tabular}\verb||\\

This noun phrase ends in an unwritten \verb|{(r)}|,
the dative marker. However, there is no ambiguity
since the verb chain has a dative reference.\\

\verb||\\
{\LARGE\fcn 𒌨𒀭𒇉 𒍑𒆗𒂵}\\
\begin{tabular}[!h]{l l l l l l l}
  \Large\fcn 𒌨𒀭𒇉 &\Large\fcn 𒍑
  &\Large\fcn 𒆗 &\Large\fcn 𒂵\\
  ur-nammu & nita & kalag & ga\\
  \multicolumn{4}{l}{\ttf (tr ur-nammu nita kalag ga)}\\
  \multicolumn{4}{l}{\em Ur-Nammu, the mighty man,}\\
  \hline\\
  \multicolumn{4}{l}{{\fcn 𒍑} • (nita) man, male}\\
  \multicolumn{4}{l}{{\fcn 𒆗} • (kalag)
    to be strong, to be mighty}\\
  \multicolumn{4}{l}{{\fcn 𒆗 𒂵} • (kalag ga)
    {\em adj. from verb}, mighty}\\
\end{tabular}\verb||\\

One may form adjectives by adding a nominalizing
\verb|{a}|-particle to a verbal root,
     {\em kalag} in the present expression.
     The nominalizing particle contracts with
     the preceding word's final \verb|g|,
     giving extra information about its correct reading.
     Different from English, Sumerian adjectives
     follow the noun they modify.\\

 \verb||\\
     
 \verb||\\
 {\LARGE\fcn 𒈗𒋀𒀊𒆠𒈠}\\
 \begin{tabular}[!h]{l l l l l l l l l}
   \Large\fcn 𒈗 &\Large\fcn 𒋀𒀊𒆠
   &\Large\fcn 𒈠\\
   lugal & urim & ma\\
   \multicolumn{3}{l}{\ttf (tr lugal urim ma)}\\
   \multicolumn{3}{l}{\em the king of Ur,}\\
 \end{tabular}\verb||\\

 \verb||\\
 \verb||\\
 As we learned from text 1, the genitive is
 formed by \verb|{k}| after vowels
 and \verb|{ak}| after consonants.
 The scribe often omitted the \verb|{(k)}|
 of \verb|{ak}|. The ``{\em m}'' of
 ``{\em ma}'' is contamination from the final
 consonant of the previous word.

\verb||\\
 {\LARGE\fcn 𒈗𒆠𒂗𒄀𒆠𒌵𒆤}\\
 \begin{tabular}[!h]{l l l l l l l l l}
   \Large\fcn 𒈗 &\Large\fcn 𒆠𒂗𒄀
   &\Large\fcn 𒆠𒌵 &\Large\fcn 𒆤 \\
   lugal & ki-en-gi & ki uri & ke4\\
   \multicolumn{4}{l}{\ttf (tr lugal ki-en-gi ki uri ke4)}\\
   \multicolumn{4}{l}{\em the king of Sumer and Akkad,}\\
   \hline\\
     \multicolumn{4}{l}{{\fcn 𒆠𒂗𒄀} • (ki-en-gi) Sumer }\\
  \multicolumn{4}{l}{{\fcn 𒆠𒌵} • (ki-uri) Akkad }\\
 \end{tabular}\verb||\\

 \verb||\\
 In ke4 ({\fcn 𒆤}), the \verb|{k}| is the
 genitive marker, and the \verb|{e}| is the ergative marker.
 You already saw the analysis of the last two
 lines in lesson 1, therefore they should pose
 no difficulty to you.
 
 \verb||\\
 {\LARGE\fcn 𒂍𒀀𒉌}\\
 \verb||\\
\begin{tabular}[!h]{l l l}
\fcn\Large 𒂍
&\fcn\Large 𒀀 &\fcn\Large 𒉌\\
  e2 & a & ni\\
\multicolumn{3}{l}{\ttf (tr e2 a ni)}\\
\multicolumn{3}{l}{\em his temple}\\
\hline\\
\multicolumn{3}{l}{{\fcn 𒂍}
  (e2)  house, temple}\\
\multicolumn{3}{l}{{\fcn 𒂍𒈨𒌍}
                    (e2 me esh-pl) houses, temples }\\
\multicolumn{3}{l}{{\fcn 𒀀𒉌}
                    (a ni) his }\\
\end{tabular}

\verb||\\
\verb||\\
\verb||\\
\verb||\\
 {\LARGE\fcn 𒈬𒈾𒆕}\\
\verb||\\
\begin{tabular}[!h]{l l l}
\fcn\Large 𒈬
&\fcn\Large 𒈾 &\fcn\Large 𒆕\\
  mu & na & du3\\
\multicolumn{3}{l}{\ttf (tr mu na du3)}\\
\multicolumn{3}{l}{\em he has built for the god}\\
\hline\\
\multicolumn{3}{l}{{\fcn 𒆕}
                    (du3) to build, to make, to plant }\\
\multicolumn{3}{l}{{\fcn 𒈬}
  (mu) {\em conjugation prefix (CP), ventive prefix,}
            here}\\
\multicolumn{3}{l}{{\fcn 𒈾}
                    (na) {\em cross-references the dative} }\\
\end{tabular}

\section*{APPENDIX 2: Conjugation}

Congratulations again! You have finished the second
lesson. As in the first lesson, this appendix
details Sumerian pronouns and verbs.
You can return to it after completing
the fifth lesson to gain an in-depth
understanding of possessive pronouns,
independent pronouns,
interrogative pronouns and verb conjugation.

\subsection*{Possessive Pronouns}
In the first lesson, you found two instances
of a possessive pronoun in the
expressions {\fcn 𒈗𒀀𒉌} (tr lugar ani) ``his master'',
and {\fcn 𒌷𒈬} (tr uru ĝu10-my) ``my city''.
Below, I've included a complete list of possessive pronouns.

\subsection*{{\Large\fcn 𒈬}  (g̃u10) my}

\subsection*{{\Large\fcn 𒍪}  (zu) thy}

\subsection*{{\Large\fcn 𒀀𒉌}  (a-ni)  his/her}

\subsection*{{\Large\fcn 𒁉}  (bi, be2) its}

\subsection*{{\Large\fcn 𒈨}  (me) our}

\subsection*{{\Large\fcn 𒍪𒉈𒉈}  (zu-ne-ne) your}

\subsection*{{\Large\fcn 𒀀𒉈𒉈} (a-ne-ne) their}

\section*{Independent pronouns}
Sumerian has a set of independent pronouns
that I advise you to learn right away.
They are very important.

\subsection*{{\Large\fcn 𒂷}  (g̃e26) I/me}

\subsection*{{\Large\fcn 𒍢}  (ze2) thou/thee}

Obs. {\Large\fcn 𒍢} (ze2) becomes {\Large \fcn 𒍝} (za)
when followed by the dative {\Large\fcn 𒊏} (ra).

\subsection*{{\Large\fcn 𒀀𒉈}  (a-ne) he/she/him/her}

\subsection*{{\Large\fcn 𒀀𒉈𒉈}  (a-ne-ne) they}
\verb||\\

\subsection*{Ex: {\Large\fcn 𒀀𒉈 𒁾 𒍝𒊏 𒈠𒊏𒀊𒋧𒈬}}

\begin{tabular}[!h]{l l l l l l l l l}
  \Large\fcn 𒀀𒉈 &\Large\fcn 𒁾
  &\Large\fcn 𒍝𒊏 &\Large\fcn 𒈠𒊏𒀊𒋧𒈬\\
  a ne & dab5 & za ra & ma ra ab shum2 mu\\
  he & the tablet & to you & will give\\
  \multicolumn{4}{l}{\em He will give you the tablet.}\\
\end{tabular}

\subsection*{Vocabulary}
{\fsm 𒀀𒉈}  (a-ne) he/she\\
\verb||\\
{\fsm 𒁾}  (dab5) the tablet\\
\verb||\\
{\fsm 𒍝𒊏}  (zara) to you\\
\verb||\\
{\fsm 𒍝}: {\fcn 𒍢} (ze2) followed by {\fcn 𒊏} (ra)
becomes {\fcn 𒍝} (za)\\

\verb||\\
Sometimes, an independent pronoun appears
with an enclitic copula (verb {\em to be})
attached to its end. Below are examples
of all independent pronouns for you to practice.

\verb||\\
{\fsm 𒆪𒇷𒍪𒂷𒈨𒂗}\\
(tr gu5 li zu (your friend)  ĝe26 me en (I am))\\
I am your friend.

\verb||\\
{\fsm 𒆪𒇷𒈬𒍢𒈨𒂗 }\\
(tr gu5 li ĝu10-my (my friend) ze2 me en (you are))\\     
You are my friend.

\verb||\\
{\fsm 𒆪𒇷𒍪𒀀𒉈𒀀𒀭}\\
(tr gu5 li zu (your friend) a ne am3 (she/he is))\\     
She is your friend.

\verb||\\
{\fsm 𒆪𒇷𒍪𒈨𒂗𒉈𒂗}\\
(tr gu5 li zu (your friend) me en ne en (we are))\\
We are your friends.

\verb||\\
{\fsm 𒆪𒇷𒈬𒈨𒂗𒍢𒂗}\\
(tr gu5 li ĝu10-my (my friend) me en ze2 en (you guys))\\
You guys are my friends.

\verb||\\
{\fsm 𒆪𒇷𒍪𒀀𒉈𒉈𒈨𒌍}\\
(tr gu5 li zu (your friends) a ne ne me esh-pl (they are))\\
They are your friends.\\

\subsection*{Ex: {\fsm 𒆪𒇷𒍪𒂷𒈨𒂗}}
\begin{tabular}[!h]{l l l l l l l l l}
\fsm 𒆪𒇷 &\fsm 𒍪 &\fsm 𒂷 &\fsm 𒈨𒂗\\ 
gu5-li & zu & ĝe26 & me en\\
friend & your & I & am\\
\multicolumn{4}{l}{\em I am your friend}\\
\end{tabular}

\subsection*{Vocabulary}
{\fsm 𒆪𒇷} (gu5 li) friend\\
\verb||\\
{\fsm 𒍪}  (zu) thy, your, {\em 2nd-person possessive pronoun}\\
\verb||\\
{\fsm 𒂷𒈨𒂗} (ge26 me en) {\em copula}, I am\\

\section*{Interrogative pronouns}
Sumerians marked yes/no interrogative sentences
only by intonation and possibly by lengthening
the final vowels, like many modern languages,
such as Spanish and Portuguese.
To ask who performed a task, Sumerians used
the interrogative word {\fcn 𒀀𒁀𒀀} (tr a ba a),
as shown below.\\

\verb||\\
{\fsm 𒂍} •  {\fsm 𒀀𒁀𒀀} •  {\fsm 𒅔𒆕}\\
(tr e2 •  a ba a •  in du3)\\
the temple • who • built?\\
Who built the temple?\\

To ask who is something, Sumerians used the
interrogative pronoun {\fcn 𒀀𒁀} (tr a-ba),
as shown in the following example:\\

\verb||\\
{\fsm 𒀀𒁀} • {\fsm 𒀭𒌓} • {\fsm 𒁶}\\
(tr a ba • utu • gin-equitative)\\
Who • Utu • is like?\\
Who is like Utu?\\

\verb||\\
\verb||\\
\verb||\\
In Sumerian, there is no wh-movement
to the beginning of the clause, like
in English and Spanish. Instead, the
interrogative words are placed immediately
before the verb.\\

\verb||\\
{\fsm 𒈗𒂊} • {\fsm 𒀀𒈾} • {\fsm 𒈬𒌦𒀝}\\
(tr lugal e • a na •  mu un ak)\\
the king • what • did he do?\\
What did the king do?\\

\verb||\\
{\fsm 𒌉𒈬} • {\fsm 𒀀𒈾} • {\fsm 𒉡𒍪}\\
(tr dumu ĝu10-my • a na • nu zu)\\
my son • what • does not know?\\
What does my son not know?\\


{\fsm 𒀀𒈾} • {\fsm 𒀀𒀭} • {\fsm 𒉈𒂊}\\
(tr a na • am3 • ne e)\\
what • is • this? \\
What is this?\\

\verb||\\
An exception to the rule of placing the
interrogative word immediately before
the verb occurs in why-questions,
as the example below shows.

\verb||\\
{\fsm 𒀀𒈾𒀸} • {\fsm 𒀀𒀭} • {\fsm 𒉌𒁺}\\
(tr a na ash • am3 • i3 ĝen)\\
what is it • that he came?\\
Why did he come?\\

The expression {\fcn 𒀀𒈾𒀸} (a-na-ash)
that one usually translates as ``why?''
means literally ``what for?''

\section*{Conjugation}
Sumerian verbs have two aspects: the hamtu (perfective)
and the marû (imperfective). For the time being,
you can translate the hamtu as the English present
perfect, and the marû, as the English future.\\

\verb||\\
hamtu: {\fsm 𒈗𒂊𒂦𒈬𒌦𒁺}\\
(tr lugal e bad3 mu un gub)\\
The king has erected a wall here.\\

\verb||\\
marû: {\fsm 𒈗𒂊𒂦𒉌𒁺𒂊}\\
(tr lugal e bad3 i3 gub e)\\
The king will erect a wall.\\

\section*{Hamtu and marû conjugation}

\subsection*{First person}
{\fsm 𒈾𒈬𒁺}\\
(tr na mu gub)\\
hamtu: I have set up a border stone.\\

\verb||\\
{\fsm 𒈾𒉌𒁺𒂗}\\
(tr na i3 gub en)\\
marû: I will set up a stone.

\subsection*{Second person singular}
{\fsm 𒈾𒈬𒂊𒁺}\\
(tr na mu e gub)\\
hamtu: You have set up a stone.\\

\verb||\\
{\fsm 𒈾𒉌𒁺𒂗}\\
(tr na i3 gub en)\\
marû: You will set up a stone.\\

\verb||\\
\verb||\\
\verb||\\

\subsection*{Third person singular (humans)}
{\fsm 𒈾𒈬𒌦𒁺}\\
(tr na mu un gub)\\
hamtu: He has set up a stone.\\

\verb||\\
{\fsm 𒈾𒉌𒁺𒂊}\\
(tr na i3 gub e)\\
marû: He will set up a stone.\\

\subsection*{First person plural}
{\fsm 𒈾𒈬𒁺𒁁𒂗𒉈𒂗}\\
(tr na mu gub be en de3 en)\\
hamtu: We have set up a stone.\\

\verb||\\
{\fsm 𒈾𒉌𒁺𒂗𒉈𒂗}\\
(tr na i3 gub en de3 en)\\
marû: We will set up a stone.\\

\subsection*{Second person plural}
{\fsm 𒈾𒈬𒂊𒁺𒁁𒂗𒍢𒂗}\\
(tr na mu e gub be en ze2 en)\\
hamtu: You have set up a stone.\\

\verb||\\
{\fsm 𒈾𒉌𒁺𒁁𒂗𒍢𒂗}\\
(tr na i3 gub be en ze2 en)\\
marû: You will set up a stone.\\

\verb||\\
\verb||\\
\verb||\\
\subsection*{Third person plural}
{\fsm 𒈾𒈬𒌦𒁺𒁁𒂠}\\
(tr na mu un gub be esh)\\
hamtu: They have set up a stone.\\

\verb||\\
{\fsm 𒈾𒉌𒁺𒁁𒂊𒉈}\\
(tr na i3 gub be e ne)\\
marû: They will set up a stone.\\

\verb||\\
Animals and plants have different pronouns
for the third person singular. Therefore,
in the third person singular, the hamtu aspect
is not the same for humans and animals.

\verb||\\
{\fsm 𒈾𒈬𒌒𒁺}\\
(tr na mu ub gub)\\
It has set up a stone.

\section*{Intransitive verb conjugation}
Intransitive verbs have the same forms
for the hamtu and the marû aspects.
Below is the complete conjugation
of the verb {\fcn 𒁺} (ĝen),
``to go'' (or ``to come'').

\subsection*{Singular}
{\fsm 𒉌𒁺𒂗}\\
(tr i3 ĝen en)\\
I went.

\verb||\\
{\fsm 𒉌𒁺𒂗}\\
(tr i3 ĝen en)\\
You went.

\verb||\\
{\fsm 𒉌𒁺}\\
(tr i3 ĝen)\\
He went.

\subsection*{Plural}
{\fsm 𒉌𒁻𒂗𒉈𒂗}\\
(tr i3 re7 en de3 en)\\
We went.

\verb||\\
{\fsm 𒉌𒁻𒂗𒍢𒂗}\\
(tr i3 re7 en ze2 en)\\
You people went.

\verb||\\
{\fsm 𒉌𒁻𒂠}\\
(tr i3 re7 esh)\\
They went.

\end{document}

