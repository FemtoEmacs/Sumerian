\documentclass[a4paper,12pt]{book}
\usepackage[english]{babel}
\usepackage[utf8]{inputenc}
\usepackage{graphicx}
\usepackage{menukeys}
\usepackage{amsmath}
\usepackage{amssymb}
\usepackage[T1]{fontenc}
\usepackage{upquote}
\usepackage{wrapfig}
\usepackage{makeidx}
\usepackage{tikz, pgf}
\usepackage{listing}
\usepackage{fancyvrb}

\makeindex

\title{An introduction to Sumerian Cuneiforms}
\author{Eduardo Costa \and Marcus Santos \and Sergio Teixeira}
\date{}


\usepackage{fontspec}

%%%\setromanfont{Noto Sans}
\newcommand{\fcn}{\setromanfont{Noto Sans Cuneiform}}
\newcommand{\ttf}{\ttfamily}

\begin{document}
\maketitle
\thispagestyle{empty}

\chapter{Ur-Nammu-9}

The Sumerian cuneiform script was the first
writing system invented by humankind.
Therefore, all educated individuals should
learn this 5,000-year-old script.
In this tutorial, we will learn how to read
and reproduce the writing on the Ur-Nammu 9 Brick.

%% \setromanfont{Noto Sans Cuneiform}

\begin{quotation}\large\fcn
𒀭𒋀𒆠

𒈗𒀀𒉌

𒌨𒀭𒇉

𒈗𒋀𒀊𒆠𒈠𒆤

𒂍𒀀𒉌

𒈬𒈾𒆕

𒂦𒋀𒀊𒆠𒈠

𒈬𒈾𒆕
\end{quotation}
There are few grammar books for Sumerian.
Unfortunately, Marie-Louise Thomsen's
{\bf ``The Sumerian Language"} does not use cuneiform,
so I cannot recommend it. This leaves us with
John Hayes' Manual of Sumerian and Joshua
Bowen's Learn to Read Ancient Sumerian.
Therefore, I advise you
to buy {\bf ``A Manual of Sumerian: Grammar and Texts"}
by Hayes to learn this ancient language in depth.
It is also a good idea to acquire
{\bf ``Learn to Read Ancient Sumerian"}
by Joshua Bowen and Megan Lewis.

\section{Disclaimer}
The authors of this book are not a scholars
in Sumerian studies in any sense.
Therefore, they may not help serious students
of cuneiforms to solve their pendencies
and questions.

For scholars and graduate students who are
writing their thesis, the authors recommend
John Hayes' {\bf Manual of Sumerian} and
Joshua Bowen's {\bf Learn to Read Ancient Sumerian}.
Hayes' manual strong points are inscriptions
and dedicatories, while Bowen and Lewis prefer
literary texts.

\section{Sentence structure}
To discuss grammar, scholars use a transliteration
of Sumerian cuneiforms to the Latin alphabet.
Below, you will find the transliteration of
the Ur-Nammu-9 document that we will study
in this lesson.
\begin{verbatim}
1- [NANNA
2-     LUGAL.ANI].{(R)} #dat           -- For his king
3- [UR-NAMMU                           -- Ur-Nammu,
4-     LUGAL.URIM5.{AK}].{E} #gen/erg  -- the king of UR,
5- [E2.ANI].{Ø}  #object               -- his temple
6- MU.NA.DU3 #verb                     -- he built
7- [BAD3.URIM5.{A(K)}].{Ø} #gen/obj    -- the city wall of Ur
8- MU.NA.DU3  #verb                    -- he built
\end{verbatim}

\section{Grammar functions in transliteration}

In the transliteration, grammar functions are
represented by indicators between braces.
In the example, the grammar functions are:
\begin{description}
\item[1,2] The benefactive has an unwritten R,
  which is represented by \verb|{(R)}|
\item[3,4] The genitive ends in \verb|{AK}| after
  consonant; the ergative ends in \verb|{E}|
\item[5] The object of the action has no ending,
  which is represented by \verb|{Ø}|
\item[7] The genitive has an unwritten K,
  which is represented by \verb|{A(K)}|
\end{description}
Square brackets delimit a noun chain, i.e.,
a noun followed by a sequence of limiting
qualifiers that may contain adjectives,
apositives and a genitive.
Example: \verb|[UR-NAMMU LUGAL.URIM5.{AK}].{E}|
means
\begin{quote}
\begin{verbatim}
[Ur-Nammu, Ur's king].{task-doer}
\end{verbatim}
\end{quote}
After the close square bracket, a braced symbol
suffix indicates the function of the noun chain.
For instance, \verb|.{E}| shows that
\verb|[UR.NAMMU...].{E}| is the doer of
the sentence's task. The \verb|{(R)}| symbol
shows that \verb|[NANNA...].{(R)}| receives
the benefits of the task:
\verb|[God Nanna].{benefactive}|.

The noun chain may contain a genitive, as was
stated in the previous paragraph. If you don't
know the role of a genitive, it is a grammar
function that shows possession. In English,
the Saxon genitive marks the possessor with
apostrophe-s and comes before the noun:
{\em Ur's king}. In Sumerian, the possessor
follows the noun and is marked with \verb|{AK}|
after consonant and \verb|{K}| after vowel:
\verb|{URIM5 MA].{K}| is equivalent
to {\em Ur's king}.

Braces represent the grammatical function endings.
For instance, the ergative function-ending
represents the doer of the task and is written
as \verb|{E} #erg|, where \verb|#erg| is a comment
that will be omitted in more advanced lessons.
The person who receives the benefit of the
action is called dative and is represented
as \verb|{RA} #dat|, where the \verb|#dat|
comment is usually omitted.

The empty ending of the object is commented
as \verb|{∅} #obj| or simply as \verb|{} #obj|.
In the example, the objects are the constructions
of king Ur-Nammu:
\begin{quote}
\begin{verbatim}
[E2 A NI].{∅}               -- his temple
[[BAD3.URIM5].{A(K)}].{∅}   -- the city wall of Ur
\end{verbatim}
\end{quote}
Unwritten endings are placed between parentheses,
such as \verb|{(R)}|.

\section{Line 1 \& 2}
The Ur-Nammu 9 document is divided into eight
rectangles. In the first rectangle,
the text {\fcn 𒀭𒋀𒆠} is written, which is
the Sumerogram for the name of Nanna, the god of the Moon.
The {\fcn 𒀭} symbol is read as \verb|an|
(or \verb|digir|) and is determinative for deity.
We will learn in the next paragraph that this word
is in the dative case; therefore, the translation
of the rectangle is {\em ``For Nanna."}\\


\begin{tabular}[!h]{l l l}
\fcn\Large 𒀭 𒋀𒆠
&\fcn\Large 𒈗 &\fcn\Large 𒀀𒉌\\
  $^d$nanna & lugal & a ni\\
\multicolumn{3}{l}{\ttf (tr an nanna lugal a ni)}\\
\multicolumn{3}{l}{\em For the god Nanna, his master,}\\
\hline\\
\multicolumn{3}{l}{{\fcn 𒀭 𒋀𒆠}
                    ($^d$nanna) the god Nanna }\\
\multicolumn{3}{l}{{\fcn 𒈗}
                    (lugal) king, master }\\
\multicolumn{3}{l}{{\fcn 𒀀𒉌}
                    (a ni) his }\\
\end{tabular}

\verb||\\
Sumerian uses symbols, called determinatives,
to make the meaning clearer. The star {\fcn 𒀭}
in front of a god's name is the determinative
of divinity. In transliteration, the determinatives
are represented as a superscript letter, such
as $^d$\verb|nanna|.

The Emacs command \verb|(tr an nanna lugal a ni)| is used
to typeset Sumerian. There are instructions about
this command on the page where you found this tutorial.

\section{Line 3 \& 4}
The third line of the Ur-Nammu-9 document contains the name of
Ur-Nammu ({\fcn 𒌨𒀭𒇉}), the king who rebuilt the temple
of $^d$Nanna and is the document's author.
The king's name is formed by {\fcn 𒌨} (\verb|ur|),
which means {\em man} or {\em dog},
and {\fcn 𒀭𒇉} ($^d$\verb|nanna|),
the Mother Earth of the Sumerians.
Therefore, the king's name, {\fcn 𒌨𒀭𒇉},
means {\em ``The Man of Nammu."}
Note that the determinative of
deity ({\fcn 𒀭}) precedes the goddess's name.\\

\begin{tabular}[!h]{l l l l l l l l}
\fcn\Large 𒌨𒀭𒇉
&\fcn\Large 𒈗 &\fcn\Large 𒋀𒀊𒆠 &
\fcn\Large 𒈠 & \fcn\Large 𒆤\\
  ur-$^d$nammu & lugal & urim & ma & ke4\\
\multicolumn{5}{l}{\ttf (tr ur nammu lugal urim ma ke4)}\\
\multicolumn{5}{l}{\em Ur-Nammu, the king of Ur,}\\
\hline\\
\multicolumn{5}{l}{{\fcn 𒌨𒀭𒇉}
                    (ur-$^d$nammu) King Ur-Nammu}\\
\multicolumn{5}{l}{{\fcn 𒈗}
                    (lugal) king, master }\\
\multicolumn{5}{l}{{\fcn 𒋀𒀊𒆠}
                    (urim$^{ki}$) the city of Ur }\\
\multicolumn{5}{l}{{\fcn 𒆠}
     (ki) {\em determinative of places} }\\
\multicolumn{5}{l}{{\fcn 𒈠}
     (ma(k)) {\em dative after consonant M} }\\
\multicolumn{5}{l}{{\fcn 𒆤}
     (ke4) {\em contraction of dative with ergative} }\\
\multicolumn{5}{l}{{\fcn 𒈠𒆤}
     (make4) {\em genitive contracted with ergative} }\\
\end{tabular} 

\verb||\\
The fourth line contains {\fcn 𒈗𒋀𒀕𒆠𒈠𒆤}
(tr lugal urim2 ma ke4), where {\fcn 𒋀𒀕𒆠} (tr urim)
represents the city that was the cult center of Nanna.
It is formed by the Sumerograms (tr shesh) ({\fcn 𒋀})
and (tr unug) ({\fcn 𒀕}).
The Sumerogram {\fcn 𒆠} is the determinative
for geographic names. Determinatives,
such as {\fcn 𒀭} ("digir" - deity)
and {\fcn 𒆠} ("ki" - place), are not pronounced.
Their role is to make the meaning of the word clearer.

The genitive case denotes possession.
Unlike the dative, English has a genitive case,
formed by an apostrophe followed by {\bf ``s."}
In English, one would say, {\bf ``Urim's King."}
In Sumerian, the genitive follows the possessor
and is marked with \verb|{ak}| after consonants
and \verb|{k}| after vowels. In this nominal
chain, the ``\verb|a|" of \verb|{ak}| was
assimilated with the previous consonant,
becoming {\fcn 𒈠} (\verb|ma|).
The Sumerogram {\fcn 𒆤} (\verb|ke4|)
represents the \verb|{k}| of the genitive
and the \verb|{e}| of the ergative.

Sumerian is an ergative language, meaning the agent
of transitive actions is marked. In Sumerian, the
ergative marker is \verb|{e}|. However, the subject
of an intransitive verb, like ``to go" or ``to sleep,"
does not receive the \verb|{e}| that marks the agent,
whom linguists call ergative. Unmarked functions,
such as the Sumerian subject of an intransitive
verb and direct object of a transitive verb,
are called absolutive and said to be marked
with the null symbol \verb|{Ø}|.
In short, for the Sumerians and
modern Basques, if the subject of a sentence
does not perform a task, it cannot be called ergative.

\section{Line 5}
The fifth rectangle introduces the
temple (22 - {\fcn 𒂍}) that Ur-Nammu built.
The expression {\fcn 𒂍𒀀𒉌} (e2 ani)
means ``{\em his temple}." It is in the absolutive
case and, therefore, receives the null
symbol mark \verb|{Ø}|, a technical way of
saying it does not bear a mark.
The noun chain {\fcn 𒂍𒀀𒉌} (e2 ani) undergoes
the consequences of the task performed.
Thus, it is often called patient, accusative or target.

\verb||\\
\begin{tabular}[!h]{l l l}
\fcn\Large 𒂍
&\fcn\Large 𒀀 &\fcn\Large 𒉌\\
  e2 & a & ni\\
\multicolumn{3}{l}{\ttf (tr e2 a ni)}\\
\multicolumn{3}{l}{\em his temple}\\
\hline\\
\multicolumn{3}{l}{{\fcn 𒂍}
  (e2 me esh) {\em plural},
  houses, temples }\\
\multicolumn{3}{l}{{\fcn 𒂍𒈨𒌍}
                    (lugal) king, master }\\
\multicolumn{3}{l}{{\fcn 𒀀𒉌}
                    (a ni) his }\\
\end{tabular}

\section{Line 6}
A verbal stem prefixed by a sequence of
particles and possibly followed by a suffix
is called a {\em verbal chain}. The verbal
chain {\fcn 𒈬𒈾𒆕} (mu-na-du3) can be translated
as ``{\em built}."\\

\begin{tabular}[!h]{l l l}
\fcn\Large 𒈬
&\fcn\Large 𒈾 &\fcn\Large 𒆕\\
  mu & na & du3\\
\multicolumn{3}{l}{\ttf (tr mu na du3)}\\
\multicolumn{3}{l}{\em he has built for the god}\\
\hline\\
\multicolumn{3}{l}{{\fcn 𒆕}
                    (du3) to build, to make, to plant }\\
\multicolumn{3}{l}{{\fcn 𒈬}
  (mu) {\em conjugation prefix (CP), ventive prefix,}
            here}\\
\multicolumn{3}{l}{{\fcn 𒈾}
                    (na) {\em cross-references the dative} }\\
\end{tabular}

\verb||\\
The verbal chain of the example has two
prefixes and a stem:\\

\begin{description}
\item[\fcn 𒈬] — Ventive Conjugation Prefix (CP).
  The Ventive CP indicates that the action occurs
  here, close to the speaker.
\item[\fcn 𒈾] — Dimensional Prefix (DP) cross-referencing
  the dative. Sumerian has a DP for each sentence component,
  except the ergative and the absolutive cases.
\item[\fcn 𒆕] — verbal stem, {\em he has built}
\end{description}

\section{Line 7 \& 8}
The noun phrase {\fcn 𒂦𒋀𒀕𒆠𒈠}
(tr bad3 urim ma) means ``{\em wall of Ur}."
The sumerogram {\fcn 𒂦} (tr bad3) means
``{\em city wall}." The \verb|{(k)}| of the
genitive is omitted, meaning it is not expressed
because it was not pronounced at
the end of a nominal phrase.\\

\verb||\\
\begin{tabular}[!h]{l l l l l l l l}
\fcn\Large 𒂦
&\fcn\Large 𒋀𒀕𒆠 &\fcn\Large 𒈠 &
\fcn\Large 𒈬 & \fcn\Large 𒈾
& \fcn\Large  𒆕\\
  bad3 & urim & ma & mu & na & du3\\
\multicolumn{6}{l}{\ttf (tr bad3 urim ma mu na du3)}\\
\multicolumn{6}{l}{\em the city wall of Ur, he has built}\\
\hline\\
\multicolumn{5}{l}{{\fcn 𒊏}
                    (ra) ra, {\em dative ending}}\\
\multicolumn{5}{l}{{\fcn 𒈾}
                    (na) {\em reference to dative} }\\
\end{tabular} 

\section{Reading the brick}
Let's read the whole brick inscription.

\begin{enumerate}
\item (tr an nanna) ({\fcn 𒀭𒋀𒆠 }) {\bf\em-- For the god Nanna...}
\item (tr lugal ani) ({\fcn 𒈗𒀀𒉌}) {\bf\em -- his master,}
  The word `lugal' means king or master. It is formed
  from `lu2,' ({\fcn 𒇽}) which means `man,'
  and `gal,' ({\fcn 𒃲}) which can be translated
  as `great.' The expression `a-ni' ({\fcn 𒀀𒉌 })
  is equivalent to the possessive pronoun `his.'
\item (tr ur-nammu) ({\fcn 𒌨𒀭𒇉}) {\bf\em -- Ur-Nammu,}
\item (tr lugal urim ki ma ke4) ({\fcn 𒈗𒋀𒀊𒆠𒈠𒆤})
  {\bf\em -- the king of Ur,}
\item (tr e2 a ni) ({\fcn 𒂍𒀀𒉌}) {\bf\em -- his temple,}
  Remember that you already learned the
  meaning of `a ni.'
\item (tr mu na du3) ({\fcn 𒈬𒈾𒆕}) {\bf\em -- he has built.}
\item (tr bad3 urim ma) ({\fcn 𒂦𒋀𒀕𒆠𒈠})
  {\bf\em -- The wall of Ur,}
\item (tr mu na du3) ({\fcn 𒈬𒈾𒆕})
  {\bf\em -- he built for Nanna.}
\end{enumerate}

\section{Translation}
The meaning of the whole document is something
like this: {\bf\em``For the god Nanna, his Master, Ur-Nammu, the King of Ur, built his temple. The king also built the city walls of Ur for Nannaa."}

\section{The method}
I will use the method I employed in this first
chapter to introduce a few other documents.
In other words, each chapter will contain
grammar, vocabulary, syllables,
and essential Sumerograms for reading
a Sumerian document. This methodology ensures
you can handle a manageable amount of information
initially.

After discussing how to read a Sumerian inscription,
each chapter contains an in-depth presentation
of the Sumerian grammar. Initially,
you can do without reading this final
grammar section. You can return to return
to it after practicing Sumerian with a few inscriptions.


\section{APPENDIX: Grammar notes}

Congratulations. You have finished the first lesson.
This appendix gives further details about the
case elements, the noun chain and the verbal chain.
If you don't feel like it, you don't need to read
it now. You can return to this lesson after completing
a few Sumerian documents.

\section*{Case elements}
The subject of a sentence is the topic of the conversation.
Besides the subject, the sentence may have other marked
components called case elements. Case elements may have
references in the verbal chain. The leading case elements
with their marks and references are:

\subsection*{Ergative: {\fcn 𒂊} \{e\} task doer}

{\fcn\Large 𒈗𒂊  𒂦 𒋀𒀕𒆠𒈠  𒈬𒈾𒆕}\\
(tr lugal e bad3 urim ma mu na du3)\\
The king built the city wall of Ur.

\subsection*{Dative: {\fcn 𒊏} 
\{ra\}/ {\fcn 𒈾} (-na-) for}

{\fcn\Large 𒎏𒀀𒉌𒊏𒈗𒂊𒂦𒋀𒀊𒆠𒈠𒈬𒈾𒆕}\\
(tr nin a ni ra lugal e bad3 urim ma mu na du3)\\
For his lady, the king built the wall of Urim.

\subsection*{Locative: {\fcn 𒀀} • \{a\}
/{\fcn 𒉌} • (-ni-) in, on}

{\fcn\Large 𒈗𒂊𒌷𒀀𒂍𒈬𒉌𒆕}\\
(tr lugal e uru a e2 mu ni du3)\\
The king built a house in the city.

\subsection*{Terminative: {\fcn 𒂠}
 • \{še\}/{\fcn 𒅆} • (-ši-) in order to}

{\fcn\Large 𒂷𒂊𒌷𒈬𒂠𒂵𒅆𒁺}\\
(tr ĝe26 e uru ĝu10-my she-goal ga shi ĝen)\\
I will go there to my city.




\end{document}

