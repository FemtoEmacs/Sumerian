\documentclass[a4paper,12pt]{book}
\usepackage{polyglossia}
\usepackage{makeidx}
\usepackage{setspace}

\newcommand*\sepstars{%
  \begin{center}
    $\star\star\star$
\end{center}}

\setotherlanguage{english}

\newcommand{\fcn}{\setmainfont{Akkadian.otf}}
\newcommand{\fcm}{\large\setmainfont{Akkadian.otf}}
\newcommand{\fsm}{\Large\setmainfont{Akkadian.otf}}

\usepackage{listings}

\lstset{basicstyle=\fsm, breaklines=true}

\makeindex

\title{An introduction to Sumerian Cuneiforms}
\author{Eduardo Costa \and Marcus Santos
  \and Sergio Teixeira}
\date{}

\begin{document}
\maketitle
\thispagestyle{empty}
\pagenumbering{arabic}
\chapter{Ur-Nammu-9}

The cuneiform script was the first
writing system invented by humankind.
Therefore, all educated individuals should
learn this 5,000-year-old script.
In this tutorial, we will learn how to read
Sumerian cuneiform.

\index{an {\fcn 𒀭} ! determinative of god}
\index{an {\fcn 𒋀𒆠} ! god Nanna}
\index{lugal {\fcn 𒈗} ! king}
\index{a ni {\fcn 𒀀𒉌} ! his/her }
\index{ur-nammu {\fcn 𒌨𒀭𒇉} ! Ur-Nammu}
\index{urim5 {\fcn 𒋀𒀊𒆠} ! the city  of Ur }
\index{ma ke4 {\fcn 𒈠𒆤} ! contr. gen/erg  }
\index{ki {\fcn 𒆠 } ! det. place}
\index{e2 {\fcn 𒂍 } ! house}
\index{mu {\fcn 𒈬 } ! venitive }
\index{na {\fcn 𒈾} ! ref. dat.}
\index{bad3 {\fcn 𒂦} ! city wall}
\index{mu na du3 {\fcn 𒈬𒈾𒆕} ! he built for her} 
\begin{quotation}\LARGE\fcn
𒀭𒋀𒆠

𒈗𒀀𒉌

𒌨𒀭𒇉

𒈗𒋀𒀊𒆠𒈠𒆤

𒂍𒀀𒉌

𒈬𒈾𒆕

𒂦𒋀𒀊𒆠𒈠

𒈬𒈾𒆕
\end{quotation}
There are few grammar books for Sumerian.
Unfortunately, Marie-Louise Thomsen's
{\bf ``The Sumerian Language"} does not use cuneiform,
so I cannot recommend it. This leaves us with
John Hayes' Manual of Sumerian and Joshua
Bowen's {\bf ``Learn to Read Ancient Sumerian''}.
Therefore, I advise you
to buy {\bf ``A Manual of Sumerian: Grammar and Texts"}
by Hayes to learn this ancient language in depth.
It is also a good idea to acquire
{\bf ``Learn to Read Ancient Sumerian"}
by Joshua Bowen and Megan Lewis.

\section{Disclaimer}
The authors of this book are not a scholars
in Sumerian studies in any sense.
Therefore, they may not help serious students
of cuneiforms to solve their pendencies
and questions.

For scholars and graduate students who are
writing their thesis, the authors recommend
John Hayes' {\bf Manual of Sumerian} and
Joshua Bowen's {\bf Learn to Read Ancient Sumerian}.
Hayes' manual strong points are inscriptions
and dedicatories, while Bowen and Lewis prefer
literary texts.

\section{Sentence structure}
To discuss grammar, scholars use a transliteration
of Sumerian cuneiforms to the Latin alphabet.
Below, you will find the transliteration of
the Ur-Nammu-9 document that we will study
in this lesson.
\begin{verbatim}
1- [nanna
2-     lugal.ani].{(r)} #dat           -- For his king
3- [ur-nammu                           -- Ur-Nammu,
4-     lugal.urim.{ak}].{e} #gen/erg   -- the king of UR,
5- [e2.ani].{}  #object                -- his temple
6- mu.na.du3 #verb                     -- he built
7- [bad3.urim5.{a(k)}].{} #gen/obj     -- the city wall of Ur
8- mu.na.du3  #verb                    -- he built
\end{verbatim}

\section{Grammar functions in transliteration}

In the transliteration, grammar functions are
represented by indicators between braces.
In the example, the grammar functions are:
\begin{description}
\item[1,2] The benefactive has an unwritten ``(r)'',
  which is represented by \verb|{(r)}|
\item[3,4] The genitive ends in \verb|{ak}| after
  consonant; the ergative ends in \verb|{e}|
\item[5] The object of the action has no ending,
  which is represented by \verb|{}|
\item[7] The genitive has an unwritten ``(k)'',
  which is represented by \verb|{a(k)}|
\item[8] The verbal chain {\fcn 𒈬𒈾𒆕} (tr mu na du3)
  starts with the ventive prefix {\fcn 𒈬},
  followed by a cross-reference {\fcn 𒈾} (tr na)
  to the dative.
\end{description}

Square brackets delimit a noun chain, i.e.,
a noun followed by a sequence of limiting
qualifiers that may contain adjectives,
apositives and a genitive.
Example: \verb|[ur-nammu lugal.urim5.{ak}].{e}|
means
\begin{quote}
\begin{verbatim}
[Ur-Nammu, Ur's king].{task-doer}
\end{verbatim}
\end{quote}
After the close square bracket, a braced symbol
suffix indicates the function of the noun chain.
For instance, \verb|.{e}| shows that
\verb|[ur.nammu...].{e}| is the doer of
the sentence's task. The \verb|{(r)}| symbol
shows that \verb|[nanna...].{(r)}| receives
the benefits of the task:
\verb|[God Nanna].{benefactive}|.

The noun chain may contain a genitive, as was
stated in the previous paragraph. If you don't
know the role of a genitive, it is a grammar
function that shows possession. In English,
the Saxon genitive marks the possessor with
[{\bf 's}] and comes before the noun:
{\em Ur's king}. In Sumerian, the possessor
follows the noun and is marked with \verb|{ak}|
after consonant and \verb|{k}| after vowel:
\verb|{urim5 ma].{k}| is equivalent
to {\em Ur's king}.

Braces represent the grammatical function endings.
For instance, the ergative function-ending
represents the doer of the task and is written
as \verb|{e} #erg|, where \verb|#erg| is a comment
that will be omitted in more advanced lessons.
The person who receives the benefit of the
action is called dative and is represented
as \verb|{ra} #dat|, where the \verb|#dat|
comment is usually omitted.

The empty ending of the object is commented
as \verb|{|$\emptyset$\verb|} #obj| or simply as \verb|{} #obj|.
In the example, the objects are the constructions
of king Ur-Nammu:
\begin{quote}
\begin{verbatim}
[e2 a ni].{}               -- his temple
[[bad3.urim5].{a(k)}].{}   -- the city wall of Ur
\end{verbatim}
\end{quote}
Unwritten endings are placed between parentheses,
such as \verb|{(r)}|.

\section{Line 1 \& 2}
The Ur-Nammu 9 document is divided into eight lines.\\

\begin{tabular}[!h]{l l l}
\fcn\Large 𒀭 𒋀𒆠
&\fcn\Large 𒈗 &\fcn\Large 𒀀𒉌\\
  $^d$nanna & lugal & a ni\\
\multicolumn{3}{l}{\texttt (tr an nanna lugal a ni)}\\
\multicolumn{3}{l}{\em For the god Nanna, his master,}\\
\hline\\
\multicolumn{3}{l}{{\fcn 𒀭 𒋀𒆠}
                    ($^d$nanna) the god Nanna }\\
\multicolumn{3}{l}{{\fcn 𒈗}
                    (lugal) king, master }\\
\multicolumn{3}{l}{{\fcn 𒀀𒉌}
                    (a ni) his }\\
\end{tabular}

\verb||\\
 In the first line,
the text {\fcn 𒀭𒋀𒆠} is written, which is
the Sumerogram for the name of Nanna, the god of the Moon.
The {\fcn 𒀭} symbol is read as \verb|an|
(or \verb|digir|) and is determinative for deity.
We will learn in the next paragraph that this word
is in the dative case; therefore, the translation
of the rectangle is {\em ``For Nanna."}

Sumerian uses symbols, called determinatives,
to make the meaning clearer. The star {\fcn 𒀭}
in front of a god's name is the determinative
of divinity. In transliteration, the determinatives
are represented as a superscript letter, such
as $^d$\verb|nanna|.

The Emacs command \verb|(tr an nanna lugal a ni)| is used
to typeset Sumerian. There are instructions about
this command on the page where you found this tutorial.

\section{Line 3 \& 4}
The third line of the Ur-Nammu-9 document contains the name of
Ur-Nammu ({\fcn 𒌨𒀭𒇉}), the king who rebuilt the temple
of $^d$Nanna and is the document's author.
The king's name is formed by {\fcn 𒌨} (\verb|ur|),
which means {\em man} or {\em dog},
and {\fcn 𒀭𒇉} ($^d$\verb|nanna|),
the Mother Earth of the Sumerians.
Therefore, the king's name, {\fcn 𒌨𒀭𒇉},
means {\em ``The Man of Nammu."}
Note that the determinative of
deity ({\fcn 𒀭}) precedes the goddess' name.\\

\begin{tabular}[!h]{l l l l l l l l}
\fcn\Large 𒌨𒀭𒇉
&\fcn\Large 𒈗 &\fcn\Large 𒋀𒀊𒆠 &
\fcn\Large 𒈠 & \fcn\Large 𒆤\\
  ur-$^d$nammu & lugal & urim & ma & ke4\\
\multicolumn{5}{l}{\texttt (tr ur nammu lugal urim ma ke4)}\\
\multicolumn{5}{l}{\em Ur-Nammu, the king of Ur,}\\
\hline\\
\multicolumn{5}{l}{{\fcn 𒌨𒀭𒇉}
                    (ur-$^d$nammu) King Ur-Nammu}\\
\multicolumn{5}{l}{{\fcn 𒈗}
                    (lugal) king, master }\\
\multicolumn{5}{l}{{\fcn 𒋀𒀊𒆠}
                    (urim$^{ki}$) the city of Ur }\\
\multicolumn{5}{l}{{\fcn 𒆠}
     (ki) {\em determinative of places} }\\
\multicolumn{5}{l}{{\fcn 𒈠}
     (ma(k)) {\em dative after the consonant ``M''} }\\
\multicolumn{5}{l}{{\fcn 𒆤}
     (ke4) {\em contraction of dative with ergative} }\\
\multicolumn{5}{l}{{\fcn 𒈠𒆤}
     (ma ke4) {\em genitive contracted with ergative} }\\
\end{tabular} 

\verb||\\
The fourth line contains {\fcn 𒈗𒋀𒀕𒆠𒈠𒆤}
(tr lugal urim2 ma ke4), where {\fcn 𒋀𒀕𒆠} (tr urim)
represents the city that was the cult center of Nanna.
It is formed by the Sumerograms (tr shesh) ({\fcn 𒋀})
and (tr unug) ({\fcn 𒀕}).
The Sumerogram {\fcn 𒆠} is the determinative
for geographic names. Determinatives,
such as {\fcn 𒀭} ("digir" - deity)
and {\fcn 𒆠} ("ki" - place), are not pronounced.
Their role is to make the meaning of the word clearer.

The genitive case denotes possession.
Unlike the dative, English has a genitive case,
formed by an apostrophe followed by {\bf ``s."}
In English, one would say, {\bf ``Urim's King."}
In Sumerian, the genitive follows the possessor
and is marked with \verb|{ak}| after consonants
and \verb|{k}| after vowels. In this nominal
chain, the ``\verb|a|" of \verb|{ak}| was
assimilated with the previous consonant,
becoming {\fcn 𒈠} (\verb|ma|).
The Sumerogram {\fcn 𒆤} (\verb|ke4|)
represents the \verb|{k}| of the genitive
and the \verb|{e}| of the ergative.

Sumerian is an ergative language, meaning the agent
of transitive actions is marked. In Sumerian, the
ergative marker is \verb|{e}|. However, the subject
of an intransitive verb, like ``to go" or ``to sleep,"
does not receive the \verb|{e}| that marks the agent,
whom linguists call ergative. Unmarked functions,
such as the Sumerian subject of an intransitive
verb and direct object of a transitive verb,
are called absolutive and said to be marked
with the null symbol \verb|{}|.
In short, for the Sumerians and
modern Basques, if the subject of a sentence
does not perform a task, it cannot be called ergative.

\section{Line 5}
The fifth rectangle introduces the
temple (e2 - {\fcn 𒂍}) that Ur-Nammu built.
The expression {\fcn 𒂍𒀀𒉌} (e2 ani)
means ``{\em his temple}." It is in the absolutive
case and, therefore, receives the null
symbol mark \verb|{}|, a technical way of
saying it does not bear a mark.
The noun chain {\fcn 𒂍𒀀𒉌} (e2 ani) undergoes
the consequences of the task performed.
Thus, it is often called patient, accusative or target.

\verb||\\
\begin{tabular}[!h]{l l l}
\fcn\Large 𒂍
&\fcn\Large 𒀀 &\fcn\Large 𒉌\\
  e2 & a & ni\\
\multicolumn{3}{l}{\texttt (tr e2 a ni)}\\
\multicolumn{3}{l}{\em his temple}\\
\hline\\
\multicolumn{3}{l}{{\fcn 𒂍}
  (e2) house, temple }\\
\multicolumn{3}{l}{{\fcn 𒂍𒈨𒌍} 
                    (e2 me esh-pl) pl. houses, temples }\\
\multicolumn{3}{l}{{\fcn 𒀀𒉌}
                    (a ni) his }\\
\end{tabular}
\index{e2 me esh-pl {\fcn 𒂍𒈨𒌍} ! houses}

\newpage
\section{Line 6}
A verbal stem prefixed by a sequence of
particles and possibly followed by a suffix
is called a {\em verbal chain}. The verbal
chain {\fcn 𒈬𒈾𒆕} (mu-na-du3) can be translated
as ``{\em built}."\\

\begin{tabular}[!h]{l l l}
\fcn\Large 𒈬
&\fcn\Large 𒈾 &\fcn\Large 𒆕\\
  mu & na & du3\\
\multicolumn{3}{l}{\texttt (tr mu na du3)}\\
\multicolumn{3}{l}{\em he has built for the god}\\
\hline\\
\multicolumn{3}{l}{{\fcn 𒆕}
                    (du3) to build, to make, to plant }\\
\multicolumn{3}{l}{{\fcn 𒈬}
  (mu) {\em conjugation prefix (CP), ventive prefix,}
            here}\\
\multicolumn{3}{l}{{\fcn 𒈾}
                    (na) {\em cross-references the dative} }\\
\end{tabular}


\verb||\\
The verbal chain of the example has two
prefixes and a stem:\\

\begin{description}
\item[Ventive Conjugation Prefix] {\fcn 𒈬}   (CP).
  The Ventive CP indicates that the action occurs
  here, close to the speaker.
\item[Dimensional Prefix] {\fcn 𒈾}
  (DP) cross-referencing
  the dative. Sumerian has a DP for each sentence component,
  except the ergative and the absolutive cases.
\item[Verbal stem] {\fcn 𒆕}  {\em he has built}
\end{description}

\section{Line 7 \& 8}
The noun phrase {\fcn 𒂦𒋀𒀕𒆠𒈠}
(tr bad3 urim ma) means ``{\em wall of Ur}."
The sumerogram {\fcn 𒂦} (tr bad3) means
``{\em city wall}." The \verb|{(k)}| of the
genitive is omitted, meaning it is not expressed
because it was not pronounced at
the end of a nominal phrase.\\

\verb||\\
\begin{tabular}[!h]{l l l l l l l l}
\fcn\Large 𒂦
&\fcn\Large 𒋀𒀕𒆠 &\fcn\Large 𒈠 &
\fcn\Large 𒈬 & \fcn\Large 𒈾
& \fcn\Large  𒆕\\
  bad3 & urim & ma & mu & na & du3\\
\multicolumn{6}{l}{\texttt (tr bad3 urim ma mu na du3)}\\
\multicolumn{6}{l}{\em the city wall of Ur, he has built}\\
\hline\\
\multicolumn{5}{l}{{\fcn 𒊏}
                    (ra) ra, {\em dative ending}}\\
\multicolumn{5}{l}{{\fcn 𒈾}
                    (na) {\em reference to dative} }\\
\end{tabular} 

%%aqui

\section{Reading the brick}
Let's read the whole brick inscription.

\begin{enumerate}
\item (tr an nanna) ({\fcn 𒀭𒋀𒆠 }) {\bf\em-- For the god Nanna...}
\item (tr lugal ani) ({\fcn 𒈗𒀀𒉌}) {\bf\em -- his master,}
  // The word `lugal' means king or master. It is formed
  from `lu2,' ({\fcn 𒇽}) which means `man,'
  and `gal,' ({\fcn 𒃲}) which can be translated
  as `great.' The expression `a-ni' ({\fcn 𒀀𒉌 })
  is equivalent to the possessive pronoun `his.'
\item (tr ur-nammu) ({\fcn 𒌨𒀭𒇉}) {\bf\em -- Ur-Nammu,}
\item (tr lugal urim ki ma ke4) ({\fcn 𒈗𒋀𒀊𒆠𒈠𒆤})
  {\bf\em -- the king of Ur,}
\item (tr e2 a ni) ({\fcn 𒂍𒀀𒉌}) {\bf\em -- his temple,}
  // Remember that you already learned the
  meaning of `a ni.'
\item (tr mu na du3) ({\fcn 𒈬𒈾𒆕}) {\bf\em -- he has built.}
\item (tr bad3 urim ma) ({\fcn 𒂦𒋀𒀕𒆠𒈠})
  {\bf\em -- The wall of Ur,}
\item (tr mu na du3) ({\fcn 𒈬𒈾𒆕})
  {\bf\em -- he built for Nanna.}
\end{enumerate}

\section{Translation}
The meaning of the whole document is something
like this: {\bf\em``For the god Nanna, his Master,
  Ur-Nammu, the King of Ur, built his temple.
  The king also built the city walls of Ur for Nanna."}

\section{The method}
I will use the method I employed in this first
chapter to introduce a few other documents.
In other words, each chapter will contain
grammar, vocabulary, syllables,
and essential Sumerograms for reading
a Sumerian document. This methodology ensures
you can handle a manageable amount of information
initially.

After discussing how to read a Sumerian inscription,
each chapter contains an in-depth presentation
of the Sumerian grammar. Initially,
you can do without reading this final
grammar section. You can return
to it after practicing Sumerian with a few inscriptions.


\section{Grammar notes}
\label{causation-intransitive-verbs}
In the expression {\em transitive verb},
the word {\em transitive} means
``affecting something or someone else.''
Therefore, a transitive verb only makes
sense if someone exerts the verbal action
on an object. On the other hand, an
intransitive verb makes sense without
any object. In a few words, without
an object to affect, the sentence
constructed around a transitive verb
does not seem complete:
\begin{quote}\em
The king built.
\end{quote}
If you say something like that, people
around you will ask: ``What did he build?''
Then you may answer:\\
\index{Causative contruction}

\verb||\\
\begin{tabular}[!h]{l l l l l l l l l}
\fcm 𒋀𒆠𒊏 &\fcm 𒈗𒂊 &\fcm 𒂍𒌉 &\fcm 𒈬𒈾𒆕\\
(tr nanna ra) & (tr lugal e) & (tr e2 tur) & (tr mu na du3)\\
\multicolumn{4}{l} {(tr nanna ra lugal e e2 tur mu na du3)}\\
\multicolumn{4}{l} {\em The king built a small house for Nanna.}\\
\end{tabular}\\

If you say that a man went out, nobody will ask
for further information.
Therefore, the verb ``to go out'' is intransitive.\\

\verb||\\
\begin{tabular}[!h]{l l l l l l l l l}
\fcm 𒇽 &\fcm 𒁀𒌓𒁺\\
(tr lu2) & (tr ba e3)\\
\multicolumn{2}{l} {(tr lu2 ba e3)}\\
\multicolumn{2}{l} {\em The man went out. }\\
\end{tabular}\\
\index{e3 {\fcn 𒌓𒁺}  ! to go out}

Here is the novelty: In Sumerian, any
transitive verb can be turned into
a transitive verb. Thus, let us
consider the sentence below.

\begin{tabular}[!h]{l l l l l l l l l}
\fcm 𒈗𒂊 &\fcm 𒇽 &\fcm 𒈬𒌦𒌓𒁺\\
(tr lugal e) & (tr lu2) & (tr mu un e3)\\
\multicolumn{3}{l} {(tr lugal e lu2 mu un e3)}\\
\multicolumn{3}{l} {\em The king expelled the man.}\\
\end{tabular}\\

Now, the verb has an object, which changed the
intransitive verb ``to go out'' into the
transitive verb ``to cause to go out.''
This method of creating transitive verbs
is called {\em causative construction.}


\chapter*{APPENDIX 1: Grammar notes}

Congratulations. You have finished the first lesson.
This appendix gives further details about the
case elements, the noun chain and the verbal chain.
If you don't feel like it, you don't need to read
it now. You can return to this lesson after completing
a few Sumerian documents.

\section{Case elements}
The subject of a sentence is the topic of the conversation.
Besides the subject, the sentence may have other marked
components called case elements. Case elements may have
references in the verbal chain. The leading case elements
with their marks and references are:

\subsection*{Ergative:  \{e\} task doer}
{\fsm 𒈗𒂊  𒂦 𒋀𒀕𒆠𒈠  𒈬𒈾𒆕}\\
(tr lugal e bad3 urim ma mu na du3)\\
The king built the city wall of Ur.
\index{e {\fcn 𒂊} ! ergative mark}

\subsection*{Dative: \{ra\} / (-na-) for}
{\fcn\Large 𒎏𒀀𒉌𒊏𒈗𒂊𒂦𒋀𒀊𒆠𒈠𒈬𒈾𒆕}\\
(tr nin a ni ra lugal e bad3 urim ma mu na du3)\\
The king built the wall of Ur for his lady.
\index{ra {\fcn 𒊏} ! dative mark}

\newpage
\subsection*{Locative:  \{a\} // (-ni-) in, on}

{\fcn\Large 𒈗𒂊𒌷𒀀𒂍𒈬𒉌𒆕}\\
(tr lugal e uru a e2 mu ni du3)\\
The king built a house in the city.
\index{a {\fcn 𒀀} ! locative mark}
\index{ni {\fcn 𒉌} ! ref. loc.}

\subsection*{Terminative: \{še\}// (-ši-) in order to}

{\fcn\Large 𒂷𒌷𒈬𒂠𒂵𒅆𒁺}\\
(tr ĝe26 uru ĝu10-my she-goal ga shi ĝen)\\
I will go there to my city.
\index{she-goal {\fcn 𒂠} ! terminative mark}
\index{shi {\fcn 𒅆} ! ref. term.}

\subsection*{Ablative: \{ta\}// (-ta-) or (-ra-) out of}

{\fcn\Large 𒌷𒋫𒁀𒋫𒁺}\\
(cn uru ta ba ta ĝen)\\
He went out from the city.
\index{ta {\fcn 𒋫} ! out of, abl. mark}

\subsection*{Comitative: \{da\} //   (-da-)  with}

{\fcn\Large 𒈗𒂊𒌉𒀀𒉌𒁕𒂍𒈬𒌦𒁕𒆕}\\
(tr lugal e dumu a ni da e2 mu un da du3)\\
The king built the house with his son.
\index{da {\fcn 𒁕} ! with, comit. mark}

\subsection*{Equitative: \{gin\} // (-gin-) like, as}

{\fcn 𒀀𒁀𒋀𒈬𒁶}\\
(tr a ba shesh ĝu10-my gin-equitative)\\
Who is like my brother?
\index{gin-equitative {\fcn 𒁶} ! like, such as}

\subsection*{Absolutive: \{$\emptyset$\} or \{\}}

{\fcn\Large 𒎏𒀀𒉌𒊏𒈗𒂊𒂦𒈬𒈾𒆕}\\
(tr nin a ni ra lugal e bad3 mu na du3)\\
For his lady, the king has built the city wall.

\newpage
\section{Dative conjugation}
When used as a prefix to a verb, the dative takes
different forms depending on the person and number
it is referring to.


\subsection*{ (-ma-) to me}
{\fcn\Large 𒂷𒊏𒈗𒂊𒂍𒈬𒈠𒆕}\\
(tr ĝe26 ra lugal e e2 mu ma du3)\\
The king built a house for me.
\index{ma {\fcn 𒈠} ! ref. dat, for me}

\subsection*{ (-ra-) to you}

{\fcn\Large 𒍢𒊏𒈗𒂊𒂍𒈬𒊏𒆕}\\
(tr ze2 ra lugal e e2 mu ra du3)\\
The king has built a house for you.
\index{ra {\fcn 𒊏} ! ref. dat., for you} 

\subsection*{ (-na-) to him/to her}

{\fcn\Large 𒎏𒊏𒈗𒂊𒂍𒈬𒈾𒆕}\\
(tr nin ra lugal e e2 mu na du3)\\
The king has built a house for the lady.
\index{na {\fcn 𒈾} ! ref. dat., for him/her}

\subsection*{ (-me-) to us}

{\fcn\Large 𒈗𒂊𒂍𒈬𒈨𒆕}\\
(tr lugal e e2 mu me du3)\\
The king has built a house for us.
\index{me {\fcn 𒈨} ! ref. dat., for us}
  
\subsection*{ (-ne-) to them}
{\fcn\Large 𒈗𒂊𒂍𒈬𒉈𒆕}\\
(tr lugal e e2 mu ne du3)\\
The king has built a house for them.
\index{ne {\fcn 𒉈}  ! ref. dat., for them}

\section{Transitive verbs}
A transitive verb describes an action that
transitions from a subject to a direct object.
In a transitive verb, the subject is the doer
of the action and is called ergative, which
is the Greek term for the person who performs a task.
In Sumerian, the ergative is marked
with {\fcn 𒂊} \verb|{e}|.
\index{Ergative}

The absolutive case is the entity that undergoes
the consequences of a task. The absolutive case
can be the person accused of a deed. In this case,
it is called accusative.

The absolutive case can also be a target of a shooting.
Or it can be the object of health care, in which case
it is called patient by the doctors.

Some linguists call {\em patient} all kinds of
absolutive cases of a transitive verb, while others
prefer the term accusative.

In Sumerian,
the absolutive case receives no mark, but the
linguists say it is marked by the
null symbol \verb|{Ø}|.

The transitive verb itself comes last in a Sumerian
sentence, and is described by a chain of affixes
surrounding the stem. This verbal chain may contain
a Modal Prefix (MP, such as {\fcn 𒉡} • nu • not),
a Conjugation Prefix (CP, such as {\fcn 𒈬}
• mu • {\em ventive}, here), initial pronominal
prefix (IPP, such as N in {\fcn 𒈬𒌦𒆪𒂊}
• mu-n.dab.e • he seizes her) and suffix
pronouns ({\fcn 𒂗𒉈𒂗} • en-de3-en
• us, {\fcn 𒌦𒍢𒂗} • un-ze2-en • you people).
Below, there are examples of all initial
pronominal prefixes.
\index{en de3 en {\fcn 𒂗𒉈𒂗} ! us}
\index{un ze2 en {\fcn 𒌦𒍢𒂗} ! you people}

\section{Initial Pronominal Prefixes (IPP)}
In the verbal chain, the Initial Pronominal
Prefixes (IPP) come after the Conjugation
Prefix (CP) that is {\fcn 𒈬} (-mu-) in the examples below. 
The {\fcn 𒈬} (-mu-) prefix is the ventive,
i.e., it shows that the
action is performed towards the speaker.
English uses different verbs for the
{\em andative} (motion away from the speaker)
and the {\em ventive} (motion towards the speaker):
{\em to take away / to bring}, {\em to go / to come}, etc.
Sumerian gets the same effect by adding
the {\em ventive} Conjugation Prefix (CP)
to the verbal chain.
\index{Initial Pronominal Prefixes}
\newpage
Below is an exhaustive list of the Initial Pronominal Prefixes
for all grammatical persons.\\
\verb||\\
(tr mu dab e)\\
{\fcn\Large 𒈬𒆪𒂊}\\
He seizes me.
\index{mu'dab e {\fcn 𒈬𒆪𒂊} ! he seizes me}

\verb||\\
(tr mu e dab e)\\
{\fcn\Large 𒈬𒂊𒆪𒂊}\\
He seizes you.
\index{mu e dab e {\fcn 𒈬𒂊𒆪𒂊} ! he seizes you}

\verb||\\
(tr mu un dab e)\\
{\fcn\Large 𒈬𒌦𒆪𒂊}\\
He seizes her.
\index{mu un dab e {\fcn 𒈬𒌦𒆪𒂊} ! he seizes her}

\verb||\\
(tr mu me dab e)\\
{\fcn\Large 𒈬𒈨𒆪𒂊}\\
He seizes us.
\index{mu me dab e {\fcn 𒈬𒈨𒆪𒂊} ! he seizes us}

\verb||\\
(tr mu un ne dab e)\\
{\fcn\Large 𒈬𒌦𒉈𒆪𒂊}\\
He seizes them.\\
\index{mu un ne dab e {\fcn 𒈬𒌦𒉈𒆪𒂊} ! he seizes them}

I have for you a complete example of a transitive
sentence below. I provide  a pronunciation
key and vocabulary, so I hope you can
scan the sentence.\\

\index{munus {\fcn 𒊩} ! woman}
\index{lu2 {\fcn 𒇽} ! man}
\index{she {\fcn 𒊺} ! barley}
\index{uru {\fcn 𒌷} ! city}
\index{shum2 {\fcn 𒋧} ! to give}
\index{ab {\fcn 𒀊} ! it, initial pron. prefix}
{\fcn\Large 𒊩𒊏𒇽𒂊𒊺𒌷𒀀𒈬𒈾𒀊𒋧𒂊}\\
(tr munus ra lu2 e she uru a mu na ab shum2 e)

\noindent
\begin{tabular}[!h]{l l l l l l l l}
  \fcn\Large 𒊩𒊏 &\fcn\Large 𒇽𒂊 &\fcn\Large 𒊺
  &\fcn\Large 𒌷𒀀 &\fcn\Large 𒈬𒈾𒀊𒋧𒂊\\
  munus ra & lu2 e & she & uru a & mu na ab shum2 e\\
  for the woman & the man & barley & in the city
  & he gave it to her\\
\end{tabular}\verb||\\

The translation of the sentence is: {\em
The man gave barley to the woman in the city.}
The person who receives the barley is
marked with the dative {\fcn 𒊏} \verb|{ra}|;
the doer of the action has the ergative
marker {\fcn 𒂊} \verb|{e}|, and the
place of the occurrence has the locative
marker {\fcn 𒀀} \verb|{a}|.

\newpage
\subsection*{Vocabulary}
{\fcn\Large 𒊩} • (munus) woman, female\\
\verb||\\
{\fcn\Large 𒊏} • (ra) {\em dative marker}\\
\verb||\\
{\fcn\Large 𒇽} • (lu2) man, male\\
\verb||\\
{\fcn\Large 𒂊} • (e) {\em ergative marker}\\
\verb||\\
{\fcn\Large 𒊺} • (še) barley, grain\\
\verb||\\
{\fcn\Large 𒌷} • (uru) city\\
\verb||\\
{\fcn\Large 𒀀} • (a) {\em locative marker}\\
\verb||\\
{\fcn\Large 𒈬} • (mu) {\em venitive conjugation prefix}, here\\
\verb||\\
{\fcn\Large 𒈾} • (na) {\em cross-reference to
the dative}, to her\\
\verb||\\
{\fcn\Large 𒀊} • (ab) {\em Initial Prefix Pronoun}, it\\
\verb||\\
{\fcn\Large 𒋧} • (shum2) to give\\

\newpage
\section{Intransitive verb}
An intransitive verb does not have a direct object.
In Sumerian, the subject of an intransitive verb
goes to the absolutive case and, therefore, is not marked.\\

\index{i3 im {\fcn 𒉌𒅎} ! {\em finite verb prefix}}
\index{ĝu10 -- uru ĝu10-my {\fcn 𒌷𒈬} ! {\em poss. pron.} my city}
\index{uru ĝu10-my {\fcn 𒌷𒈬}  ! {\em poss. pron.} my city}
\index{uru ĝu10-my she-goal {\fcn 𒌷𒈬𒂠} ! {\em terminative}} 
\index{ĝen {\fcn 𒁺}  ! to come}
\verb||\\
{\fcn\Large 𒈗 𒌷𒈬𒂠 𒉌𒅎𒁺}\\
(tr lu2 uru ĝu10-my she-goal i3 im ĝen)\\
\begin{tabular}[!h]{l | l | l | l | l l l l}
  \fcn\Large 𒈗 &\fcn\Large 𒌷𒈬𒂠
  &\fcn\Large 𒉌  𒅎
  &\fcn\Large 𒁺\\
  lu2 & uru ĝu10 she-goal & i3  im & ĝen\\
  the king & to my city
  &\em finite verb prefix & came\\
\end{tabular}\verb||\\


The translation of the above example is:
{\em The king came to my city.} You find
the vocabulary necessary to scan this
example below.\\
\verb||\\
{\fcn\Large 𒈗} • (lugal) king \\
\verb||\\
{\fcn\Large 𒉌𒅎} • (im) {\em finite verb marker} \\
\verb||\\
{\fcn\Large 𒁺} • (g̃en) to come\\
\verb||\\
{\fcn\Large 𒂠} • (še3) to, towards\\
\verb||\\
{\fcn\Large 𒌷} • (uru) city\\
\verb||\\
{\fcn\Large 𒌷𒈬} • (uru.ĝu10) my city\\
\verb||\\
{\fcn\Large 𒌷𒈬𒂠} • (uru ĝu10 she-goal) to my city\\
\verb||\\

\newpage
\section{Modal Prefix (MP)}
The modal prefixes express modality, i.e.,
relationships to reality or truth.
You can only learn the indicative and negation
modal prefixes for now. You may learn the other
prefixes when you encounter them in Sumerian documents
and inscriptions.
\begin{quotation}
{\large\bf Indicative: ($\emptyset$-)}\\
In Sumerian, the indicative is unmarked.
The empty prefix \verb|/Ø-/| may represent
this fact in transliteration. However,
people rarely show unmarked prefixes.\\
\index{nu un gu7 {\fcn 𒉡𒌦𒅥}  ! {\em neg.} he didn't eat}
{\large\bf Negation:  /nu-/}\\
{\fcn\Large 𒉡𒌦𒅥}\\
(tr nu un gu7)\\
He didn't eat it.\\

\index{hhe2- {\fcn 𒃶𒅁𒅥𒂊}  ! {\em let him} eat}
{\large\bf Let him:  hhe2-}\\
{\fcn\Large 𒃶𒅁𒅥𒂊}\\
(tr hhe2 ib gu7 e)\\
Let him eat it.\\

\index{hha an gu7 {\fcn 𒄩𒀭𒅥} ! he ate it, {\em indeed} }
{\large\bf Indeed:  hha-}\\
{\fcn\Large 𒄩𒀭𒅥}\\
(tr hha an gu7)\\
He ate it, indeed.\\

\index{ga {\fcn 𒂵} ! {\em cohortative}, let us}
{\large\bf Cohortative:  ga-}\\
{\fcn\Large 𒂵𒉌𒌈𒃻𒊑𒂗𒉈𒂗}\\
(tr ga i3 ib2 ĝar re en ne en)\\
Let us put it there.\\

\index{na {\fcn 𒈾𒀊𒅥𒂊} ! {\em prohibitive} }
{\large\bf Prohibitive: na-}
{\fcn\Large 𒈾𒀊𒅥𒂊}\\
(tr na ab gu7 e)\\
He must not eat it.
\end{quotation}

\newpage
\section{Conjugation Prefix (CP)}
The main Conjugation Prefixes (CP) are \verb|/mu-/|
to indicate that the action occurs here,
 \verb|/ba/| to form
middle/passive voice, \verb|/i3/| to create
a finite verbal tense, and \verb|/ma/| in
combination with \verb|/ra/| of benefit.
Below, you will find a fairly complete list
of Conjugation Prefixes, but you need
to learn only \verb|/mu-/| and \verb|/i3/|
for this first lesson.
\begin{quotation}
{\large\bf Here:}\\
{\fcn\Large 𒈬} - {\fcn\Large 𒌦𒁺}\\
(tr mu un re6)\\
He brought it here.\\
\index{re6 {\fcn 𒁺}  ! to bring}
\index{mu un re6 {\fcn 𒈬𒌦𒁺}
  ! {\em ventive,} he brought it here }

{\large\bf Finite verb:}\\
{\fcn\Large 𒉌} - {\fcn\Large 𒅎𒁺}\\
(tr i3 im ĝen)\\
He came here.\\
\index{i3 im {\fcn 𒉌𒅎} ! {\em fin. verb prefix}  }

{\large\bf Finite verb, followed by open vowel:}\\
{\fcn\Large 𒉈} - {\fcn\Large 𒅔𒁺}\\
(tr bi2 in re6)\\
He made the team bring it.\\
\index{bi2 in re6 {\fcn 𒉈𒅔𒁺}
  ! {\em fin. verb + open vowel}}

{\large\bf Finite verb, followed by ra:}\\
{\fcn\Large 𒈠} - {\fcn\Large 𒊏𒀭𒁺}\\
(tr ma ra an re6)\\
He brought it here to you.\\
\index{ma ra an re6 {\fcn 𒈠𒊏𒀭𒁺}
    ! {\em fin. verb + ra}}

{\large\bf Middle voice:}\\
{\fcn\Large 𒁀} - {\fcn\Large 𒀭𒁺}\\
(tr ba an re6)\\
He took it for himself.\\
{\em Obs. The middle voice with {\texttt /ba/}
  indicates an action that affects the doer.}\\
\index{ ba an re6 {\fcn 𒁀𒀭𒁺} ! {\em middle voice} }

{\large\bf Passive voice:}\\
{\fcn\Large 𒁀} - {\fcn\Large 𒁺}\\
(tr ba re6)\\
It was brought.
\end{quotation}
\index{ba re6 {\fcn 𒁀𒁺} ! {\em passive voice} }
\newpage
\section{Nominal chain}
\label{nominal-chain}
In Sumerian, most adjectives are formed from
verbs by adding the suffix {\fcn 𒀀} \verb|{a}|.
For example, the verb below means to be strong.
\begin{quote}
{\fsm 𒆗}  (kalag) to be strong
\end{quote}
To form an adjective from kalag, one adds an \verb|{a}|.
In Sumerian, different from English, the adjectives
follow the noun.\\
\index{kalag {\fcn 𒆗} ! to be strong  }

The expression below means {\bf\it mighty king}.
Note that the adjective follows the verb,
and the {\fcn 𒀀} marker contracts with
the previous consonant to form the
{\fcn 𒂵} (ga) syllable.

\index{lugal kalag ga {\fcn 𒈗𒆗𒂵} ! mighty king }
{\fcn\Large 𒈗 𒆗 𒂵}\\
(tr lugal kalag ga)\\
a mighty king\\

\index{saxon genitive}
In English, the Saxon genitive is marked
with S and precedes the verb. Therefore,
one writes ``{\bf\it Elil's Warrior}."
In Sumerian, the genitive is marked
with \verb|{k}| after a vowel
and \verb|{ak}| after a consonant.
Like the adjective, the genitive follows the noun.
The \verb|{k}| of the genitive was rarely
written except when combined with the ergative.
In this case, it was written
as {\fcn 𒆤} \verb|{ke4}|.\\

Below, there is another example of
the adjective {\fcn 𒀀} \verb|{a}|
marker contracting with the previous
consonant to form an open syllable.\\

{\fcn\Large 𒂍𒈗𒆷}\\
(tr e2 lugal la)\\
the king's house\\

\verb||\\
Now, let us examine a somewhat longer
example of a noun chain.\\

\index{ama {\fcn 𒂼} ! mother  }
\verb||\\
{\fcm 𒂼𒀀𒉌𒊏} •  {\fcm 𒌉𒈗𒆷𒆤}
   •   {\fcm 𒂍 𒈬𒈾𒆕}\\
\verb||\\
(tr ama a ni ra  • dumu lugal la ke4  • e2 mu na du3)\\
The king's son has built a house for his mother.


\chapter{Inscription in Inanna's temple}

\begin{quotation}\LARGE\fcn
  𒀭𒈹 𒎏𒀀𒉌
  
  𒌨𒀭𒇉
  
  𒍑𒆗𒂵
  
  𒈗𒋀𒀊𒆠𒈠
  
  𒈗𒆠𒂗𒄀 𒆠𒌵𒆤
  
  𒂍𒀀𒉌
  
𒈬𒈾𒆕
\end{quotation}

Translation:
{\bf\em For Inanna, his lady, Ur-Nammu,
  the mighty man, the king of Ur,
  the king of Sumer and Akad, built her temple.}

\section{Sentence structure}
\begin{verbatim}
1- [inanna nin a ni].{(r)}             -- For Inanna, his Lady,
2- [ur-nammu                           -- Ur-Nammu,
3-   [nita kalag].{a}                  -- the mighty man,
4-   [lugal urim ma].{(k)}             -- the king of Ur,
5-   [lugal ki-en-gi ki uri].{k}].{e}  -- the king of Sumer and Akkad,
6- [e2 a ni].{}                        -- her (Inanna's) temple
7- mu na du3                           -- built.
\end{verbatim}

\newpage
From now on, the sentence structure
will not contain the comments \verb|.{k} #gen|
for the genitive, \verb|.{r} #dat| for the dative
or \verb|.{e} #erg| for the ergative (doer of the task).
The suffixes \verb|.{r}| for the dative,
\verb|.{k}| for the genitive
and \verb|.{e}| for the ergative
should suffice for showing the grammatical
function of the noun chain and its components.
However, functional suffixes you didn't learn
in the previous lessons will be commented on.

\section{Annotations}

%% {\LARGE\fcn 𒀭𒈹 𒎏𒀀𒉌}\\
\begin{tabular}[!h]{l l l l l l l}
  \fcm 𒀭 &\fcm 𒈹
  &\fcm 𒎏 &\fcm 𒀀𒉌\\
  an & inanna & nin & a ni\\
  \multicolumn{4}{l}{\texttt (tr an inanna nin a ni)}\\
  \multicolumn{4}{l}{\em For Inanna, his lady,}\\
  \hline\\
  \multicolumn{4}{l}{{\fcn 𒀭𒈹} • ($^d$inana) Inanna}\\
  \multicolumn{4}{l}{{\fcn 𒎏} • (nin) lady, queen, mistress}\\
  \multicolumn{4}{l}{{\fcn 𒀀𒉌} • (a ni) his, her}\\
\end{tabular}\verb||\\
\index{inanna {\fcn 𒈹} ! the goddess Inanna}
\index{nin {\fcn 𒎏} ! lady, queen}

This noun phrase ends in an unwritten \verb|{(r)}|,
the dative marker. However, there is no ambiguity
since the verb chain has a dative reference.\\

\verb||\\
%%{\LARGE\fcn 𒌨𒀭𒇉 𒍑𒆗𒂵}\\
\begin{tabular}[!h]{l l l l l l l}
  \fcm 𒌨𒀭𒇉 &\fcm 𒍑
  &\fcm 𒆗 &\fcm 𒂵\\
  ur-nammu & nita & kalag & ga\\
  \multicolumn{4}{l}{\texttt (tr ur-nammu nita kalag ga)}\\
  \multicolumn{4}{l}{\em Ur-Nammu, the mighty man,}\\
  \hline\\
  \multicolumn{4}{l}{{\fcn 𒍑} • (nita) man, male}\\
  \multicolumn{4}{l}{{\fcn 𒆗} • (kalag)
    to be strong, to be mighty}\\
  \multicolumn{4}{l}{{\fcn 𒆗 𒂵} • (kalag ga)
    {\em adj. from verb}, mighty}\\
\end{tabular}\verb||\\
\index{nita {\fcn 𒍑}  ! man, male}
\index{kalag {\fcn 𒆗}  ! to be mighty}
\index{kalag ga {\fcn 𒆗𒂵}  ! {\em adj. from verb}, mighty}

One may form adjectives by adding an
\verb|{a}|-suffix to a verbal root,
     {\em kalag} in the above expression.
     This nominalizing suffix contracts with
     the preceding word's final \verb|g|,
     giving extra information about its correct reading.
     Different from English, Sumerian adjectives
     follow the noun they modify.\\
\index{Nominalizing suffix A}
     
 \newpage
 \noindent    
 %%{\LARGE\fcn 𒈗𒋀𒀊𒆠𒈠}\\
 \begin{tabular}[!h]{l l l l l l l l l}
   \fcm 𒈗 &\fcm 𒋀𒀊𒆠
   &\fcm 𒈠\\
   lugal & urim & ma\\
   \multicolumn{3}{l}{\texttt (tr lugal urim ma)}\\
   \multicolumn{3}{l}{\em the king of Ur,}\\
 \end{tabular}\verb||\\

 As we learned from text 1, the genitive is
 formed by \verb|{k}| after vowels
 and \verb|{ak}| after consonants.
 The scribe often omitted the \verb|{(k)}|
 of \verb|{ak}|. The ``{\em m}'' of
 ``{\em ma}'' is contamination from the final
 consonant of the previous word.

\verb||\\
 %%{\LARGE\fcn 𒈗𒆠𒂗𒄀𒆠𒌵𒆤}\\
 \begin{tabular}[!h]{l l l l l l l l l}
   \fcm 𒈗 &\fcm 𒆠𒂗𒄀
   &\fcm 𒆠𒌵 &\fcm 𒆤 \\
   lugal & ki-en-gi & ki uri & ke4\\
   \multicolumn{4}{l}{\texttt (tr lugal ki-en-gi ki uri ke4)}\\
   \multicolumn{4}{l}{\em the king of Sumer and Akkad,}\\
   \hline\\
     \multicolumn{4}{l}{{\fcn 𒆠𒂗𒄀} • (ki-en-gi) Sumer }\\
  \multicolumn{4}{l}{{\fcn 𒆠𒌵} • (ki-uri) Akkad }\\
 \end{tabular}\verb||\\
 
 
 \verb||\\
 In ke4 ({\fcn 𒆤}), the \verb|{k}| is the
 genitive marker, and the \verb|{e}| is the ergative marker.
 You already saw the analysis of the last two
 lines in lesson 1, therefore they should pose
 no difficulty to you.
 
 \verb||\\
 %%{\LARGE\fcn 𒂍𒀀𒉌}\\
 \verb||\\
\begin{tabular}[!h]{l l l}
\fcm 𒂍
&\fcm 𒀀 &\fcm 𒉌\\
  e2 & a & ni\\
\multicolumn{3}{l}{\texttt (tr e2 a ni)}\\
\multicolumn{3}{l}{\em his temple}\\
\hline\\
\multicolumn{3}{l}{{\fcn 𒂍}
  (e2)  house, temple}\\
\multicolumn{3}{l}{{\fcn 𒂍𒈨𒌍}
                    (e2 me esh-pl) houses, temples }\\
\multicolumn{3}{l}{{\fcn 𒀀𒉌}
                    (a ni) his }\\
\end{tabular}


\noindent
%%{\LARGE\fcn 𒈬𒈾𒆕}\\
\verb||\\
\verb||\\
\begin{tabular}[!h]{l l l}
\fcm 𒈬
&\fcm 𒈾 &\fcm 𒆕\\
  mu & na & du3\\
\multicolumn{3}{l}{\texttt (tr mu na du3)}\\
\multicolumn{3}{l}{\em he has built for the god}\\
\hline\\
\multicolumn{3}{l}{{\fcn 𒆕}
                    (du3) to build, to make, to plant }\\
\multicolumn{3}{l}{{\fcn 𒈬}
  (mu) {\em conjugation prefix (CP), ventive prefix,}
            here}\\
\multicolumn{3}{l}{{\fcn 𒈾}
     (na) {\em cross-references the dative} }\\
\end{tabular}

%%%begin copy
\section{Verbs}
The introduction of an ergative subject
into the sentence is the preferred method
of expressing causation with intransitive
verbs, as you learned on
page~\pageref{causation-intransitive-verbs}.

\verb||\\
\begin{tabular}[!h]{l l l l l l l l l}
\fcm 𒇽𒅆𒌨𒂊 &\fcm 𒊩 𒌀 &\fcm 𒈬𒌦𒁺\\
(tr lu2 hhulu e) & (tr munus sumun) & (tr mu un ĝen)\\
\multicolumn{3}{l} {(tr lu2 hhulu e munus sumun mu un ĝen)}\\
\multicolumn{3}{l} {\em The bad man caused the old woman to go.}\\
\end{tabular}\\
\index{sumun {\fcn 𒌀}  ! old}
\index{hhulu {\fcn 𒅆𒌨} ! bad}

Consider a sentence:
``The powerful king caused the man to build a
house.'' This sentence has three participants,
to wit, the mighty king, the man, and the task
of building a house.
One of the participants forced the other to
perform the task. In Sumerian, the dative case
marked by {\fcn 𒊏} (ra) identifies the person who is
caused to do the task. In the third person singular,
the Sumerians used the conjugation prefix
{\fcn 𒉌}  (ni) to referece this kind of dative.\\
\index{ni (ni) ! to him, {\em ref. dat. of part.}}

\verb||\\
\begin{tabular}[!h]{l l l l l l l l l}
\fcm 𒈗𒆗𒂵𒂊 &\fcm 𒇽𒊏 &\fcm 𒂍 &\fcm 𒈬𒉌𒅔𒆕\\
(tr lugal kalag ga e) & (tr lu2 ra) & (tr e2) & (tr mu ni in du3)\\
\multicolumn{4}{l} {(tr lugal kalag ga e lu2 ra e2 mu ni in du3)}\\
\multicolumn{4}{l} {\em The powerful king made the man
      to build a house.}\\
\end{tabular}\\

In the second person, the verbal chain would have
{\fcn 𒊑}  (ri) as reference. In the example below,
(za ra) (za ra, you) is usually omitted since
the conjugation prefix {\fcn 𒉌}  (ni) makes clear
who was caused to build the house.\\
\index{ri (ri) ! to you, {\em ref. dat. of part.}}

\verb||\\
\begin{tabular}[!h]{l l l l l l l l l}
\fcm 𒈗𒆗𒂵𒂊 &\fcm 𒍝𒊏 &\fcm 𒂍 &\fcm 𒈬𒊑𒅔𒆕\\
(tr lugal kalag ga e) & (tr za ra) & (tr e2) & (tr mu ri in du3)\\
\multicolumn{4}{l} {(tr lugal kalag ga e za ra e2 mu ri in du3)}\\
\multicolumn{4}{l} {\em The powerful king made you build a house.}\\
\end{tabular}\\

To make a long story short, in sentences with three
participants, the dative indicates the person
that the ergative participant causes to do something.
However, you must be careful in distinguishing
dative of the participant that was caused to do
something from the dative of the beneficiary.
%%%end copy


\chapter*{APPENDIX 2: Conjugation}

Congratulations on finishing another
lesson. This appendix
details Sumerian pronouns and verbs.
After completing the fifth lesson, you can return
to it  to gain an in-depth
understanding of verbs.

\section{Possessive Pronouns}
You already learned a possessive pronoun:
{\fcn 𒈗𒀀𒉌} (tr lugar ani) ``{\em his master}''.
Below, I've included a complete list of
possessive pronouns.
\begin{quotation}
\noindent
{\bf (tr e2 ĝu10)} -- my house\\
{\fcm 𒂍 𒈬}\\


\noindent
{\bf (tr e2 zu)} -- thy house\\
{\fcm 𒂍𒍪}\\
\index{zu {\fcn 𒍪}  ! your}

\noindent
{\bf (tr e2 a ni)} -- his house\\
{\fcm 𒂍𒀀𒉌}\\

\noindent
{\bf (tr e2 bi)} -- its house\\
{\fcm 𒂍𒁉}\\
\index{bi {\fcn 𒁉}  ! its}

\noindent
{\bf (tr e2 me)} -- our house\\
{\fcm 𒂍𒈨}\\
\index{me {\fcn 𒈨} ! our}

\noindent
{\bf (tr e2 zu ne ne)} -- your house\\
{\fcm 𒂍𒍪𒉈𒉈}\\
\index{zu ne ne {\fcn 𒍪𒉈𒉈} ! your}

\noindent
{\bf (tr e2 a ne ne)} -- their house\\
{\fsm 𒂍𒀀𒉈𒉈}\\
\end{quotation}
\index{a ne ne {\fcn 𒀀𒉈𒉈} ! their}

\newpage
\section{Independent pronouns}
Sumerian has a set of independent pronouns
that I advise you to learn right away.
They are very important.
\begin{quotation}
\noindent
{\bf (ĝe26) I/me}\\
{\fcm 𒂷}\\
\index{ĝe26 {\fcn 𒂷}  ! I/me}

\noindent
{\bf (ze2) thou/thee}\\
{\fcm 𒍢}\\
Obs. {\Large\fcn 𒍢} (ze2) becomes {\Large \fcn 𒍝} (za)
when followed by the dative {\Large\fcn 𒊏} (ra).\\
\index{ze2 {\fcn 𒍢}  ! thou/thee}

\noindent
{\bf (a-ne) he/she/him/her}\\
{\fcm 𒀀𒉈}\\
\index{a ne {\fcn 𒀀𒉈}   ! he/she, him/her}

\noindent
{\bf (a-ne-ne) they}\\
{\Large\fcn 𒀀𒉈𒉈}
\verb||\\
\end{quotation}
\index{a ne ne {\fcn 𒀀𒉈𒉈}   ! they/them}

\noindent
\begin{tabular}[!h]{l l l l l l l l l}
  \fcm 𒀀𒉈 &\fcm 𒁾
  &\fcm 𒍝𒊏 &\fcm 𒈠𒊏𒀊𒋧𒈬\\
  a ne & dab5 & za ra & ma ra ab shum2 mu\\
  he & the tablet & to you & will give\\
  \multicolumn{4}{l}{\em He will give you the tablet.}\\
\end{tabular}

\subsection*{Vocabulary}
\begin{quotation}
\noindent
{\fsm 𒀀𒉈}  (a-ne) he/she\\
\verb||\\
{\fsm 𒁾}  (dab5) the tablet\\
\verb||\\
{\fsm 𒍝𒊏}  (zara) to you.
{\fcn 𒍢} (ze2) plus {\fcn 𒊏} (ra)
becomes {\fcn 𒍝𒊏}
\end{quotation}
\index{za ra (za ra) ! {\fcn 𒍢}  plus {\fcn 𒊏}}
\index{dab5 {\fcn 𒆪}  ! tablet}

\newpage
Sometimes, an independent pronoun appears
with an enclitic copula (verb {\em to be})
attached to its end, as shown below.\\
\index{to be ! {\em enclitic copula}}

\begin{quotation}
\noindent
\begin{tabular}[!h]{l l l l l l l l l}
\fsm 𒆪𒇷 &\fsm 𒍪 &\fsm 𒂷 &\fsm 𒈨𒂗\\ 
gu5-li & zu & ĝe26 & me en\\
friend & your & I & am\\
\multicolumn{4}{l}{\em I am your friend}\\
\end{tabular}\verb||\\
\index{ĝe26 me en {\fcn 𒂷𒈨𒂗}  ! I am}


\verb||\\
{\fsm 𒆪𒇷𒈬𒍢𒈨𒂗 }\\
(tr gu5 li ĝu10-my (my friend) ze2 me en (you are))\\     
You are my friend.
\index{ze2 me en {\fcn 𒍢𒈨𒂗}  ! you are}

\verb||\\
{\fsm 𒆪𒇷𒍪𒀀𒉈𒀀𒀭}\\
(tr gu5 li zu (your friend) a ne am3 (she/he is))\\     
She is your friend.
\index{gu5 li {\fcn 𒆪𒇷}  ! friend}
\index{a ne am3 {\fcn 𒀀𒉈𒀀𒀭}  ! she/he is}

\verb||\\
{\fsm 𒆪𒇷𒍪𒈨𒂗𒉈𒂗}\\
(tr gu5 li zu (your friend) me en ne en (we are))\\
We are your friends.
\index{me en ne en {\fcn 𒈨𒂗𒉈𒂗}  ! we are}

\verb||\\
{\fsm 𒆪𒇷𒈬𒈨𒂗𒍢𒂗}\\
(tr gu5 li ĝu10-my (my friend) me en ze2 en (you guys))\\
You guys are my friends.
\index{me en ze2 en (me en ze2 en) ! {\em pl.} you are}

\verb||\\
{\fsm 𒆪𒇷𒍪𒀀𒉈𒉈𒈨𒌍}\\
(tr gu5 li zu (your friends) a ne ne me esh-pl (they are))\\
They are your friends.\\
\end{quotation}
\index{a ne ne esh-pl {\fcn 𒀀𒉈𒉈𒌍}  ! they are}

\newpage
\section{Interrogative pronouns}
Sumerians marked yes/no interrogative sentences
only by intonation and possibly by lengthening
the final vowels, like many modern languages,
such as Spanish and Portuguese.

To ask who performed a task, Sumerians used
the interrogative word {\fcn 𒀀𒁀𒀀} (tr a ba a),
as shown below.\\

\verb||\\
{\fsm 𒂍} •  {\fsm 𒀀𒁀𒀀} •  {\fsm 𒅔𒆕}\\
(tr e2 •  a ba a •  in du3)\\
the temple • who • built?\\
Who built the temple?\\
\index{a ba a {\fcn 𒀀𒁀𒀀}  ! who?}

To ask who is something, Sumerians used the
interrogative pronoun {\fcn 𒀀𒁀} (tr a-ba),
as shown in the following example:\\

\verb||\\
{\fsm 𒀀𒁀} • {\fsm 𒀭𒌓} • {\fsm 𒁶}\\
(tr a ba • utu • gin-equitative)\\
Who • Utu • is like?\\
Who is like Utu?\\

In Sumerian, there is no wh-movement
to the beginning of the clause, like
in English and Spanish. Instead, the
interrogative words are placed immediately
before the verb.\\

\verb||\\
{\fsm 𒈗𒂊} • {\fsm 𒀀𒈾} • {\fsm 𒈬𒌦𒀝}\\
(tr lugal e • a na •  mu un ak)\\
the king • what • did he do?\\
What did the king do?\\
\index{a na {\fcn 𒀀𒈾}  ! what?}

\newpage
\noindent
{\fsm 𒌉𒈬} • {\fsm 𒀀𒈾} • {\fsm 𒉡𒍪}\\
(tr dumu ĝu10-my • a na • nu zu)\\
my son • what • does not know?\\
What does my son not know?\\
\index{nu zu {\fcn 𒉡𒍪}  ! he does not know}

\verb||\\
{\fsm 𒀀𒈾} • {\fsm 𒀀𒀭} • {\fsm 𒉈𒂊}\\
(tr a na • am3 • ne e)\\
what • is • this? \\
What is this?\\

\verb||\\
An exception to the rule of placing the
interrogative word immediately before
the verb occurs in why-questions,
as the example below shows.

\verb||\\
{\fsm 𒀀𒈾𒀸} • {\fsm 𒀀𒀭} • {\fsm 𒉌𒁺}\\
(tr a na ash • am3 • i3 ĝen)\\
what is it • that he came?\\
Why did he come?\\
\index{a na ash {\fcn 𒀀𒈾𒀸}  ! why?}

The expression {\fcn 𒀀𒈾𒀸} (a-na-ash)
that one usually translates as ``why?''
means literally ``what for?''

\newpage
\section{Conjugation}
Sumerian verbs have two aspects: the hamtu (perfective)
and the marû (imperfective). For the time being,
you can translate the hamtu as the English present
perfect, and the marû, as the English future.\\
\index{hamtu conjugation}
\index{marû conjugation}

\noindent
hamtu: {\fsm 𒈗𒂊𒂦𒈬𒌦𒁺}\\
(tr lugal e bad3 mu un gub)\\
The king has erected a wall here.\\
\index{gub {\fcn 𒁺}  ! to erect}

\noindent
marû: {\fsm 𒈗𒂊𒂦𒉌𒁺𒂊}\\
(tr lugal e bad3 i3 gub e)\\
The king will erect a wall.

\section{Hamtu and marû conjugation}
Marie-Louise Thomsen says that the transitive
verb distinguishes the hamtu conjugation with
pronominal prefixes, while the marû conjugation
has pronominal suffixes. As for intransitive
verbs, both the hamtu and the marû have
pronominal suffixes.

\subsection*{First person}
{\fsm 𒈾𒈬𒁺}\\
(tr na mu gub)\\
hamtu: I have set up a border stone.\\

\noindent
{\fsm 𒈾𒉌𒁺𒂗}\\
(tr na i3 gub en)\\
marû: I will set up a stone.
\index{na {\fcn 𒈾}  ! stone}

\newpage
\subsection*{Second person singular}
{\fsm 𒈾𒈬𒂊𒁺}\\
(tr na mu e gub)\\
hamtu: You have set up a stone.\\

\noindent
{\fsm 𒈾𒉌𒁺𒂗}\\
(tr na i3 gub en)\\
marû: You will set up a stone.


\subsection*{Third person singular (humans)}
{\fsm 𒈾𒈬𒌦𒁺}\\
(tr na mu un gub)\\
hamtu: He has set up a stone.\\

\noindent
{\fsm 𒈾𒉌𒁺𒂊}\\
(tr na i3 gub e)\\
marû: He will set up a stone.

\subsection*{First person plural}
{\fsm 𒈾𒈬𒁺𒁁𒂗𒉈𒂗}\\
(tr na mu gub be en de3 en)\\
hamtu: We have set up a stone.\\

\noindent
{\fsm 𒈾𒉌𒁺𒂗𒉈𒂗}\\
(tr na i3 gub en de3 en)\\
marû: We will set up a stone.


\subsection*{Second person plural}
{\fsm 𒈾𒈬𒂊𒁺𒁁𒂗𒍢𒂗}\\
(tr na mu e gub be en ze2 en)\\
hamtu: You have set up a stone.\\

\noindent
{\fsm 𒈾𒉌𒁺𒁁𒂗𒍢𒂗}\\
(tr na i3 gub be en ze2 en)\\
marû: You will set up a stone.

\subsection*{Third person plural}
{\fsm 𒈾𒈬𒌦𒁺𒁁𒌍}\\
(tr na mu un gub be esh-pl)\\
hamtu: They have set up a stone.\\

\verb||\\
{\fsm 𒈾𒉌𒁺𒁁𒂊𒉈}\\
(tr na i3 gub be e ne)\\
marû: They will set up a stone.\\

\verb||\\
Animals and plants have different pronouns
for the third person singular. Therefore,
in the third person singular, the hamtu aspect
is not the same for humans and animals.

\verb||\\
{\fsm 𒈾𒈬𒌒𒁺}\\
(tr na mu ub gub)\\
It has set up a stone.

\newpage
\section{Intransitive verb conjugation}
Intransitive verbs have the same forms
for the hamtu and the marû aspects.
Below is the complete conjugation
of the verb {\fcn 𒁺} (ĝen),
``to go'' (or ``to come'').
\index{ĝen {\fcn 𒁺}  ! to come, to go}

\subsection*{Singular}
{\fsm 𒉌𒁺𒂗}\\
(tr i3 ĝen en)\\
I went.

\verb||\\
{\fsm 𒉌𒁺𒂗}\\
(tr i3 ĝen en)\\
You went.

\verb||\\
{\fsm 𒉌𒁺}\\
(tr i3 ĝen)\\
He went.

\subsection*{Plural}
{\fsm 𒉌𒁻𒂗𒉈𒂗}\\
(tr i3 re7 en de3 en)\\
We went.

\verb||\\
{\fsm 𒉌𒁻𒂗𒍢𒂗}\\
(tr i3 re7 en ze2 en)\\
You people went.

\verb||\\
{\fsm 𒉌𒁻𒂠}\\
(tr i3 re7 esh)\\
They went.

%%%begin addition
\newpage
\section{True adjectives}
\label{true-adjectives}
\index{Position of adjectives}
As you learned on page~\pageref{nominal-chain},
Sumerian adjectives follow the noun they modify.
Then, ``mighty king'' becomes {\fcn 𒈗𒆗𒂵} 
(lugal kalag ga -- {\em king mighty}) in Sumerian.
You also learned that most adjectives are
formed from verbs by adding the
suffix {\fcn 𒀀} \verb|{a}|.
The verb below means to be strong.
\begin{quote}
{\fsm 𒆗}  (kalag) to be strong
\end{quote}
The expression below means {\bf\it mighty king}.
Pay attention to the fact
that the adjective follows the verb,
and the {\fcn 𒀀} marker contracts with
the previous consonant to form the
{\fcn 𒂵} (ga) syllable.
\begin{quote}
{\fcm 𒈗 𒆗 𒂵}\\
(tr lugal kalag ga)\\
{\em a mighty king}\\
\end{quote}
Besides the adjectives formed from
verbs, Sumerian has a few true adjectives.
Below is a list of the most common
adjectives that are not formed from verbs.\\
\index{kaskal {\fcn 𒆜}  ! road}
\index{daĝal {\fcn 𒂼}  ! wide}
\index{tur {\fcn 𒌉}  ! small}
\index{mahh {\fcn 𒈤}  ! great}
\index{sukud {\fcn 𒃴}  ! high}
\index{babbar {\fcn 𒌓}  ! white}
\index{giggi {\fcn 𒈪}  ! black}
\index{tum9 dir {\fcn 𒅎𒋛𒀀}  ! cloud}
\index{hhur saĝ {\fcn 𒄯𒊕}  ! mountain}
\index{True adjectives}
\verb||\\
\begin{tabular}[!h]{l l l l l l l l l}
  {\fcm 𒆜}  • {\fcm 𒂼}  & kaskal  • daĝal & wide road\\
  \\
    {\fcm 𒂍}  • {\fcm 𒃲}  & e2  • gal & big house\\
  \\
{\fcm 𒂍}  • {\fcm 𒌉} & e2  • tur & small house\\
\\
{\fcm 𒇽} • {\fcm 𒈤}  & lu2 • mahh & great man\\
\\
{\fcm 𒄯𒊕} • {\fcm 𒃴}   & hhur saĝ • sukud & high mountain\\
\\
{\fcm 𒂍}  • {\fcm 𒌓}   & e2  • babbar & white house\\
\\
{\fcm 𒅎𒋛𒀀}  •  {\fcm 𒈪}    & tum9 dir  • giggi & black cloud\\
\\
{\fcm 𒂍}   • {\fcm 𒉋}  & e2 • gibil & new house\\
\\
  {\fcm 𒂍}  • {\fcm 𒌀}  & e2 • sumun & old house\\
\end{tabular}
%%%end addition


\chapter{Ur-Nammu-31}
\begin{quotation}\fsm
𒀭𒎏𒃲
  
𒎏𒀀𒉌

𒌨𒀭𒇉

𒍑𒆗𒂵

𒈗𒋀𒀊𒆠𒈠

𒈗𒆠𒂗𒄀𒆠𒌵𒆤

𒉆𒋾𒆷𒉌𒂠

𒀀𒈬𒈾𒊒
\end{quotation}
{\em For Ningal, his lady, Ur-Nammu,
  the mighty man, the king of Ur,
  the king of Sumer and Akkad,
  dedicated this vessel for the protection of his life.}

\verb||\\
{\large\bf Sentence structure}
\begin{verbatim}
1- [ningal                            -- For Ningal,
2-     nin a ni].{(r)}                -- his Lady,
3- [ur-nammu                          -- Ur-Nammu,
4-   nita kalag.{a} #adjective        -- the mighty man,
5-   [lugal urim5 ma].{(k)}           -- the king of Ur,
6-   [lugal ki-en-gi ki uri].{k}].{e} -- the king of Sumer and Akkad,
7- [nam til3 a ni].{she3} #goal       -- for the sake of his life,
8- a mu na ru                         -- dedicated (this vessel).
\end{verbatim}

\newpage
\section{Verbal chain}
A Sumerian verb appears
as a chain of affixes in a particular order,
which is depicted in the table below for
the most common occurrences.\\

\noindent
\begin{tabular}[!h]{|l | l | l | l | l | l l l l l}
  \hline
  Modal  & Indicative & Negative
  & Coortative & Desiderative\\
  Prefix&$.\emptyset$ &\fsm 𒉡 &\fsm 𒂵 &\fsm 𒄩
       {\texttt or} 𒃶\\
  & null prefix & nu & ga & hha or hhe2\\
  \hline
  Conjugation  & Finite & Coordinator & Ventive & Middle Voice\\
  Prefix&\fsm 𒉌 &\fsm 𒅔𒂵 &\fsm 𒈬 &\fsm 𒁀\\
  & i3 & 'n ga & mu & ba\\
  \hline
  Dimensional & Dative &Locative &Directive &\\
  Prefix &\fsm 𒈾&\fsm 𒉌 &\fsm 𒂠 & \\
  & -na- & -ni- & -she3- &\\
  \hline
  Ergative & i3 du3 & e du3 & in du3 & ib2 du3 \\
  Infix   &\fsm 𒉌𒆕&\fsm 𒂊𒆕 &\fsm 𒅔𒆕 &\fsm 𒌈𒆕\\
  &I built & thou built & s/he built & they built\\
  \hline
  Verbal stem &\multicolumn{4}{l|}{\fsm 𒆕}\\
  \hline
\end{tabular}
\index{Verbal chain}

\subsection*{Example:}
\begin{tabular}[!h]{l | l| l| l | l l l l l l}
  \fcm 𒀭𒎏𒃲𒊏 &\fcm 𒈗𒂊 &\fcm 𒌷𒀀 &\fcm 𒂍
  &\fcm 𒉌 𒈾𒉌𒅔𒆕\\
  (tr an nin gal ra) & (tr lugal e) & (tr uru a) &
  (tr e2) & (tr i3 na ni in du3)\\
  for Ningal & the king & in the city & a temple & he built\\
  dative & ergative & locative & absolutive & ref. dat/loc\\
\end{tabular}
\index{nin gal {\fcn 𒀭𒎏𒃲} ! the goddess Ningal}

\newpage
\section{Annotations}

\noindent
\begin{tabular}[!h]{l l l l l l}
  \fsm  𒀭  𒎏𒃲 &\fsm 𒎏 &\fsm 𒀀𒉌\\
  (an) ningal & nin & a ni\\
  \multicolumn{3}{l}{(tr an ningal nin a ni)}\\
  \multicolumn{3}{l}{\em For Ningal, his lady,}\\
  \hline\\
  \fcm 𒀭𒎏𒃲
  &\multicolumn{2}{l}{($^d$Nin-gal) goddess of prisons}\\
  \fcm 𒎏
  &\multicolumn{2}{l}{(nin) lady}\\
\end{tabular}

\verb||\\
\verb||\\
\begin{tabular}[!h]{l l l l l l}
  \fsm  𒌨𒀭𒇉 &\fsm 𒍑 &\fsm 𒆗 &\fsm 𒂵\\
  ur-nammu & nita & kalag & ga\\
  \multicolumn{4}{l}{(tr ur-nammu nita kalag ga)}\\
  \multicolumn{4}{l}{\em Ur-Nammu, the mighty man,}\\
  \hline\\
  \fcm 𒍑
  &\multicolumn{3}{l}{(nita) man}\\
  \fcm 𒆗
  &\multicolumn{3}{l}{(kalag) to be mighty}\\
  \fcm 𒂵
  &\multicolumn{3}{l}{(ga) {\em adj. from verb}}\\
\end{tabular}

\verb||\\
\verb||\\
\begin{tabular}[!h]{l l l l l l}
  \fsm 𒈗 &\fsm 𒋀𒀊𒆠 &\fsm 𒈠\\
  lugal & urim &  ma\\
  \multicolumn{3}{l}{(tr lugal urim ma)}\\
  \multicolumn{3}{l}{\em the king of Ur,}\\
  \hline\\
  \fcm 𒈗
  &\multicolumn{2}{l}{(lugal) king}\\
  \fcm 𒋀𒀊𒆠
  &\multicolumn{2}{l}{(urim) the city of Ur}\\
  \fcm 𒈠
  &\multicolumn{2}{l}{(ma) {\em contr. of gen. with ``m''
    of ``urim''}}\\
\end{tabular}


\verb||\\
\verb||\\
\begin{tabular}[!h]{l l l l l l}
  \fsm 𒈗 &\fsm 𒆠𒂗𒄀 &\fsm 𒆠𒌵 &\fsm 𒆤\\
  lugal & ki-en-gi & ki uri & ke4\\
  \multicolumn{4}{l}{(tr lugal ki-en-gi ki uri ke4)}\\
  \multicolumn{4}{l}{\em the king of Sumer and Akkad,}\\
  \hline\\
  \fcm 𒆠𒂗𒄀
  &\multicolumn{3}{l}{(ki-en-gi) Sumer}\\
  \fcm 𒆠𒌵
  &\multicolumn{3}{l}{(ki uri) Akkad}\\
  \fcm 𒆤
  &\multicolumn{3}{l}{(ke4) {\em adj. contraction gen/erg}}\\
\end{tabular}

\verb||\\
\verb||\\
\begin{tabular}[!h]{l l l l l l}
  \fsm 𒉆 &\fsm 𒋾 &\fsm 𒆷
  &\fsm 𒉌 &\fsm 𒂠\\
  nam & til3 & la & ni & she3\\
  \multicolumn{5}{l}{(tr nam til3 la ni she3)}\\
  \multicolumn{5}{l}{\em for the sake of his life,}\\
  \hline\\
  \fcm 𒉆//𒂠
  &\multicolumn{4}{l}{(nam + genitive//she3) because of}\\
  \fcm 𒉆𒁉𒂠 
  &\multicolumn{4}{l}{(nam bi she3) because of this}\\
  \fcm 𒋾
  &\multicolumn{4}{l}{(til3) to live, to be alive}\\
  \fcm 𒋾𒆷
  &\multicolumn{4}{l}{(til3 la(k)) of his life}\\
\end{tabular}
\index{nam // she3 {\fcn 𒉆} // {\fcn 𒂠}  ! because of}
\index{nam bi she3 {\fcn 𒉆𒁉𒂠} ! because of this}
\index{til3 {\fcn 𒋾} ! to live, to be alive}
\index{til3 la {\fcn 𒋾𒆷} ! {\em gen.} of his life}

\verb||\\
The novelty in this inscription is the terminative
marked with {\fcn 𒂠} (she3). According to Wikipedia,
{\em Sumerians used the terminative case ``-še'' to
  indicate end-points in space or time and
  the targets or the goals of the action.}
\index{Terminative}

\verb||\\
\verb||\\
\begin{tabular}[!h]{l l l l l l}
  \fsm 𒀀 &\fsm 𒈬 &\fsm 𒈾
  &\fsm 𒊒\\
  a & mu & na & ru\\
  \multicolumn{4}{l}{(tr a mu na ru)}\\
  \multicolumn{4}{l}{\em dedicated it (this object).}\\
  \hline\\
  {\fcm 𒀀} // {\fcm 𒊒} 
  &\multicolumn{3}{l}{(a//ru) to dedicate}\\
\end{tabular}
\index{a // ru {\fcn 𒀀}  // {\fcn 𒊒}  ! to dedicate}

\chapter*{APPENDIX 3: Imperative}

The prefix /ḫa/ {\fcn 𒄩} expresses a request
to the second person. It can be considered
a polite form of imperative.\\
\index{Imperative}

\verb||\\
\begin{tabular}[!h]{l l l l l l l}
\fsm 𒊺 &\fsm 𒈬 &\fsm 𒄩 &\fsm 𒈬 &\fsm 𒉐\\
she     & ĝu10-my & hha & mu & tum3\\
\multicolumn{5}{l}{(tr she ĝu10-my hha mu tum3)}\\
\multicolumn{5}{l}{\em You should bring my barley.}\\
\hline\\
\fsm 𒊺 &\multicolumn{4}{l}{(she) barley, grain}\\
\fsm 𒈬 &\multicolumn{4}{l}{(ĝu10) my,
  {\em first-person possessive pronoun}}\\
\fsm 𒈬 &\multicolumn{4}{l}{(mu) {\em ventive particle}, here }\\
\fsm 𒉐 &
\multicolumn{4}{l}{(tum3) {\em Alternative form of} {\fcn 𒁺} (tum2),
  to bring }\\
\end{tabular}\verb||\\
\index{tum3 {\fcn 𒉐} ! to bring}
\index{tum2 {\fcn 𒁺} ! to bring}

\verb||\\
On the other hand, the imperative expresses
a direct command to a person. It is expressed
by re-shaping the verbal phrase: imperative
forms start with the hamtu base followed by
the prefixes of the finite verbal form.\\

\verb||\\
\begin{tabular}[!h]{l l l l l l l l l l}
  \fsm  𒋧 &\fsm 𒂷&\fsm 𒀊\\
  shum2    & ma2-me & ab-it\\
  \multicolumn{3}{l}{(tr shum2 ma2-me ab-it)}\\
  \multicolumn{3}{l}{\em Give it to me.}\\
  \hline\\
  \fsm 𒋧  &\multicolumn{2}{l}{(šum2) to give}\\
  \fsm 𒂷
  &\multicolumn{2}{l}{(ma2)
    {\em 1st-person personal pronoun:}
    to me, I, with me}\\
  \fsm 𒀊  &\multicolumn{2}{l}{(ab) {\em 3rd-person pronoun:} it}\\
\end{tabular}
\index{ma2-me {\fcn 𒂷}  ! to me, I, with me}

\newpage
\noindent
Compare the imperative with the declarative sentence:\\

\verb||\\
\begin{tabular}[!h]{l l l l l l l l}
  \fsm 𒂷 &\fsm 𒀀𒉈 &\fsm 𒋧\\
  ma2 & a ne & shum2\\
  \multicolumn{3}{l}{(tr ma2-me ane-him shum2)}\\
  \multicolumn{3}{l}{\em He gave it to me.}\\
  \hline\\
  \multicolumn{3}{l}{{\fcm 𒋧} (shum2) to give }\\
  \multicolumn{3}{l}{{\fcm 𒂷} (ma2)
    {\em personal pronoun:} to me, I, with me }\\
  \multicolumn{3}{l}{{\fcm 𒀀𒉈} (a ne) he, she }\\
  \multicolumn{3}{l}{{\fcm 𒀀𒉈𒁕𒉡𒈨𒀀} (a ne da nu me a) without him }\\
\end{tabular}\verb||\\

\verb||\\
Plural imperative forms add the
suffix {\fcn 𒌦𒍢𒂗} (tr un ze2 en) that
you will learn in a future lesson. This suffix
means ``You people.''\\

\verb||\\
\begin{tabular}[!h]{l l l l l l l l l}
  \fsm 𒎐 &\fsm 𒈬 &\fsm 𒁺 &\fsm 𒈬 &\fsm 𒌦𒍢𒂗\\
  nin9-sister & ĝu10-my & tum2 & mu & un ze2 en\\
  \multicolumn{5}{l}{(tr nin9-sister ĝu10-my tum2 mu un ze2 en)}\\
  \multicolumn{5}{l}{\em You people, bring in my sister.}\\
  \hline\\
  \multicolumn{5}{l}{ {\fcm 𒎐} (nin9) sister }\\
  \multicolumn{5}{l}{{\fcm 𒈬} (ĝu10-my) my}\\
  \multicolumn{5}{l}{{\fcm 𒈬}(mu) {\em ventive prefix}, here }\\
  \multicolumn{5}{l}{{\fcm 𒁺} (tum2) to bring,
    pl. {\fcn 𒁻}}\\
  \multicolumn{5}{l}{{\fcm 𒌦𒍢𒂗} (un ze2 en)
    {\em suffix pronoun}: you people }\\
\end{tabular}

\newpage
\noindent
Another example of imperative:\\

\verb||\\
\begin{tabular}[!h]{l l l l l l l l l l}
  \fsm 𒅗 &\fsm 𒀀𒉈 &\fsm 𒀊\\
  dug4 & ane-him & ab-it\\
  \multicolumn{3}{l}{(tr dug4 ane-him ab-it)}\\
  \multicolumn{3}{l}{\em Say it to him.}\\
  \hline\\
  \multicolumn{3}{l}{{\fcm 𒅗} (dug4) to speak, to say}\\
  \multicolumn{3}{l}{{\fcm 𒀀𒉈} (ane-him) he/him, she/her}\\
  \multicolumn{3}{l}{{\fcm 𒀊} (ab-it) {\em 3rd-person pron.:} it}\\
\end{tabular}
\index{dug4 {\fcn 𒅗} ! to speak, to say}
\index{ane-him {\fcn 𒀀𒉈} ! he/him, she/her}

\verb||\\
\verb||\\
In the imperative, the prefix {\fcn 𒉌} (i3)
is often replaced by /-a/:\\

\verb||\\
{\fsm 𒂍𒂠𒉌𒁺}\\
(tr e2 she3 i3 ĝen)\\
``He went home.''\\

\verb||\\
{\fsm 𒂍𒂠𒁺𒈾}\\
(tr e2 she3 ĝen na)\\
Go home!
\index{e2 she3 ĝen na {\fcn 𒂍𒂠𒁺𒈾} ! go home}

\newpage
\section{Conjunctions}
All languages have conjunctions
to connect sentences, and Sumerian is no exception.
An important conjunction is tukun-be2,
which means ``if.''\\

\verb||\\
\begin{tabular}[!h]{l l l l l l l l l l}
 \large\fcn 𒋗𒃻𒌉𒇲 &\large\fcn 𒁾𒁉𒋫
 &\large\fcn 𒅗 &\large\fcn 𒈬𒁕𒀊 𒌤
 &\large\fcn 𒂍𒂠 &\large\fcn 𒁺𒈾\\
 tukun-be2 & dub bu2 ta & gu3 & mu da ab de2
 & e2 she3 & ĝen na\\
 if & this tablet & out & can read & to house & go\\
 \multicolumn{6}{l}{(tr tukum dub be2 ta gu3
   mu da ab de2 e2 she3 ĝen na)}\\
 \multicolumn{6}{l}{\em If you can read out this
 tablet, go home.}\\
\end{tabular}
\index{tukun-be2 {\fcn 𒋗𒃻𒌉𒇲𒁉}  ! {\em conj.} if}

\subsection*{Vocabulary}

\verb||\\
{\fsm 𒉈𒂗} • (ne-en) this thing\\
\verb||\\
{\fsm 𒋗𒃻𒌉𒇲𒁉} • (tukun-be2) if\\
\verb||\\
{\fsm 𒅗} // {\fsm 𒌤} • (gu3//de2) to read out,
     {\em requires ablative}\\
\verb||\\
{\fsm 𒁾} • (dub) tablet\\
\verb||\\
{\fsm 𒁾𒁉} • (dub.be2) this tablet\\
\index{gu3 // de2 {\fcn 𒅗}  // {\fcn 𒌤}  ! to read out}


\verb||\\
Other important conjunctions
are {\fcn 𒅇} (tr u3) ``/also||and/,''
{\fcn 𒌓} (tr ud) ``/when/,''
{\fcn 𒌓𒁕}  (tr ud da) ``/when||if/,''
{\fcn 𒋗𒃻𒌉𒇲𒁉} (tr tukum bi) ``/if/''
and {\fcn 𒂗𒈾}  (tr en na)  ``/until/.''
\index{u3 {\fcn 𒅇} ! also, and}
\index{ud {\fcn 𒌓} ! when}
\index{ud da {\fcn 𒌓𒁕} ! when, if}
\index{tukum bi {\fcn 𒋗𒃻𒌉𒇲𒁉} ! if}
\index{en na {\fcn 𒂗𒈾} ! until} 


\newpage
\subsection*{Mark of an entrepreneur}
\verb||\\
{\fsm 𒀚 𒅇 𒉆𒆬𒍪}\\
(tr lipish u3 nam-ku3-zu)\\
Courage and Sagacity\\
\index{nam-ku3-zu {\fcn 𒉆𒆬𒍪} ! intelligence}
\index{lipish {\fcn 𒀚} ! courage}

\subsection*{Vocabulary}
\verb||\\
{\fsm 𒀚}  (lipish) emotion, anger, rage, courage\\
\verb||\\
{\fsm 𒅇}  (u3) and\\
\verb||\\
{\fsm 𒉆𒆬𒍪}  (nam-ku3-zu) intelligence, sagacity\\

\subsection*{A colaborator is a brother}
\verb||\\
{\fsm 𒍝𒂊} • {\fsm 𒅇} • {\fsm 𒂷𒂊} • {\fsm 𒋀} • {\fsm 𒈨𒂗𒉈𒂗}\\
za-e • u3 • g̃a2-e • shesh • me.en.de3.en\\
you • and • I • brothers • we are\\
You and I are brothers.\\
\index{shesh {\fcn 𒋀} ! brother}

\subsection*{Vocabulary}
\verb||\\
{\fsm 𒍝𒂊} • (za-e /ze/) Alternative form of 𒍢, you\\
\verb||\\   
{\fsm 𒅇} • (u3) and\\
\verb||\\    
{\fsm 𒂷𒂊} • (g̃a2-e) {\em Alternative form of}
  {\fcn 𒂷} (g̃e26), “I”\\
\verb||\\
{\fsm 𒋀} • (šeš /šeš/) brother, collaborator\\
\verb||\\
{\fsm 𒈨𒂗𒉈𒂗} • (me.en.de3.en) we are\\

\section*{Dedication to Gilgamesh}
Gilgamesh was the first king of Uruk. His rule
probably took place in the beginning of the
Dynastic Period, c. 2900 – 2350 BC, and he became
a major figure in Sumerian legend during the
Third Dynasty of Ur, from circa 2112 to circa 2004 BC.\\

\verb||\\
{\fsm 𒀭𒄑𒉋𒂵𒎌𒊏}\\
{\fsm 𒌨𒀭𒇉𒈗𒆠𒂗𒄀𒆠𒌵𒆤}\\
{\fsm 𒌓𒂍𒀭𒋀𒆠𒈬𒆕𒀀}\\
{\fsm 𒀀𒈬𒈾𒊒}\\

\verb||\\
{\large\fcn 𒀭𒄑𒉋𒂵𒎌 𒊏}\\
(tr bil3-ga-mesh3 ra)\\
For Gilgamesh,
\begin{quote}
{\fcm 𒊏} • (ra) {\em Dative marker}, to || for\\
\end{quote}

\verb||\\
{\fcm 𒌨𒀭𒇉} • {\fcm 𒈗} • {\fcm 𒆠𒂗𒄀}
     • {\fcm 𒆠𒌵} • {\fcm 𒆤}\\
(tr ur-nammu  • lugal  • ki-en-gi • ki uri • ke4)\\
Ur-Nammu, the king of Sumer and Akkad,\\
\begin{quote}
{\fcm 𒆠𒂗𒄀} • (ki-en-gi) Sumer\\
\verb||\\  
{\fcm 𒆠𒌵} • (ki-uri) Akkad
\end{quote}

\verb||\\
{\fcm 𒌓} • {\fcm 𒂍𒀭𒋀𒆠} • {\fcm 𒈬𒆕} • {\fcm 𒀀}\\
(tr ud • e2 an nanna • mu du3 • a)\\
when he built the temple of Nanna,\\
\begin{quote}
Obs. {\fcn 𒀀} (a) is the locative marker.\\
{\fcm 𒌓} • (ud) when\\
{\fcm 𒂍𒀭𒋀𒆠} • (e2 an nanna) temple of the god Nanna\\
{\fcm 𒈬𒆕} • (mu-du3) he built here
\end{quote}

\newpage
\noindent
{\fcm 𒉈𒂗} • {\fcm 𒀀} / {\fcm 𒈬𒈾} / {\fcm 𒊒}\\
(tr ne-en • a / mu-e / ru)\\ 
he dedicated this object.\\
\begin{quote}
{\fcm 𒉈𒂗} • (ne-en) this thing, this object\\
{\fcm 𒀀}//{\fcm 𒊒} • (a // ru) to dedicate\\
Obs. the verb {\fcn 𒀀}//{\fcn 𒊒} is split around its prefixes
\end{quote}
The {\fcn 𒀀}/ (a/) component comes before the prefix chain,
and the /{\fcn 𒊒} (/ru) component comes after.
In the vocabulary, the two components
of such a verb are separated by a double
slash, C1//C2. Examples:
\begin{quote}
{\fcm 𒅗}//{\fcm 𒌤} • (gu3//de2) to read out,
  {\em requires ablative}\\
{\fcm 𒀀}//{\fcm 𒊒} • (a // ru) to dedicate\\
{\fcm 𒊕𒄑}//{\fcm 𒊏} • (sag̃-g̃iš-ra) to commit murder
\end{quote}

\section*{Ur-Nammu's Law}
\noindent
{\fsm 𒋗𒃻𒌉𒇲𒁉 𒇽𒅇 𒊕𒄑 𒉈𒅔 𒊏}\\
{\fsm 𒇽𒁉𒉌𒄤𒂊}\\
\verb||\\
(tr tukun-be2 lu2 u3 saĝ gish bi in ra lu2 bi i3 gaz e)\\
If a man commits murder,\\
this man will be executed.\\

\verb||\\
{\fcm 𒋗𒃻𒌉𒇲𒁉} • {\fcm 𒇽} • {\fcm 𒅇}
     • {\fcm 𒊕𒄑}/ {\fcm 𒉈𒅔} /{\fcm 𒊏}\\
(tr tukun-be2 • lu2 • u3 • saĝ-gish / bi in / ra)\\
if • a man • and • commit murder,\\
\begin{quote}  
{\fcm 𒋗𒃻𒌉𒇲𒁉} • (tukun-be2) if\\
{\fcm 𒇽} • (lu2) man\\
{\fcm 𒅇} • (u3) and\\
{\fcm 𒊕𒄑}//{\fcm 𒊏} • (sag̃-g̃iš-ra) to commit murder
\end{quote}

\newpage
\noindent
{\fcm 𒇽𒁉} • {\fcm 𒉌𒄤𒂊}\\
(tr lu2 bi • i3 gaz e)\\
that man • will be executed\\
\begin{quote}
{\fcm 𒇽𒁉} • (lu2 bi or lu2 be2) this man\\
{\fcm 𒉌} • (i3) {\em finite verb marker in the verbal chain}\\
{\fcm 𒄤} • (gaz) to kill, to slaughter, to execute
\end{quote}

\subsection*{Vocabulary for the examples of conjunction}
\noindent
{\fcm 𒍝𒂊} • (za-e /ze/) {\em Alternative form of} {\fcn 𒍢}, you\\
\verb||\\    
{\fcm 𒉈𒂗} • (ne-en) this thing\\
\verb||\\
{\fcm 𒀀}// {\fcm 𒊒} • (a // ru) to dedicate\\
\verb||\\
{\fcm 𒌓} • (ud) sun, day, when\\
\verb||\\
{\fcm 𒆠𒂗𒄀} • (ki-en-gi) Sumer\\
\verb||\\
{\fcm 𒆠𒌵} • (ki-uri) Akkad\\
\verb||\\
{\fcm 𒄤} • (gaz) to kill, to execute\\
\verb||\\
{\fcm 𒊕𒄑}//{\fcm 𒊏} • (sag̃-g̃iš-ra) to commit murder\\
\verb||\\
{\fcm 𒇽} • (lu2) man\\
\verb||\\
{\fcm 𒇽𒁉} • (lu2 bi or lu2 be2) this man\\
\verb||\\
{\fcm 𒉈𒂗} • (ne en) this object\\
\verb||\\
{\fcm 𒉈} • (ne) this object\\
\verb||\\
{\fcm 𒋗𒃻𒌉𒇲𒁉} • (tukun-be2) if\\
\index{za e {\fcn 𒍝𒂊} ! Alternative form of {\fcn 𒍢}, you} 
\index{ne-en {\fcn 𒉈𒂗} ! this thing}
\index{gaz {\fcn 𒄤} ! to execute}
\index{saĝ ĝish ra {\fcn 𒊕𒄑}  // {\fcn 𒊏} ! to commit murder}
\index{lu2 bi {\fcn 𒇽𒁉}  ! this man}

%%%%begin new indexes
\chapter{Ur-Nammu-23}

{\fsm
𒀭𒂗𒆤\\
𒈗𒆳𒆳𒊏\\
𒈗𒀀𒉌\\
𒌨𒀭𒇉\\
𒈗𒋀𒀊𒆠𒈠\\
𒈗𒆠𒂗𒄀𒆠𒌵𒆤\\
𒂍𒀀𒉌\\
𒈬𒈾𒆕\\
𒀀𒇉𒂗𒂟𒉣\\
𒀀𒇉𒉻𒀭𒈹𒅗𒉌\\
𒈬𒈾𒁀𒀠\\
}
\verb||\\
Translation:
{\em For Enil, the king of all the lands, his master,
  Ur-Nammu, the king of Sumer and Akkad, built his temple.
  The king also dredged the Enerinnun canal for Enlil.}

\newpage
\section{Sentence structure}
The text starts with the benefactive,
that ends in \verb|{(r)}|, not expressed.

{\fsm 𒀭𒂗𒆤

𒈗𒆳𒆳𒊏

𒈗𒀀𒉌} 

The plural of nouns that refer to human beings
is formed by a suffixed ``\verb|ene|''. The plural of things,
plants and animals is usually unmarked.
Reduplication
-- such as ``\verb|kur kur|'' ({\fcn 𒆳𒆳}) --
conveys the idea of totality: "all the lands."

Then comes the agent, with the E prefix
combined with the genitive ending
into KE4 ({\fcn 𒆤}).

The next in the line is the object that
was built, to wit, his temple {\fcn 𒂍𒀀𒉌}
(tr e2 a ni).
\begin{verbatim}
1- [enlil                          -- For Enlil,
2-   [lugal kur kur].{ra(K)}       -- the king of all lands,
3-   [lugar ani]].{(r)}            -- his king,
4- [ur nammu                       -- Ur-Nammu,
5-   [lugal urim].{ma(k)}          -- king of Ur,
6-   lugal [kiengi kiuri].{k}].{e} -- king of Sumer and Akkad,
7- [e2 ani].{}                     -- his temple
8- mu na du3                       -- build
9- [id2 en erin2 nun               -- The Enerinnun canal,
10- [id2 nidba].{k}.ani].{}        -- his canal of food offering,
11- mu na ba al                    -- (the king) dredged for him.
\end{verbatim}

\section{Annotations}

\begin{tabular}[!h]{l l l l l l l l}
  \fcm 𒀭𒂗𒆤 &\fcm 𒈗 &\fcm 𒆳𒆳𒊏 &\fcm 𒈗𒀀𒉌\\
  an en lil2  & lugal     & kur kur ra & lugal a ni\\
  \multicolumn{4}{l}{(tr an en lil2 lugal kur kur ra lugal a ni) }\\
  \multicolumn{4}{l}{\em For Enlil, king of all lands, his master,}\\
  \hline\\
  \multicolumn{4}{l}{{\fcm 𒀭𒂗𒆤}
     (d-en-lil2) Enlil, the king of gods}\\
  \multicolumn{4}{l}{{\fcm 𒆳}  (kur) mountain,land, country}\\
\end{tabular}
\index{kur {\fcn 𒆳} ! mountain, land, country}

\newpage
\noindent
\begin{tabular}[!h]{l l l l l l l l l}
\fcm 𒌨𒀭𒇉 &\fcm 𒈗 &\fcm 𒋀𒀊𒆠 &\fcm 𒈠\\
(tr ur-nammu) & (tr lugal) & (tr urim5) & (tr ma)\\
\multicolumn{4}{l} {(tr ur-nammu lugal urim5 ma)}\\
\multicolumn{4}{l} {\em Ur-Nammu, the king of Ur,}\\
\end{tabular}\verb||\\

\verb||\\
\begin{tabular}[!h]{l l l l l l l l l}
\fcm 𒈗 &\fcm 𒆠𒂗𒄀 &\fcm 𒆠𒌵 &\fcm 𒆤\\
(tr lugal) & (tr ki-en-gi) & (tr ki uri) & (tr ke4)\\
\multicolumn{4}{l} {(tr lugal ki-en-gi ki uri ke4)}\\
\multicolumn{4}{l} {\em the king of Sumer and Akkad, }\\
\end{tabular}\verb||\\

\verb||\\
\begin{tabular}[!h]{l l l l l l l l l}
\fcm 𒂍 &\fcm 𒀀𒉌 &\fcm 𒈬𒈾𒆕\\
(tr e2) & (tr a ni) & (tr mu na du3)\\
\multicolumn{3}{l} {(tr e2 a ni mu na du3)}\\
\multicolumn{3}{l} {\em he has built the god's temple.}\\
\end{tabular}\\

\verb||\\
\begin{tabular}[!h]{l l l l l l l l l}
\fcm 𒀀𒇉 &\fcm 𒂗 &\fcm 𒂟 &\fcm 𒉣\\
(tr id2) & (tr en) & (tr erin2) & (tr nun)\\
\multicolumn{4}{l} {(tr id2 en erin2 nun)}\\
\multicolumn{4}{l} {\em The Enerinnun canal,}\\
\end{tabular}\\
\index{id2 {\fcn 𒀀𒇉} ! canal}

\verb||\\
\begin{tabular}[!h]{l l l l l l l l l}
\fcm 𒀀𒇉 &\fcm 𒉻𒀭𒈹 &\fcm 𒅗 &\fcm 𒉌 &\fcm 𒈬𒈾𒁀𒀠\\
(tr id2) & (tr nidba) & (tr ka)
  & (tr ni) & (tr mu na ba al)\\
\multicolumn{5}{l} {(tr id2 nidba ka ni mu na ba al)}\\
\multicolumn{5}{l} {\em his canal of food offerings,
  the king dredged for him.}\\
\end{tabular}\\
\index{nidba ka {\fcn 𒉻𒀭𒈹𒅗}  ! food offerings}
\index{ba al {\fcn 𒁀𒀠} ! to excavate, to dig}


\subsection*{Vocabulary}

\verb||\\
{\fcm 𒀀𒇉}  (id2) river, watercourse, canal\\

\noindent
{\fcm 𒉻}  (kurum6) food ration\\
\index{kurum6 {\fcn 𒉻}  ! food ration}

\noindent
{\fcm 𒉻𒀭𒈹𒅗} (kurum6 inanna.k) food of Inanna\\
\index{kurum6 an inanna {\fcn 𒉻𒀭𒈹𒅗} ! food of Inanna}

\noindent
{\fcm 𒁀}// {\fcm 𒀠}  (ba-al) to excavate, to dig\\

\section{Proper Adjectives}
On page~\pageref{true-adjectives}, you learned
that most Sumerian adjectives are formed
from verbs that express qualifications, such as
``{\em te be mighty}'' (mahh {\fcn 𒈤}).
The {\fcn 𒀀}  (a) suffix transforms such
verbs into adjectives. However, there are
proper adjectives that you should learn
by heart since they are very few.
Below is the rest of the list of proper adjectives.\\


\verb||\\
\begin{tabular}[!h]{l l l l l l l l}
  {\fcm 𒈗}   •  {\fcm 𒅆𒂠} & lugal •  libir & former king\\
  &&\\
  {\fcm 𒅗}  • {\fcm 𒍣}  & inim • zid & true word\\
  &&\\
  {\fcm 𒅗}   • {\fcm 𒈜} & inim • lul & false word\\
  &&\\
  {\fcm 𒊺}  • {\fcm 𒅆𒌨}  & (she) • (hhulu) & bad barley\\
  &&\\
  {\fcm 𒈾}  • {\fcm 𒂂}  & (na) • (dugud) & heavy stone\\
  &&\\
  {\fcm 𒃻}  • {\fcm 𒄭} & (ninda)  • (du10) & sweet food\\
  &&\\
  {\fcm 𒁈𒋛𒂵}  • {\fcm 𒆬}  & (barag sig9 ga) • (ku3) & holly shrine\\
  &&\\
  {\fcm 𒃻}  • {\fcm 𒂖}  & (ĝar)  • (sikil) & clean place\\
  &&\\
  {\fcm 𒊩}   • {\fcm 𒁲}  & (munus)  • (silim) & healthy woman\\
  &&\\
  {\fcm 𒃻}   • {\fcm 𒋀}  & (ninda)  • (sis) & bitter food\\
\end{tabular}
\index{libir {\fcn 𒅆𒂠} ! former }
\index{zid {\fcn 𒍣}  ! true}
\index{lul {\fcn 𒈜}  ! false}
\index{dugud {\fcn 𒂂}  ! heavy}
\index{du10 {\fcn 𒄭}  ! sweet}
\index{ku3 {\fcn 𒆬}  ! holly}
\index{sikil {\fcn 𒂖}  ! clean}
\index{silim {\fcn 𒁲}  ! healthy}
\index{sis {\fcn 𒋀}  ! bitter}



\chapter*{Appendix 4: Numerals}

You completed lesson four. Then you should
be able to read many votive Sumerian artifacts
that you find in museums around the world.

To boost your reading skills, I recommend that
you go back to the first lesson and read
the contents of the {\bf APPENDIX: Gramar notes}.
Thus, you will improve your holding of case
elements, and learn that the dative changes
depending on the person to whom the scribe
is dedicating a building or an object.
You will also receive introductory concepts
of transitive verbs, intransitive verbs,
the hamtu and the marû conjugation.

However, before returning to the first lesson,
you may want to read about Sumerian numerals,
and learn to count things in cuneiform.

\section{Sumerian Numerals}
\label{basic-numbers}
To count things, modern people use ten digits:
1, 2, 3, 4, 5, 6, 7, 8, 9 and 0.
Therefore, it is said that we use base 10.
Computers use only two digits to perform
calculations: 0 and 1. Then, computers work
with base 2. As we will see below, Sumerians
used base 60. Therefore, they needed 60 digits
to count things.
\index{Sumerian numerals}

You will be happy to learn that we still use
the Sumerian method of counting when we deal
with navigation and time. That is the reason
for having 60 minutes in an hour and 60 seconds
in a minute. Besides this, the latitude and
the longitude that determines a position on
the Earth's surface is measured in degrees,
where each degree is divided in 60 minutes.

Since the distance from the North Pole
is 90 degrees and ten thousand km, each
degree of latitude has 10000/90,
roughly 111 km. If you divide 111 km
by 60 to discover the length of one minute,
you get 1852 km, which is a nautical mile.


\subsection*{Numbers from 1 to 9: dish}


\begin{tabular}[!h]{l l | l l| l l|}
 {\fcm 𒁹}  & 1 -- (tr dish)   & {\fcm 𒈫} & 2 -- (tr 2-dish) & 
 {\fcm 𒐈}  &  3 -- (tr 3-dish) \\
 &&&&&\\
 {\fcm 𒐉}  & 4 -- (tr 4-dish) & {\fcm 𒐊}  & 5 -- (tr 5-dish) &   
 {\fcm 𒐋}  &  6 -- (tr 6-dish)  \\
 &&&&&\\
 {\fcm 𒐌} & 7 -- (tr 7-dish) & {\fcm 𒐍}  & 8 -- (tr 8-dish) & 
 {\fcm 𒐎} &  9 -- (tr 9-dish)     \\
\end{tabular}

\index{dish {\fcn 𒁹}  ! one}
\index{2-dish {\fcn 𒈫}   ! two}
\index{3-dish {\fcn 𒐈} ! three}
\index{4-dish {\fcn 𒐉}  ! four}
\index{5-dish {\fcn 𒐊}   ! five}
\index{6-dish {\fcn 𒐋}   ! six}
\index{7-dish {\fcn 𒐌}   ! seven}
\index{8-dish {\fcn 𒐍}   ! eight}
\index{9-dish {\fcn 𒐎}   ! nine}

\subsection*{Numbers from 10 to 50}

\begin{tabular}[!h]{l l | l l| l l|}
 {\fcm 𒌋}   & 10 -- (tr 1-u)   & {\fcm 𒎙}  & 20 -- (tr 2-u) & 
 {\fcm 𒌍}   &  30 -- (tr 3-u) \\
 &&&&&\\
 {\fcm 𒐏}   & 40 -- (tr 4-u) & {\fcm 𒐐}   & 50 -- (tr 5-u) &   
 {\fcm 𒐏𒈫}   &  42 -- (tr 4-u 2-dish)  \\
 \end{tabular}

%% \item 54 (tr 5-u 4-dish) {\fcm 𒐐𒐉} 
%% \item 42 (tr 4-u 2-dish) {\fcm 𒐏𒈫} 

\index{1-u {\fcn 𒌋} ! ten}
\index{2-u {\fcn 𒎙}  ! twenty}
\index{3-u {\fcn 𒌍} ! thirty}
\index{4-u {\fcn 𒐏}  ! forty}
\index{5-u {\fcn 𒐐}  ! fifty}

\index{4-u 2-dish {\fcn 𒐏𒈫}  ! 42}

\subsection*{Numbers from 60 to 360}
In the same way that
we use the digit 1 to represent
the numbers one and  ten,
the Sumerians used {\fcn 𒐕}  to represent
both one and sixty.
 
\verb||\\
\begin{tabular}[!h]{l l | l l|l l}
  {\fcm 𒐕}    & 60 -- (ĝesh)   & {\fcm 𒐖}   & 120 -- (2-ĝesh)
  & {\fcm 𒐗}    & 180 -- (3-ĝesh)  \\
  &&&&&\\
  {\fcm 𒐘}     &  240 -- (4-ĝesh) & {\fcm 𒐙}    & 300 -- (5-ĝesh)
  & {\fcm 𒐚}     & 360 -- (6-ĝesh)  \\
  &&&&&\\
  {\fcm 𒐛}    & 420 -- (7-ĝesh) & {\fcm 𒐜}    & 480 -- (8-ĝesh)
  & {\fcm 𒐝}     & 540 -- (9-ĝesh)  \\
\end{tabular}

\index{ĝesh {\fcn 𒐕} ! sixty}
\index{2-ĝesh {\fcn 𒐖}  ! 120}

\subsection*{Numbers from 600 to 3600}
The geshu wedges are used to represent
both multiples of 600 and the
numbers 70, 80, 90, 100 and 110,
as shown below.

\verb||\\
\begin{tabular}[!h]{l l | l l|}
  {\fcm 𒐞}      & 600/70 -- (tr 1-geshu)
  & {\fcm 𒐟}     & 1200/80 -- (tr 2-geshu)\\
 &&&\\
 {\fcm 𒐠}      &  1800/90 -- (tr 3-geshu) 
 & {\fcm 𒐡}   & 4 -- (tr 4-geshu)\\
&&&\\
 {\fcm 𒐢}   & 5 -- (tr 5-geshu)
 &  {\fcm 𒊹}   &  3600 -- (tr 1-shar2)  \\
 \end{tabular}


\chapter{Ur-Nammu-5}
\begin{quotation}
{\fsm 𒀭𒈗𒀭𒊑𒉈} \\
    
{\fsm 𒈗𒀀𒉌} \\

{\fsm 𒌨𒀭𒇉} \\

{\fsm 𒈗𒋀𒀊𒆠𒈠𒆤} \\

{\fsm 𒄑𒊬𒈤} \\

{\fsm 𒈬𒈾𒁺} \\

{\fsm 𒁈𒆠𒂖𒆷} \\

{\fsm 𒈬𒈾} 
\end{quotation}
Translation:
{\em Ur-Nammu, the king of Ur, planted a magnificent
  garden for An, the king of gods. He also has built
  a dais in a pure place for the god.}

\newpage
\section{Sentence structure}
\begin{verbatim}
1- [an lugal [diĝir.{re.ne}]      -- For An, king of the gods,
2-    lugal a ni].{(r)}           -- his master,
3- [ur-nammu                      -- Ur-Nammu,
4-   [lugal urim5].{ma (k)}].{e}  -- the king of Ur,
5- [gish kiri6 mah].{Ø}           -- an outstandig garden
6- mu na gub                      -- planted.
7- [barag [ki sikil].{la}].{Ø}    -- a dais in a pure place
8- mu na du3                      -- (the king) built (for An).
\end{verbatim}

\section{Annotations}
\verb||\\
\begin{tabular}[!h]{l l l l l l l l l}
\fcm 𒀭 &\fcm 𒈗 &\fcm 𒀭 &\fcm 𒊑𒉈\\
(tr an) & (tr lugal) & (tr digir) & (tr re ne)\\
\multicolumn{4}{l} {(tr an lugal digir re ne)}\\
\multicolumn{4}{l} {\em For An, the king of the gods,}\\
\hline\\
\multicolumn{4}{l} {{\fcm 𒀭} (an) sky, the sky god Ān }\\
\multicolumn{4}{l} {{\fcm 𒀭} (digir) deity, god/goddess }\\
\multicolumn{4}{l} {{\fcm 𒀭𒊑𒉈} (diggir-rene) gods }\\
\end{tabular}\\
\index{an {\fcn 𒀭} ! sky, the sky god An}
\index{digir {\fcn 𒀭} ! god/goddess}
\index{digir digir re ne {\fcn 𒀭𒀭𒊑𒉈}  ! gods}


\verb||\\
\begin{tabular}[!h]{l l l l l l l l l}
\fcm 𒈗 &\fcm 𒀀𒉌\\
(tr lugal) & (tr a ni)\\
\multicolumn{2}{l} {(tr lugal a ni)}\\
\multicolumn{2}{l} {\em his master, }\\
\end{tabular}\\

\verb||\\
\begin{tabular}[!h]{l l l l l l l l l}
\fcm 𒌨𒀭𒇉 &\fcm 𒈗 &\fcm 𒋀𒀊𒆠𒈠𒆤\\
(tr ur-nammu) & (tr lugal) & (tr urim5 ma ke4)\\
\multicolumn{3}{l} {(tr ur-nammu lugal urim5 ma ke4)}\\
\multicolumn{3}{l} {\em the king of Ur,}\\
\end{tabular}\\

\verb||\\
\begin{tabular}[!h]{l l l l l l l l l}
\fcm 𒄑 &\fcm 𒊬 &\fcm 𒈤 &\fcm 𒈬𒈾𒁺\\
(tr gish) & (tr kiri6) & (tr mah) & (tr mu na gub)\\
\multicolumn{4}{l} {(tr gish kiri6 mah mu na gub)}\\
\multicolumn{4}{l} {\em a magnificent garden he planted.}\\
\end{tabular}\\

\newpage
\begin{tabular}[!h]{l l l l l l l l l}
\fcm 𒁈 &\fcm 𒆠 &\fcm 𒂖 &\fcm 𒆷 &\fcm 𒈬𒈾𒆕\\
(tr barag) & (tr ki) & (tr sikil) & (tr la) & (tr mu na du3)\\
\multicolumn{5}{l} {(tr barag ki sikil la mu na du3)}\\
\multicolumn{5}{l} {\em He also has built a dais
  in a pure place for An.  }\\
\end{tabular}\\

\subsection*{Vocabulary}
\noindent
{\fcm 𒄑}   (g̃eš, g̃iš) tree\\
\index{ĝish {\fcn 𒄑} ! tree}

\noindent
{\fcm 𒊬}   (kiri6) orchard, garden plot\\
\index{kiri6 {\fcn 𒊬} ! orchad, garden plot}

\noindent
{\fcm 𒈤}   (maḫ) to be lofty, magnificent\\
\index{mah {\fcn 𒈤}  ! to be magnificent}

\noindent
{\fcm 𒁈}   (barag) dais, throne\\
\indent{barag {\fcn 𒁈}  ! dais, throne}

\noindent
{\fcm 𒂖}   (sikil) to be pure, clean\\
\indent{sikil {\fcn 𒂖}  ! to be clean}

\section{Writing numbers}
On page~\pageref{basic-numbers}, you learned
how to write basic numerals in Sumerian.
In this section, you will learn how to combine
these basic numerals.

Modern people use ten digits to count:
1, 2, 3, 4, 5, 6,
7, 8, 9 and 0. Therefore, computer scientists
say the contemporary world uses base ten.
In base 10, a digit $n$ can
represent $n\times 1$, $n\times 10$, $n\times 100$, etc.

How do we know which value the digit stands
for? By its position in the numerical string.
If the digit comes first from right to left,
it simply represents its unities.
If it comes second, its basic value must
be multiplied by ten. If it comes third
in the numerical string, its basic value
is multiplied by 100. Then, 342 represents
$3\times 100 + 4\times 10 + 2$.

A system where the interpretation of a digit
depends on its position in the numerical
string is called place-value notation
or positional numerical notation. Like us,
Sumerians used a place value notation,
but they counted in base 60. Therefore,
the symbol {\fcn 𒁹} can be assigned the
values 1, 60, 3600, etc. If {\fcn 𒁹}
comes first in the numerical string,
then its value is 1. If it comes
second, it represents 60. If it
is the third digit from left to right,
it represents $1\times 60\times 60$,
which produces 3600.

There are further details in this story that we need to
clarify. Modern arithmetic students have zero to fill the empty places
in a numerical figure. Therefore,
they can interpret a value without
ambiguity. Sumerians invented zero
only late in their history. Then,
they needed to interpret the
number from context. They could
also place the numerical string
in boxes and leave empty 
boxes where we would place a zero.
Let us see one example. Below,
you can see the number 1273.\\

\verb||\\
\begin{tabular}[!h]{| l | l | l | l l l l l l}
  \hline
  &&\\
\fcm 𒐖 &\fcm 𒐞 &\fcm 𒐈\\
\hline
&&\\ 
(tr 2-ĝesh) & (tr 1-geshu) & (tr 3-dish)\\
\hline
$2\times 60$ & $60+10$ & $3$\\
1200 & 70 & 3\\
\hline
\end{tabular}\\

\vspace{0.4cm}
Cardinal numbers show how many things
one is dealing with. In Sumerian,
cardinal numbers come after the noun,
exactly like adjectives. Therefore,
{\fcn 𒈬𒅓} (mu imin) means
{\em ``seven years''}. Here,
{\fcn 𒅓} (imin) is the name of
the number seven. Just like one
can write 7 or seven in English,
one can say (7-dish) or (imin)
in Sumerian. Consider the example below.

\verb||\\
\begin{tabular}[!h]{l l l l l l l l l}
  \fcm 𒀭𒈾𒊏𒄠𒀭𒂗𒍪 &\fcm 𒈬𒅓𒀀𒀭
  &\fcm 𒈬𒌦𒄀𒂗\\
(tr an na ra am an suen) & (tr mu imin am3)
& (tr mu un ge en)\\
\multicolumn{3}{l} {\em For seven years,
  Naram-Suen was motionless.}\\
\hline\\
\multicolumn{3}{l} {{\fcm 𒈬}  (mu) year}\\
\multicolumn{3}{l}{{\fcm 𒄀} (ge) to be firm,
  to be motionless }
\end{tabular}\\
\index{Cardinal numbers}
\index{mu {\fcn 𒈬} ! year}
\index{ge {\fcn 𒄀} !  to be firm}

\verb||\\
A literal translation of {\fcn 𒈬𒅓𒀀𒀭}
(mu imin am3) could be: {\em ``years
  that are seven.''} Another example:

\verb||\\
\begin{tabular}[!h]{l l l l l l l l l}
\fcm 𒉣𒈨𒅓𒂊 &\fcm 𒋝𒉏𒋫 &\fcm 𒋗𒈬𒊏𒅔𒊬𒍑\\
(tr abgal imin e) & (tr sig nim ta) &
(tr shu mu ra in mu2 ush)\\
\multicolumn{3}{l} {\em The seven sages have
  enlarged it for you in the lowlands and highlands.}\\
\hline\\
\multicolumn{3}{l}{{\fcm 𒉣𒈨} (abgal) sage}\\
\multicolumn{3}{l}{{\fcm 𒋝}  (sig) lowland}\\
\multicolumn{3}{l}{{\fcm 𒉏}  (nim) highland}\\
\multicolumn{3}{l}{{\fcm 𒋗} // {\fcm 𒊬}
  (shu//mu2) to enlarge}
\end{tabular}\\
\index{abgal {\fcn 𒉣𒈨} !  sage}
\index{sig {\fcn 𒋝} !  lowland}
\index{nim  {\fcn 𒉏} !  highland}
\index{shu//mu2 {\fcn 𒋗} //{\fcn 𒊬} ! to enlarge}



\chapter{Ama gi}
\begin{quotation}\fsm\onehalfspacing
\noindent
𒀭𒈗𒂍𒈹𒊏\\
𒂗𒋼𒈨𒈾\\
𒊮𒅆𒊒 𒁕 𒀭𒀏 𒆤\\
𒄑𒌆𒉿~𒋧𒈠 ~ 𒀭𒂗𒆠𒅗𒆤\\
𒉺𒋼𒋛𒃲𒀭𒎏𒄈𒋢\\
𒌉𒂗𒀭𒈾𒁺\\
𒉺𒋼𒋛 𒉢𒁓𒆷𒆠𒅗𒆤\\
𒀭𒎏𒄈𒋢𒊏\\
𒀊𒂁𒊒\\
𒈬𒈾𒆕\\

\noindent
𒂼𒄄𒉢𒁓𒆷𒆠~~~ 𒂊𒃻\\
𒂼𒌉𒈬𒉌𒄄\\
𒌉𒂼𒈬𒉌𒄄
\end{quotation}

\newpage
\section{Translation}
{\em For Lugalemush, Entemena,
 the chosen of Nanshe's heart,
 the general governor of Ningirsu,
 the son of Enannatum,
 the governor of Lagash,
 has built the shrine of Dugru.
 He instituted a remission of Lagash's obligations.
 He returned the mother to her children.
 He returned the children to their mother.}

\section{Annotations}

\begin{tabular}[!h]{l l l l l l l l l}
\fcm 𒀭 &\fcm 𒈗𒂍𒈹 &\fcm 𒊏\\
(tr an) & (tr lugal e2 mush3) & (tr ra)\\
\multicolumn{3}{l} {(tr an lugal e2 mush3 ra)}\\
\multicolumn{3}{l} {\em For the divine Lugalemush}\\
\hline\\
\multicolumn{3}{l} { {\fcm 𒀭𒈗𒂍𒈹} (an lugal e2 mush3)
   Lugalemush, {\em Inanna's husband}}\\
\end{tabular}\\
\index{lugal e2 mush3 {\fcn 𒈗𒂍𒈹}
  ! the god Lugalemush}

\sepstars

\verb||\\
\begin{tabular}[!h]{l l l l l l l l l}
\fcm 𒂗𒋼𒈨𒈾 &\fcm 𒊮 &\fcm 𒅆𒊒 𒁕 &\fcm 𒀭𒀏 𒆤\\
(tr en-te-me-na) & (tr sha3) & (tr pad3 da)
& (tr nanshe ke4)\\
\multicolumn{4}{l} {(tr en-te-me-na sha3
  pad3 da nanshe ke4)}\\
\multicolumn{4}{l}
            {\em Entemena, the chosen of Nanshe's heart,}\\
\hline\\
\multicolumn{4}{l}
    { {\fcm 𒂗𒋼𒈨𒈾} (en-te-me-na) Entemena }\\
\multicolumn{4}{l} {{\fcm 𒊮}   (sha3) heart}\\
\multicolumn{4}{l} {{\fcm 𒊮 𒈬𒁀𒅗}
  (tr sha3 mu ba ka)  in the middle of that year }\\
\multicolumn{4}{l} {{\fcm 𒅆𒊒} (pad3) to find, to choose }\\
\multicolumn{4}{l} {{\fcm 𒀭𒀏}
  (nanshe) the goddess Nanshe }\\
\end{tabular}\\
\index{en-te-me-na {\fcn 𒂗𒋼𒈨𒈾}
  ! Entemena of Lagash }
\index{sha3 {\fcn 𒊮} ! heart, middle}
\index{sha3 mu ba ka {\fcn 𒊮𒈬𒁀𒅗}
  ! in the middle of that year}
\index{pad3 {\fcn 𒅆𒊒} ! to find, to choose}
\index{nanshe {\fcn 𒀭𒀏} ! the goddess Nanshe}

\sepstars

\verb||\\
\begin{tabular}[!h]{l l l l l l l l l}
\fcm 𒄑𒌆𒉿 &\fcm 𒋧𒈠 &\fcm 𒀭𒂗𒆠𒅗𒆤\\
(tr geshtug2) & (tr shum2 ma) & (tr enki ka ke4)\\
\multicolumn{3}{l} {(tr geshtug2 shum2 ma enki ka ke4)}\\
\multicolumn{3}{l} {\em given wisdom by Enki,}\\
\hline\\
\multicolumn{3}{l}{{\fcm 𒄑𒌆𒉿}  (geshtug2) intelligence}\\
\multicolumn{3}{l}{{\fcm 𒋧}  (shum2) to give}\\
\end{tabular}\\


\newpage
\noindent
\begin{tabular}[!h]{l l l l l l l l l}
\fcm 𒉺𒋼𒋛 &\fcm 𒃲 &\fcm 𒀭𒎏𒄈𒋢\\
(tr ensi2) & (tr gal) & (tr an ningirsu)\\
\multicolumn{3}{l} {(tr ensi2 gal an ningirsu)}\\
\multicolumn{3}{l} {\em the general governor of Ningirsu }\\
\hline\\
\multicolumn{3}{l} {{\fcm 𒉺𒋼𒋛} (ensi2) governor}\\
\multicolumn{3}{l} {{\fcm 𒃲}  (gal),
  pl. {\fcm 𒃲𒃲}  (gal gal) big, large, great}\\
\multicolumn{3}{l} {{\fcn 𒎏𒄈𒋢} (ningirsu)
  the city of Ningirsu}\\
\end{tabular}\\
\index{ensi2 {\fcn 𒉺𒋼𒋛} ! governor, ruler}
\index{gal {\fcn 𒃲} ! large, great}
\index{ningirsu {\fcn 𒎏𒄈𒋢} ! the city of Ningirsu}

\sepstars

\verb||\\
\begin{tabular}[!h]{l l l l l l l l l}
\fcm 𒌉 &\fcm 𒂗𒀭𒈾𒁺\\
(tr dumu) & (tr en-an-na-tum2)\\
\multicolumn{2}{l} {(tr dumu en-an-na-tum2)}\\
\multicolumn{2}{l} {\em the son of Enannatum,}\\
\hline\\
\multicolumn{2}{l}{{\fcm 𒌉}
  (dumu) child, son, daughter}\\
\multicolumn{2}{l}{{\fcm 𒂗𒀭𒈾𒁺}  (en-an-na-tum2)
  Enannatum I of Lagash}\\
\end{tabular}\\
\index{dumu {\fcn 𒌉} ! son, daughter}
\index{en-an-na-tum2 {\fcn 𒂗𒀭𒈾𒁺} ! Enannatum I}

\sepstars

\verb||\\
\begin{tabular}[!h]{l l l l l l l l l}
\fcm 𒉺𒋼𒋛 &\fcm 𒉢𒁓𒆷𒆠 &\fcm 𒅗 &\fcm 𒆤\\
(tr ensi2) & (tr lagash ki) & (tr ka) & (tr ke4)\\
\multicolumn{4}{l} {(tr ensi2 lagash ki ka ke4)}\\
\multicolumn{4}{l} {\em the governor of Lagash,}\\
\hline\\
\multicolumn{4}{l}{{\fcm 𒉢𒁓𒆷} (lagash)
  the city of Lagash}\\
\multicolumn{4}{l}{{\fcm 𒆤} (ke4) {\em gen. contracted
  with erg.}}
\end{tabular}\\
\index{lagash {\fcn 𒉢𒁓𒆷} ! the city of Lagash} 

\sepstars

\newpage
\noindent
\begin{tabular}[!h]{l l l l l l l l l}
\fcm 𒀭𒎏𒄈𒋢𒊏 &\fcm 𒀊𒂁𒊒 &\fcm 𒈬𒈾𒆕\\
(tr an ningirsu ra) & (tr esh3 dug ru) & (tr mu na du3)\\
\multicolumn{3}{l} {(tr an ningirsu ra esh3
  dug ru mu na du3)}\\
\multicolumn{3}{l} {\em he built the shrine
   of Dugru for Ningirsu. }\\
\end{tabular}\\
\index{esh3 {\fcn 𒀊} ! shrine}
\index{ningirsu {\fcn 𒎏𒄈𒋢} ! the city of Ningirsu}

\sepstars

%%%%daqui
\verb||\\
\begin{tabular}[!h]{l l l l l l l l l}
\fcm 𒂼𒄄 &\fcm 𒉢𒁓𒆷𒆠 &\fcm 𒂊𒃻\\
(tr ama gi4) & (tr lagash ki) & (tr e gar)\\
\multicolumn{3}{l} {(tr ama gi4 lagash ki e gar)}\\
\multicolumn{3}{l}
   {\em He instituted a remission of the obligations 
     of Lagash. }\\
\hline\\
\multicolumn{3}{l}{{\fcm 𒂼𒄄}  (ama gi4)
  freedom from debt or bondage}\\
\multicolumn{3}{l}{{\fcm 𒃻}  (gar),
  marû {\fcm 𒂷𒂷}  (gar gar) to institute}\\
\multicolumn{3}{l}{{\fcm 𒂊} (e)
  {\em Finite verb marker before roots containing
           the vowel ``a''}}\\
\end{tabular}\\
\index{ama gi4 {\fcn 𒂼𒄄}
  ! freedom from debt}\\
\index{gar {\fcn 𒃻}  ! to institute}\\
\index{e {\fcn 𒂊} ! {\em finite verb marker}}\\

\sepstars

Ama-gi4 means freedom, liberty,
the right to return to one's mother,
and remission. The verb e-gar ({\fcn 𒂊𒃻} )
means ``to place'' but conveys the idea
of ``to institute.''

\sepstars

According to Marie-Louise Thomsen,
the finite verb Conjugation Prefix
{\fcn 𒉌}  (i3) has the
variant {\fcn 𒂊}  (e) in Old Sumerian
texts from Lagash, Uruk, Ur and Umma.
In particular {\fcn 𒂊}  /-e/ is
used immediately before verbal roots
containing the vowels ``a'' and ``e'',
e.g., {\fcn 𒂊𒃻}  (e ĝar),
{\fcn 𒂊𒀝}  (e ak),
{\fcn 𒂊𒇲}  (e la2)
and {\fcn 𒂊𒈨𒀀}  (e me a). The finite
verb marker {\fcn 𒂊}  (e) is also
used before the case elements
{\fcn 𒁕}  \verb|/-da-/|,
{\fcn 𒈾}  \verb|/-na-/|,
{\fcn 𒉈}  \verb|/-ne-/|,
{\fcn 𒊺}  \verb|/-she-/|
and {\fcn 𒋫}  \verb|/-ta-/|.
%%%%até

\newpage
\noindent
\begin{tabular}[!h]{l l l l l l l l l}
\fcm 𒂼 &\fcm 𒌉 &\fcm 𒈬𒉌𒄄\\
(tr ama) & (tr dumu) & (tr mu ni gi4)\\
\multicolumn{3}{l} {(tr ama dumu mu ni gi4)}\\
\multicolumn{3}{l}{\em He returned the
  mother to her child. }\\
\hline\\
\multicolumn{3}{l}{ {\fcm 𒂼}  (ama) mother}\\
\multicolumn{3}{l}{ {\fcm 𒌉}  (dumu) child}\\
\multicolumn{3}{l}{{\fcm 𒄄} (gi4) to send back}\\
\multicolumn{3}{l}{{\fcm 𒉌} (ni) {\em loc.,} to the place}\\
\multicolumn{3}{l}{{\fcm 𒈬} (mu) {\em ventive,} here}\\
\end{tabular}\\
\index{ama {\fcn 𒂼} ! mother}
\index{dumu {\fcn 𒌉} ! child}
\index{gi4 {\fcn 𒄄} ! to send back}
\index{ni {\fcn 𒉌} ! {\em loc.} back to}
\index{mu {\fcn 𒈬}  ! {\em ventive}, here}

\sepstars

\verb||\\
\begin{tabular}[!h]{l l l l l l l l l}
\fcm 𒌉 &\fcm 𒂼 &\fcm 𒈬𒉌𒄄\\
(tr dumu) & (tr ama) & (tr mu ni gi4)\\
\multicolumn{3}{l} {(tr dumu ama mu ni gi4)}\\
\multicolumn{3}{l} {\em He returned the child
  to her mother.}\\
\end{tabular}\\

\newpage
\section{Reduced relative clause}
Suppose you want to say, in Sumerian,
that {\em ``Nanshe's heart has chosen Entemena.''}
You could write something thus:

\verb||\\
\begin{tabular}[!h]{l l l l l l l l l}
\fcm 𒊮𒀭𒀏𒆤 &\fcm 𒂗𒋼𒈨𒈾 &\fcm 𒉌𒅆𒊒\\
(tr sha3 nanshe ke4) & (tr en-te-me-na) & (tr i3 pad3)\\
\multicolumn{3}{l} {(tr sha3 nanshe ke4 en-te-me-na i3 pad3)}\\
\multicolumn{3}{l} {\em Nanshe's heart has chosen Entemena.}\\
\end{tabular}\\

In the example, the reader knows that the
heart of Nanshe performs the task due
to the genitive/ergative
marker {\fcn 𒆤}  (ke4).

In English, one uses a relative
clause to say something like that:
{\em ``Entemena, whom Nanshe's heart
has chosen, built the shrine of
Dugru for Ningirsu.''} In this example,
to qualify Entemena, 
one uses the clause: {\em "whom Nanshe's heart
has chosen.''} Therefore, such a clause
plays the role of an adjective and
is called ``relative clause.''

English has an abbreviated form of
relative clause, which is called
{\bf\em reduced relative clause}:
{\em ``Entemena, chosen by Nanshe's heart
-- governor of Lagash -- has built the
temple of Dugru for Ningirsu.''}
Sumerian also has reduced relative clauses,
as shown in the present inscription.

\verb||\\
\begin{tabular}[!h]{l l l l l l l l l}
\fcm 𒂗𒋼𒈨𒈾 &\fcm 𒊮𒅆𒊒𒁕 &\fcm 𒀭𒀏𒆤\\
(tr en-te-me-na) & (tr sha3 pad3 da) & (tr nanshe ke4)\\
\multicolumn{3}{l} {(tr en-te-me-na sha3 pad3 da nanshe ke4)}\\
\multicolumn{3}{l} {\em Entemena, chosen by Nanshe's heart,}\\
\end{tabular}\\

\sepstars

\verb||\\
\begin{tabular}[!h]{l l l l l l l l l}
\fcm 𒉺𒋼𒋛 &\fcm 𒉢𒁓𒆷𒆠 &\fcm 𒅗 &\fcm 𒆤\\
(tr ensi2) & (tr lagash ki) & (tr ka) & (tr ke4)\\
\multicolumn{4}{l} {(tr ensi2 lagash ki ka ke4)}\\
\multicolumn{4}{l} {\em the governor of Lagash,}\\
\end{tabular}\\

\sepstars

\verb||\\
\begin{tabular}[!h]{l l l l l l l l l}
\fcm 𒀭𒎏𒄈𒋢𒊏 &\fcm 𒀊𒂁𒊒 &\fcm 𒈬𒈾𒆕\\
(tr an ningirsu ra) & (tr esh3 dug ru) & (tr mu na du3)\\
\multicolumn{3}{l} {(tr an ningirsu ra esh3
  dug ru mu na du3)}\\
\multicolumn{3}{l} {\em he built the shrine
   of Dugru for Ningirsu. }\\
\end{tabular}\\
\index{Reduced relative clause}



\chapter{Relative clause}
\begin{quotation}\fsm\onehalfspacing
\noindent
𒌨𒀭𒇉\\
𒈗𒋀𒀊𒆠𒈠\\
𒈗𒆠𒂗𒄀𒆠𒌵\\
𒇽𒂍𒀭𒂗𒆤𒇲𒅔𒆕𒀀\\
\end{quotation}
Translation: {\em Ur-Nammu, the king of Ur,
  the king of Sumer and Akkad,
  the man who built the temple of Enlil. }

\newpage
\section{Annotations}
\noindent
\begin{tabular}[!h]{l l l l l l l l l}
\fcm 𒌨𒀭𒇉 &\fcm 𒈗 &\fcm 𒋀𒀊𒆠 &\fcm 𒈠\\
(tr ur-nammu) & (tr lugal) & (tr urim5) & (tr ma)\\
\multicolumn{4}{l} {(tr ur-nammu lugal urim5 ma)}\\
\multicolumn{4}{l} {\em Ur-Nammu, the king of Ur,}\\
\end{tabular}\\

\sepstars

\verb||\\
\begin{tabular}[!h]{l l l l l l l l l}
\fcm 𒈗 &\fcm 𒆠𒂗𒄀 &\fcm 𒆠𒌵\\
(tr lugal) & (tr ki-en-gi) & (tr ki uri)\\
\multicolumn{3}{l} {(tr lugal ki-en-gi ki uri)}\\
\multicolumn{3}{l} {\em  the king of Sumer and Akkad,}\\
\hline\\
\multicolumn{3}{l}{{\fcm 𒆠𒂗𒄀}  (ki-en-gi) Sumer}\\
\multicolumn{3}{l}{{\fcm 𒆠𒌵}  (ki uri) Akkad}\\
\end{tabular}\\
\index{ki-en-gi {\fcn 𒆠𒂗𒄀} ! Sumer}
\index{ki uri {\fcn 𒆠𒌵} ! Akkad}

\sepstars

\verb||\\
\begin{tabular}[!h]{l l l l l l l l l}
\fcm 𒇽 &\fcm 𒂍 &\fcm 𒀭𒂗𒆤𒇲 &\fcm 𒌦𒆕𒀀\\
(tr lu2) & (tr e2) & (tr en-lil2 la2) & (tr un du3 a)\\
\multicolumn{4}{l} {(tr lu2 e2 en-lil2 la2 un du3 a)}\\
\multicolumn{4}{l} {\em the man who built the temple
   of Enlil.}\\
\hline\\
\multicolumn{4}{l} {{\fcm 𒇽} (lu2) man}\\
\multicolumn{4}{l} {{\fcm 𒀭𒂗𒆤𒇲} (en-lil2)
  the god Enlil}\\
\end{tabular}\\
\index{lu2 {\fcn 𒇽} ! man }
\index{en-lil2 {\fcn 𒀭𒂗𒆤} ! the god Enlil}

\sepstars

\section{Relative clause}
\index{Relative clause}
In a Sumerian relative clause,
there are two elements. The first element
is the head noun, LU2 ({\fcn 𒇽} ),
the person who built the temple.
The second element is a verbal phrase,
which is transformed into an adjective
by the suffix A ({\fcn 𒀀} ). There is
no need for a relative pronoun,
such as ``who'' or ``that.''
The verbal phrase follows the noun directly.

Pay attention to an important point:
the relative clause ends
in the adjective-forming A-morpheme ({\fcn 𒀀} ).
After all, relative clauses are adjectives.

\chapter{The Finite Verb}
Let us analyze the chapter about
the {\bf finite verb} in Marie-Louise
Thomsen's {\em The Sumerian Language}.
This exercise will show the reader
how to cope with a book where all
examples are given in transliterate
form, without sumerograms.

According to Thomsen, the finite form
is a verbal construction with a
prefix chain and infix pronouns.
It has three conjugations: the intransitive
conjugation, the transitive hamtu
and the transitive marû conjugation.
The components of the finite form
are given below.

\index{Finite form}
\begin{itemize}
\item Modal Prefixes (MP) -- nu {\fcn 𒉡},
  ba ra {\fcn 𒁀𒊏} , na {\fcn 𒈾}, ga {\fcn 𒂵},\\
  ha {\fcn 𒄩}, sha {\fcn 𒊭}, u {\fcn 𒌋}
\item Conjugation Prefixes (CP) -- i3 {\fcn 𒉌},
  ga {\fcn 𒂵} , mu {\fcn 𒈬} , ba {\fcn 𒁀} , bi {\fcn 𒁉} 
\item Pronominal Prefixes -- e/a {\fcn 𒂊} /{\fcn 𒀀},
  ?n {\fcn 𒌦}, ?b {\fcn 𒀊}
\item Verbal stem
\item ed (e)/(de3)/
\item Pronominal Suffixes -- en {\fcn 𒂗},
  e {\fcn 𒂊}, enden {\fcn 𒂗𒉈𒂗},\\
  en ze2 en {\fcn 𒂗𒍢𒂗}, esh2 {\fcn 𒂠},
  en ne {\fcn 𒂗𒉈}
\item Syntactic suffix {\bf\em -a-} {\fcn 𒀀}
\item Pospositions -- e {\fcn 𒂊} , ra {\fcn 𒊏} ,
  ta {\fcn 𒋫} , da {\fcn 𒁕}, etc.
\end{itemize}

\section{Intransitive and Transitive verbs}
In principle, the Sumerian verbal root is
neither transitive nor intransitive,
but neutral concerning transitivity.
The best way to decide about the
transitivity is to count the number
of participants in the action.
If there is only one participant,
one must interpret the verb as
intransitive. Two participants
indicate that the verb is transitive.\\

\verb||\\
{\fcm 𒉽𒅊𒉣𒈨𒉈𒇽𒆠𒂖𒍪𒀊𒉣𒆠𒆠𒂠𒅎𒈠𒉌𒅔𒆭𒆭} 

\noindent
\begin{tabular}[!h]{l l l l l l l l l}
  \fcn 𒉽𒅊𒉣𒈨𒉈 &\fcn 𒇽𒆠𒂖 &\fcn 𒍪𒀊𒉣𒆠𒆠𒂠
  &\fcn 𒅎𒈠𒉌𒅔𒆭𒆭\\
isimud4 de3 & lu2 ki sikil & abzu eridu ki she3 & im ma ni in ku4 ku4\\
\multicolumn{4}{l} {(tr isimud4 de3 lu2 ki sikil abzu
  eridu ki she3 im ma ni in ku4 ku4)}\\
\multicolumn{4}{l} {\em Isimud made the girl enter Abzu Eridu. }\\
\end{tabular}\\

\verb||\\
\begin{tabular}[!h]{l l l l l l l l l}
\fcm 𒆠𒂖 &\fcm 𒍪𒀊𒉣𒆠𒆠𒂠 &\fcm 𒉌𒆭\\
(tr ki sikil) & (tr abzu eridu ki she3) & (tr i3 ku4)\\
\multicolumn{3}{l} {(tr ki sikil abzu eridu ki she3 i3 ku4)}\\
\multicolumn{3}{l} {\em The girl entered Abzu Eridu. }\\
\end{tabular}\\
\index{ki sikil {\fcn 𒆠𒂖}  ! girl}
\index{lu2 ki sikil {\fcn 𒇽𒆠𒂖}  ! girl}

\section{One participant conjugation}
Below, is the complete conjugation
of the intransitive
verb {\bf ku4} {\fcn 𒆭}.\\


\noindent
\begin{tabular}[!h]{l l l l l l l}
  I entered & {\fcm 𒂷𒂊𒉌𒆭𒊑𒂗}\\
  &  (ĝa2 e i3 ku4 re en)\\
  you entered & {\fcm 𒍝𒂊𒉌𒆭𒊑𒂗}\\
  & (za e i3 ku4 re en)\\
  he entered & {\fcm 𒀀𒉈𒉌𒆭}\\
  & (a ne i3 ku4)\\
  the man entered & {\fcm 𒇽𒉌𒆭}\\
  & (lu2 i3 ku4) \\
  we entered & {\fcm 𒈨𒂗𒉈𒂗𒉌𒆭𒊑𒂗𒉈𒂗}\\
  & (me en de3 en i3 ku4 re en de3 en)\\
  you entered (pl.) & {\fcm 𒈨𒂗𒍢𒂗𒉌𒆭𒊑𒂗𒍢𒂗}\\
  & (me en ze2 en i3 ku4 re en ze2 en)\\
 they entered & {\fcm 𒀀𒉈𒉈𒉌𒆭𒊒𒂠} \\
  & (a ne ne i3 ku4 ru esh2)\\
\end{tabular}
\index{ku4 (ku4) ! to enter}


\section{Two-participant conjugation}
In the two-participant hamtu conjugation,
the 1st-singular person
has no subject mark and the 1st-plural
has only a suffix.\\

\noindent
\begin{tabular}[!h]{l l l l l l l l}
  I raised the head. & {\fcm 𒂷𒂊𒊕𒉌𒌈𒍣} \\
  & (ĝa2 e saĝ i3 ib2 zig3)\\
  You raised the head. & {\fcn 𒍝𒂊𒊕𒈨𒂊𒍣}  \\
  The man raised the head. & {\fcm 𒇽𒂊𒊕𒅔𒍣}  \\
  & (lu2 e saĝ in zig3)\\
  The ox raised the head. & {\fcm 𒄞𒂊𒊕𒌈𒍣} \\
  & (gud e saĝ ib2 zig3)\\
  We raised the head. & {\fcm 𒈨𒂗𒉈𒂗𒊕𒉌𒌈𒍣𒄀𒂗𒉈𒂗} \\
  & (me en de3 en saĝ i3 ib2 zig3 ge en de3 en)\\
  You raised the head. & {\fcm 𒈨𒂗𒍢𒂗𒊕𒈬𒂊𒍣𒄀𒂗𒍢𒂗} \\
 & (me en ze2 en saĝ mu e zig3 ge en ze2 en) \\
  They raised the head. & {\fcm 𒀀𒉈𒉈𒊕𒅔𒍣𒄀𒂠}   \\
  & (a ne ne saĝ in zig3 ge esh2)\\
\end{tabular}
\index{saĝ {\fcn 𒊕}  ! head}
\index{gud {\fcn 𒄞} ! bull}
\index{zig3 {\fcn 𒍣}   ! to rise}
  
\section{Three participant construction}
In English, three participant constructions have
the form: {\em X caused Y to attack Z}.
In this pattern, there is an underlying
two-participant sentence,
which is {\em Y attacked Z}.
In Sumerian, the subject of the
underlying two-participant sentence
is marked with the dative postposition
{\fcn 𒊏}  (ra). The verbal chain
references to this dative with
{\fcn 𒉌}  (ni) for the 3rd-singular and
{\fcn 𒊑}  (ri) for the 2nd-singular person.


\verb||\\
\begin{tabular}[!h]{l l l l l l l l l}
\multicolumn{3}{l}{\fsm 𒀭𒂗𒆤𒇷𒃮𒋗𒃻𒉡𒈬𒉌𒌇}\\
&&\\
\fcm 𒀭𒂗𒆤𒇷 &\fcm 𒃮𒋗𒃻 &\fcm 𒉡𒈬𒉌𒌇\\
an en lil2 le & gaba shu ĝar
& nu mu ni tuku\\
\multicolumn{3}{l} {(tr an en lil2 le gaba shu ĝar
  nu mu ni tuku)}\\
\multicolumn{3}{l} {\em Enlil did not
  let him have a rival. }\\
\hline\\
\multicolumn{3}{l}{{\fcm 𒃮𒋗𒃻}  (gaba shu ĝar) rival}\\
\multicolumn{3}{l}{{\fcm 𒌇}  (tuku) to have, to acquire}\\
\end{tabular}\\
\index{tuku {\fcn 𒌇}  ! to have, to acquire}
\index{gaba shu ĝar {\fcn 𒃮𒋗𒃻}  ! rival}

\verb||\\
\begin{tabular}[!h]{l l l l l l l l l}
\multicolumn{3}{l}{\fsm 𒆳𒊑𒃮𒋗𒃻𒉆𒈬𒊑𒅔𒌇𒌦 }\\
&&\\
\fcm 𒆳𒊑 &\fcm 𒃮𒋗𒃻 &\fcm 𒉆𒊑𒅔𒌇𒌦\\
kur re & gaba shu ĝar
& nam ri in tuku un\\
\multicolumn{3}{l} {(tr kur re gaba shu ĝar
  nam ri in tuku un)}\\
\multicolumn{3}{l} {\em I will not let you
   have a rival in the mountains.}\\
\end{tabular}\\


\verb||\\
\begin{tabular}[!h]{l l l l l l l l l}
  \multicolumn{5}{l}{\fsm 𒍝𒂊𒈨𒂗𒅗𒈬𒀭𒆠𒀀𒃮𒊑𒉡𒁀𒂊𒉌𒌇}\\
&&&&\\
  \fcm 𒍝𒂊𒈨𒂗 &\fcm 𒅗𒈬 &\fcm 𒀭𒆠𒀀
  &\fcm 𒃮𒊑 &\fcm 𒉡𒁀𒂊𒉌𒌇\\
  za e me en & inim ĝu10 & an ki a
  & gaba ri & nu ba e ni tuku\\
  \multicolumn{5}{l} {(tr za e me en inim ĝu10
    an ki a gaba ri nu ba e ni tuku)}\\
\multicolumn{5}{l} {\em You did not let my word
   have a rival in heaven and earth.}\\
\end{tabular}\\
\index{inim {\fcn 𒅗}  ! word}
\index{gaba//ri {\fcn 𒃮}  // {\fcn 𒊑}  ! to equal}

\section{The imperfective suffix -ed-}
Many scholars believe that {\fcn 𒂊} (e)
is a marû stem formant. People who do not accept
this theory introduced a prefix /-ed-/ to
account for occurrences of {\fcn 𒂊}  (-e)
in the 3rd person
singular marû of intransitive verbs.
If a vowel follows the \verb|-ed-| prefix,
the /d/ is realized, but
the  /e/ is omitted.
Otherwise, only the /e/ is expressed.\\
\begin{quotation}
\noindent
{\fcm 𒉌𒀄𒉈𒂗}\\
i3 zahh3 de3 en\\
I will escape.\\

\verb||\\
{\fcm 𒉌𒀄𒂊}\\
i3 zahh3 e\\
He will escape.
\end{quotation}

The sentence below consists just of a verbal chain.
The verb {\fcn 𒂃}  (du8 -- to hold, to detain)
is marked with the imperfective suffix /-ed-/.
The prefix {\fcn 𒁀𒊏}  (ba ra) expresses
a strong negative statement.
\begin{quote}
  {\fcm 𒁀𒊏𒁀𒂃𒉈}\\
  (tr ba ra ba du8 de3)\\
  I will not hold her back.
\end{quote}



\printindex
\end{document}

